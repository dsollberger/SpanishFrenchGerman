\subsection{Adverbs 2}

\begin{center}\begin{tabular}{r|l}
  \textbf{Deutsch} & \textbf{English} \\
	\hline
	weder \rule{1cm}{0.4pt} noch \rule{1cm}{0.4pt} & neither \rule{1cm}{0.4pt} noch \rule{1cm}{0.4pt} \\
	sowohl \rule{1cm}{0.4pt} als auch \rule{1cm}{0.4pt}& both \rule{1cm}{0.4pt} and \rule{1cm}{0.4pt} \\
	bereits & already \\
	selbst & even \\
	damit & with it \\
	dar{\"u}ber & about it \\
	allein & alone \\
	nun & so \\
	dabei & with me \\
	dazu & for it \\
	selber & yourself \\
	zuerst & [at] first (e.g. ``Mit dem Kopf zuerst'' == ``Head first'') \\
	zuletzt & [at] last \\
	Genauso wie damals? & Just like then? \\
	daher & hence \\
	Sonst noch etwas? & Something else? \\
	einmal & once (e.g. ``einmal im Monat'' == monthly) \\
	jedenfalls & in any case \\
	v{\"o}llig & entirely \\
	durchaus & definitely \\
	mindestens & at least \\
	meistens & usually \\
	au{\ss}erdem & also/furthermore/in addition \\
	besonders & especially \\
	kaum & barely \\
	{\"u}berhaupt & at all (i.e. to add emphasis) \\
	jedoch & however \\
	erneut & once again \\
\end{tabular}\end{center}


\pagebreak
\subsection{Nature 2}

\begin{center}\begin{tabular}{r|l}
  \textbf{Deutsch} & \textbf{English} \\
	\hline
	das Wasser & water \\
	\Red{die Wiese} & meadow \\
	\Blue{der Strand} & beach \\
	\Red{die Umwelt} & environment \\
	\Blue{der Bernstein} & amber \\
	\Red{die welt} & world \\
	\Red{die Tierwelt} & fauna/animal kingdom \\
	\Blue{der See} & lake \\
	das Gras & grass \\
	\Red{die Welle} & wave \\
	das All & [outer] space \\
	\Blue{der Wald} & forest \\
	%w{\"a}ldern & 
	\Blue{der Fluss} & river (e.g. ``Der Rhein ist ein Fluss.'') \\
	\Blue{der Bach} & creek/brook/stream \\
	\Red{die Pflanze} & plant \\
	das Klima & climate \\
	\Blue{der Strom} & river/power/stream (Ein Strom ist ein gro{\ss} Fluss.) \\
	\Red{die Landwirtschaft} & agriculture \\
	\Red{die W{\"u}ste} & desert \\
\end{tabular}\end{center}


\pagebreak
\subsection{Genitive Case}

\begin{itemize}
  \item  Er ist der Sohn meiner Schwester. \\ He is my sister's son.
	\item  Das ist die letzte Stunde des Tages. \\ This is the last hour of the day.
	\item  Das ist die Dame des Hauses. \\ That is the lady of the house.
	\item  Er ist der Vater ihrer Freundin. \\ Her is her friend's father.
	\item  Wie ist der Name deiner Lehrerin. \\ What is your teacher's name?
	\item  Das ist das Auto deiner Schwester. \\ That is your sister's car.
	\item  Ich mag keinen dieser H{\"u}te. \\ I do not like these hats.
	\item  Er tr{\"a}gt den Hut seiner Freundin. \\ He is wearing his friend's hat.
	\item  Willst du eines dieser B{\"u}cher? \\ Do you want one of these books?
	\item  Freitag ist der f{\"u}nfte Tag der Woche. \\ Friday is the fifth day of the week.
	\item  Das ist ein Pferd. \\ That is a horse.
	\item  Ein Sechstel reicht. \\ A sixth is enough. \\
	\item  Der Hund gibt einem Mann einen Apfel. \\ The dog gives an apple to a man.
	\item  Wir zeigen einem Kind die Katze. \\ We are showing the cat to the child.
	\item  Er gibt einer Dame den Apfel. \\ He gives the apple to a lady.
	\item  Ich zeige einem Kind meinen Schuh. \\ I am showing a child my shoe.
	\item  Das ist der Weg des Herzens. \\ That is the way of the heart.
\end{itemize}


\pagebreak
\subsection{Occupations 2}

\begin{center}\begin{tabular}{r|l}
  \textbf{Deutsch} & \textbf{English} \\
	\hline
	\Blue{der Verfasser (die Verfasseren)}, \Red{die Verfasserin (die Verfasserinnen)} & writer/author(s) \\
	\Blue{der Betreiber (die Betreiberen)}, \Red{die Betreiberin (die Betreiberinnen)} & operator(s) \\
	\Blue{der Entwickler (die Entwickleren)}, \Red{die Entwicklerin (die Entwicklerinnen)} & developer(s) \\
	das Personal & staff/personnel \\
	\Blue{der Empf{\"a}nger (die Empf{\"a}ngeren)}, \Red{die Empf{\"a}ngerin (die Empf{\"a}ngerinnen)} & recipient/addressee(s) \\
	\Blue{der Teilnehmer (die Teilnehmeren)}, \Red{die Teilnehmerin (die Teilnehmerinnen)} & participant(s) \\
	\Red{die Werkstatt} & workshop \\
	das Handwerk & profession/craft/handiwork \\
	\Blue{der Schneider (die Schneideren)}, \Red{die Schneiderin (die Schneiderinnen)} & tailor(s) \\
	\Blue{der Hersteller (die Herstelleren)}, \Red{die Herstellerin (die Herstellerinnen)} & manufactuer/producer(s) \\
	die Spezialit{\"a}t & specialty \\
	\Blue{der Expert (die Experten)}, \Red{die Expertin (die Expertinnen)} & expert/professional(s) \\
	\Blue{der Richter (die Richteren)}, \Red{die Richterin (die Richterinnen)} & judge(s) \\
  \Blue{der Rechtsanwalt (die Rechtsanw{\"a}lte)}, \Red{die Rechtsanwaltin (die Rechtsanw{\"a}ltinnen)} & lawyer(s) \\
\end{tabular}\end{center}


\pagebreak
\subsection{Perfect}

The Perfekt is used to describe past events. In spoken German, the Perfekt is preferred over the Pr{\"a}teritum. Using the Pr{\"a}teritum in normal conversation may sound unnatural or pretentious.  There are a few exceptions to this rule of thumb. sein (to be), haben (to have), wissen (to know), and the modal auxiliaries d{\"u}rfen (to be allowed to), k{\"o}nnen (to be able to), m{\"u}ssen (to have to), sollen (to be supposed to), wollen (to want to) are used in the Pr{\"a}teritum in informal contexts as well.  In contrast to the English present perfect, the German Perfekt is not used to describe events that started in the past and are still ongoing. In such cases, German speakers use the present tense, e.g. I have been living here for three years translates to Ich lebe seit drei Jahren hier.

\subsubsection{How is the Perfekt formed?}

The Perfekt is formed by combining a conjugated form of haben (to have) or sein (to be) in the present tense with the past participle of the main verb.  The vast majority of verbs take haben. Verbs that take sein have to be intransitive, i.e. they can't take an object, and they have to indicate a change of position or condition. sein (to be), bleiben (to stay), and passieren (to happen) take sein even though they don't indicate a change of position or condition.\footnote{For complete conjugation charts, check out Canoo and Verbix.}

\begin{itemize}
  \item  In order to form the past participle of a weak verb, add the prefix ge- and the suffix -t or -et to the stem, e.g. machen (to do/to make) becomes ge-mach-t.
\end{itemize}

\begin{center}\begin{tabular}{r|l}
  \textbf{Pr{\"a}sens} & \textbf{Perfekt} \\
	\hline
	ich mache (I do/make) & ich habe gemacht (I have done/made) \\
	du machst (you do/make) & du hast gemacht (you have done/made) \\
	er/sie/es macht (he/she/it does/makes) & er/sie/es hat gemacht (he/she/it has done/made) \\
	wir machen (we do/make) & wir haben gemacht (we have done/made) \\
	ihr macht (you do/make) & ihr habt gemacht (you have done/made) \\
	sie/Sie machen (they/you do/make) & sie/Sie haben gemacht (they/you have done/made) \\
\end{tabular}\end{center}

\begin{itemize}
  \item  Strong verbs add the prefix ge-, change the stem vowel or the entire stem, and add the suffix -t, -et or -en, e.g. nennen (to call) becomes ge-nann-t, sein becomes ge-wes-en, sprechen (to speak/to talk) becomes ge-sproch-en. These forms are not quite predictable. You need to memorize them.
	\item  A separable prefix will precede the ge- prefix, e.g. aufmachen (to open) becomes auf-ge-mach-t.
	\item  An inseparable prefix will replace the ge- prefix, e.g. veröffentlichen (to publish) becomes ver-öffentlich-t.
	\item  Verbs that end in -ieren will not be prefixed, e.g. informieren becomes informier-t.
\end{itemize}

\pagebreak
\begin{center}\begin{tabular}{r|l||r|l}
  \textbf{Deutsch} & \textbf{English} & \textbf{Deutsch} & \textbf{English} \\
	\hline
	haben & have & gefragt & asked \\
	gegessen & ate & interessiert & interested \\
	gesehen & seen & informiert & informed \\
	gelernt & learned & verletzt & injured \\
	gelesen & read & gegangen & went \\
	geschlafen & slept & gewesen & been \\
	gespielt & played & erz{\"a}hlt & told/recounted \\
	behalten & kept & gerannt & ran \\
	erfahren & found out & getrunken & drank \\
	ver{\"o}ffenlicht & published & gekocht & cooked \\
	beraten & advised & gesprochen & spoken \\
	verbessert & improved & gefunden & found \\
	besucht & visited & gemacht & made/done \\
	gegeben & gave & gesucht & searched/looked \\
	geschrieben & written & gesagt & say/said \\
	geschwommen & swam & gestellt & ordered \\
	gelaufen & ran & verkauft & sold \\
	gekommen & came & vergessen & forgot \\
	gestohlen & stole & bestellt & ordered [food] \\
\end{tabular}\end{center}

Here are some tougher examples:

\begin{itemize}
  \item  Es hat mir sehr gut gefallen. \\ I liked it a lot.
	\item  Ich habe keinen Hunger gehabt.  \\ I have not been hungry.
	%\item  ist \\ [sometimes sets up perfect tense for a strong verb]
	\item  Was ist geschehen? \\ What happenend?
	\item  Er hat das Sofa nicht gemocht. \\ He did not like the sofa.
\end{itemize}


\pagebreak
\subsection{Adjectives:  Nominative 1}

\begin{center}\begin{tabular}{r|l||r|l}
  \textbf{Deutsch} & \textbf{English} & \textbf{Deutsch} & \textbf{English} \\
	\hline
	m{\"u}de & tired & linken & [on the] left \\
	gro{\ss}te(n) & largest & wichtigsten & most important \\
	{\"o}ffentliche(n) & public & normalen & normal \\
	neuen & new & richtige & correct/right \\
	letzter & last & individuellen & individual \\
	spezielle(n) & special & private & private \\
	h{\"o}heren & taller/higher & zuf{\"a}lliger & random \\
	aktuelle & current & besonderen & special \\
	gesamte & whole/entire/total & passende(n) & fitting/matching \\
	internationale(n) & international & weiterer & another \\
	{\"u}brigen & spare & interessante & interesting \\
	einzelnen & individual & jugendliche & teenage  \\
	zus{\"a}tzliche & another & notwendigen & necessary  \\ 
\end{tabular}\end{center}

\begin{itemize}
  \item  Mein \textit{eigener} Hund ist gro{\ss}er als ich. \\ My \textit{own} dog is bigger than me.
	\item  Noct nicht Neues? \\ Nothing new yet?
	\item  Ein Umzug \\ A move
\end{itemize}


\pagebreak
\subsection{Adjectives:  Accusative}

\begin{center}\begin{tabular}{r|l||r|l}
  \textbf{Deutsch} & \textbf{English} & \textbf{Deutsch} & \textbf{English} \\
	\hline
	deutschen & German & modernen & modern \\
	n{\"a}chsten & next & technischen & technical \\
	detaillierte & detailed & alte & old \\
	langen & long & eigene & of my/your/his/her/our/their own \\
	verschiedenen & various & pers{\"o}nlichen & personal \\
	verf{\"u}gbaren & available & vergangenen & former/past/gone by \\
	komplette & entire/whole/complete & einzige & only one \\
	zust{\"a}ndigen & responsible/relevant/appropriate & zweite & second \\
	verschiedene & different/several/various & normalen & normal \\
	starken & strong & gleiche & same \\
	lange & long & kostenlose & free \\
	h{\"o}here & higher & ehemaligen & former/old/previous \\
\end{tabular}\end{center}

\begin{itemize}
  \item  Ich m{\"o}chte ein neues Auto. \\ I would like a new car.
	\item  Ich habe neue Schuhe an. \\ I am wearing new shoes.
	\item  Er mag seine neuen Schuhe. \\ He likes his new shoes.
	\item  Er arbeitet den ganzen Tag. \\ He works all day.
	\item  Wir haben den ganzen Tag gespielt. \\ We have played all day.
	\item  Lest die offenen B{\"u}cher. \\ Read the open books.
	\item  Haben Sie ein weiteres Bad? \\ Do you [formal] have another bathroom?
	\item  Sie haben eine gemeinsame Wohnung. \\ They have an apartment together.
	\item  Wir trinken eine weitere Tasse Kaffee. \\ We are drinking another cup of coffee.
\end{itemize}


\pagebreak
\subsection{Adjectives:  Dative}

\begin{itemize}
  \item  Wir schenken der gesamten Familie Wein. \\ We are giving the entire family wine.
	\item  Wir spielen mit den deutschen Studenten. \\ We play with the German [university] students.
	\item  Am zweiten Januar. \\ On the second of January.
	\item  Die guten alten Zeiten. \\ The good old days.
	\item  Er geht zu der privaten Krankenversicherung. \\ He is going to the private health insurance.
	\item  Am zweiten Tag kommen sie. \\ They are coming on the second day.
	\item  Sprichst du mit der bekannten {\"A}rztin? \\ Are you talking to that famous doctor?
	\item  Wir kochen nur in unserer eigenen K{\"u}che. \\ We only cook in our own kitchen.
	\item  Was passeirt mit Ihren pers{\"o}nlichen Daten? \\ What happens with your personal data?
	\item  Er spricht von dem vongangenen Tagen. \\ He speaks of the days gone by.
	\item  Der Patient spricht mit privaten {\"A}rzten. \\ The patient talks with the private doctors.
	\item  Das ist der Name der heutigen Gemeinde. \\ That is the name of today's community.
\end{itemize}

\pagebreak
\subsection{Directions}

\begin{center}\begin{tabular}{r|l||r|l}
  \textbf{Deutsch} & \textbf{English} & \textbf{Deutsch} & \textbf{English} \\
	\hline
	hin & to & \Red{die Richtung} \rule{1cm}{0.4pt} & [direction toward] \rule{1cm}{0.4pt} \\
	her [``here''] & from & \Blue{der Norden} & the North \\
	rein [``rhine''] & inside & \Blue{der Osten} & the East \\
	herein [``HAAR-rhine''] & [from] inside & \Blue{der Westen} & the West \\
	hierher [``here here''] & this way & \Blue{der S{\"u}den} & South \\
	heraus & out & weg & away \\
	herum & around/about/over & unterwegs & on the way \\
	links & left & zur{\"u}ck & back \\
	rechts & right & voraus & ahead/in front \\
	biegen \rule{1cm}{0.4pt} ab & turn \rule{1cm}{0.4pt} \\
	geht hinaus & is going out \\
	geht raus & is going outside \\
\end{tabular}\end{center}

\begin{itemize}
  \item  Heraus aus meinem Haus! \\ Get out of my house!
  \item  Er rennt im Haus herum. \\ He runs around the house.
  \item  Keiner kommt je hierher. \\ Nobody ever comes this way.
  \item  Hin und her \\ back and forth
  \item  Wir sind unterwegs. \\ We are on the road.
  \item  Er geht weit weg. \\ He is going far away.
\end{itemize}


\pagebreak
\subsection{Adjectives:  Nominative 2}

\begin{itemize}
  \item  Das sind unterschiedliche Tassen. \\ These are different cups.
  \item  Das sind zus{\"a}tzliche Eier. \\ Those are additional eggs.
  \item  Das sind externe Mitarbeiter. \\ These are external employees.
  \item  der h{\"o}chste Berg von Africa. \\ the highest mountain of Africa
  \item  Das sind die politischen Zeitungen. \\ Those are the political newspapers.
  \item  Das ist ein gelber Vogel. \\ That is a yellow bird.
  \item  alle m{\"o}glichen Leute. \\ all sorts of people
  \item  Ich trage einfache Kleidung. \\ I wear simple clothes.
  \item  die neuesten Zeitungen \\ the newest newspapers
  \item  Das sind g{\"u}nstige Ferien! \\ These are cheap vacations!
  \item  Das sind die neuesten Opfer. \\ Those are the newest victims.
  \item  Die guten alten Zeiten! \\ The good old days!
  \item  Der ehemalige Bahnhof ist jetzt ein Restaurant. \\ The former train station is now a restaurant.
  \item  Es sind die gleichen Leute. \\ They are the same people.
\end{itemize}


\pagebreak
\subsection{Adverbs 3}

\begin{center}\begin{tabular}{r|l||r|l}
  \textbf{Deutsch} & \textbf{English} & \textbf{Deutsch} & \textbf{English} \\
	\hline
	daf{\"u}r & for that & wohl & probably \\
	daran & about that & ubrigens & by the way \\
	darauf & on that & davon & away \\
	drin & in that & sogar & even \\
	ansonsten & otherwise & zumindest & at least \\
	eigentlich & actually & wenigstens & at least \\
	darum & because & anders & different \\
	deshalb & because of that & herzlich & cordially \\
	also & therefore & bisher & so far \\
	allerdings & certainly & zugleich & simultaneously \\
	soweit & ready & trotzdem & anyway \\	
\end{tabular}\end{center}

\begin{itemize}
  \item  Nat{\"u}rlich nicht. \\ Of course not.
  \item  Schauen wir mal. \\ Let us see.
  \item  Es geht auch anders. \\ There is another way.
  \item  Bis gleich! \\ See you in a bit!
\end{itemize}


\pagebreak
\subsection{Preterite}

\subsubsection{When is the Pr{\"a}teritum used?}

The Pr{\"a}teritum (also called Imperfekt) is used to describe past events. Its use is mostly limited to formal writing and formal speech. In informal writing and speech, the Perfekt (e.g. Ich habe geschlafen) tends to be preferred. Using the Pr{\"a}teritum in normal conversation may sound unnatural or pretentious.

There are a few exceptions to this rule of thumb. sein (to be), haben (to have), wissen (to know), and the modal auxiliaries d{\"u}rfen (to be allowed to), k{\"o}nnen (to be able to), m{\"u}ssen (to have to), sollen (to be supposed to), wollen (to want to) are used in the Pr{\"a}teritum in informal contexts as well.



\subsubsection{How is the Pr{\"a}teritum formed?}

The Pr{\"a}teritum of regular weak verbs is formed by adding -(e)te, -(e)test, -(e)ten, or -(e)tet to the stem.

\begin{itemize}
  \item  The verb m{\"o}chten (would like to/to want to), which is technically the subjunctive of m{\"o}gen, does not have a preterite form. Instead, the preterite of wollen (to want [to]) is used.
  
  \begin{center}\begin{tabular}{r|l}
    \textbf{Present} & \textbf{Pr{\"a}teritum} \\
    \hline
    ich will (I want) & ich wollte (I wanted) \\
    du willst (you want) & du wolltest (you wanted) \\
    er/sie/es will (he/she/it wants) & er/sie/es wollte (he/she/it wanted) \\
    wir wollen (we want) & wir wollten (we wanted) \\
    ihr wollt (you want) & ihr wolltet (you wanted) \\
    sie/Sie wollen (they/you want) & sie/Sie wollten (they/you wanted) \\
  \end{tabular}\end{center}
  
  \item  The Pr{\"a}teritum of strong verbs is not quite predictable. They usually change the stem and add -st, -en, -t, or no ending at all.
  
  \begin{center}\begin{tabular}{r|l}
    \textbf{Present} & \textbf{Pr{\"a}teritum} \\
    \hline
    ich finde (I find) & ich fand (I found) \\
    du findest (you find) & du fandest (you found) \\
    er/sie/es findet (he/she/it finds) & er/sie/es fand (he/she/it found) \\
    wir finden (we find) & wir fanden (we found) \\
    ihr findet (you find) & ihr fandet (you found) \\
    sie/Sie finden (they/you find) & sie/Sie fanden (they/you found) \\
  \end{tabular}\end{center}
  
  \begin{center}\begin{tabular}{r|l}
    \textbf{Present} & \textbf{Pr{\"a}teritum} \\
    \hline
    ich bin (I am) & ich war (I was) \\
    du bist (you are) & du warst (you were) \\
    er/sie/es ist (he/she/it is) & er/sie/es war (he/she/it was) \\
    wir sind (we are) & wir waren (we were) \\
    ihr seid (you are) & ihr wart (you were) \\
    sie/Sie sind (they/you are) & sie/Sie waren (they/you were) \\
  \end{tabular}\end{center}
\end{itemize}

\begin{itemize}
  \item  Ich sah es mit meiner Frau. \\ I watched it with my wife.
  \item  Wir a{\ss}en Fisch zu Mittag. \\ We ate fish at lunch.
  \item  Das war knapp. \\ That was close.
  \item  Er war vorher zu Hause. \\ He was at home.  (That is, think of ``vorher'' as `before now'.)
  \item  Sie ging und sprach dabei. \\ She walked and talked at the same time.
  \item  Er gab Auf. \\ He gave up.
  \item  Er nahm ihre Hand. \\ He took her hand.
  \item  Ich dachte sofort an dich. \\ I thought of you immediately.
  \item  Wie dachtest du dar{\"u}ber? \\ What did you think of that?
  \item  Wir lasen viele B{\"u}cher. \\ We read many books.
  \item  Ich verlor den Schl{\"u}ssel. \\ I lost the key.
  \item  Er schlief auf dem Tisch. \\ He slept on the table.
  \item  Der Wind lie{\ss} nach. \\ The wind died down.
  \item  Er ist st{\"a}rker als je zuvor. \\ He is stronger than ever before.
  \item  Ich konnte letzte Nacht nicht schlafen. \\ I could not sleep last night.
  \item  Ich stand dort. \\ I stood there.
  \item  Ich wollte dir Gl{\"u"}ck w{\"u}nschen. \\ I wanted to wish you luck.
  \item  Ich zeigte auf ihn. \\ I pointed at him.
\end{itemize}


\subsection{Weather}

\begin{center}\begin{tabular}{r|l||r|l}
  \textbf{Deutsch} & \textbf{English} & \textbf{Deutsch} & \textbf{English} \\
	\hline
	das Wetter & weather & nass & wet \\
	\Blue{der Regen} & rain & trocken & dry \\
	regnen & to rain & \Red{die Wolke} & cloud \\
	\Blue{der Schnee} & snow & \Blue{der Blitz} & lightning \\
	schneien & to snow & \Blue{der Donner} & thunder \\
	\Blue{der Grad} & degree & \Blue{der Regenbogen} & rainbow \\
	\Blue{der Sturm} & storm & \Blue{der Regenschirm} & umbrella \\
	scheinen & to shine & das Gewitter & thunderstorm (i.e. collection of rain) \\
\end{tabular}\end{center}

\begin{itemize}
  \item  das Auge des Sturms \\ the eye of the storm
\end{itemize}

\pagebreak
\subsection{Objects 2}

\begin{center}\begin{tabular}{r|l||r|l}
  \textbf{Deutsch} & \textbf{English} & \textbf{Deutsch} & \textbf{English} \\
	\hline
	\Red{die Schere} & scissors & \Red{die Stelle} & position \\
	\Blue{der Katalog} & catalog & \Blue{der Boden} & bottom \\
	\Blue{der Plan} & plan & das Geschenk & present/gift \\
	\Red{die Sache} & matter & das Zubeh{\"o}r & equipment \\
	das Produkt & product & das Paket & package \\
\end{tabular}\end{center}

\begin{itemize}
  \item  Ich esse ein St{\"u}ck Kuchen. \\ I am eating a piece of cake.
\end{itemize}


\pagebreak
\subsection{Communications 1}

Believe it or not, people still use landline phones, especially in business contexts. A (tele)phone can be a cellphone or a landline phone. The word (tele)phone is to the word cellphone what the word pet is to the word dog, i.e. generic vs. specific.
\begin{itemize}
  \item  the tele(phone) = das Telefon
  \item  the cellphone (the mobile phone) = das Handy / das Mobiltelefon
  \item  Nicht jedes Telefon ist ein Handy.
\end{itemize}
Regardless of whether you always refer to your cellphone as a phone, in this course, you will not be able to use (tele)phone/Telefon and cellphone/Handy interchangeably.

\begin{center}\begin{tabular}{r|l||r|l}
  \textbf{Deutsch} & \textbf{English} & \textbf{Deutsch} & \textbf{English} \\
	\hline
	\Red{die Kommunikation} & communication & \Blue{Der Artikel} & article \\
	rufen & to call & \Red{die Presse} & press \\
	das Gespr{\"a}ch & conversation & \Red{die Nachrict} & news \\
	\Red{die Vorwahl} & area code & \Blue{der Dialog} & dialogue \\
	\Blue{der Computer} & computer & \Blue{der Fernseher} & television ("remote sight") \\
	\Blue{der Monitor} & monitor & das Fernsehen & television ("remote sight") \\
	\Red{die Tastatur} & keyboard & \Red{die Rede} & speech \\
	das Kabel & cable (cord) & \Red{die Diskussion} & discussion \\
	\Red{die Information(en)} & information & \Red{die Medien} & media \\
	\Red{die Festplatte} & hard drive ("fixed plate") & \Red{die Sendung} & program/broadcast \\
	\Red{die Zeitschrift} & magazine & das Medium & medium \\
	das Interview & interview \\
\end{tabular}\end{center}

\begin{itemize}
  \item  Ihr steht in der Zeitung. \\ You are in the newspaper (i.e. not literally).
\end{itemize}


\pagebreak
\subsection{Future 2}

\begin{itemize}
  \item  Ich werde hier nachfragen. \\ I will inquire here.
  \item  Er wird dir einen Stuhl anbieten. \\ He will offer a chair to you.
  \item  Sie wird es vorschlagen. \\ She will suggest it.
  \item  Wir werden es abbrechen. \\ We will cancel it.
  \item  Werden Sie sparen. \\ Will you save money?
  \item  Der Autor wird uns zitieren. \\ The author will quote us.
  \item  Wir werden mehr Zeit ben{\"o}tigen. \\ We will require more time.
  \item  Wir werden den Artikel auf seinem Computer herunderladen. \\ We will download the article on his computer.
  \item  Wir werden den Garten erweitern. \\ We will expand the garden.
  \item  Sie wird das Auto erwerben. \\ She will acquire the car.
  \item  Du wirst das Haus sichern. \\ You will secure the house.
  \item  Sie wird das Krankenhaus bald verlassen. \\ She will leave the hospital soon.
  \item  Ich werde es mir seblst verdeinen. \\ I will earn it myself.
\end{itemize}


\pagebreak
\subsection{Internet and Social Media}

\begin{center}\begin{tabular}{r|l||r|l}
  \textbf{Deutsch} & \textbf{English} & \textbf{Deutsch} & \textbf{English} \\
	\hline
	das Netz & net & \Blue{der Anhang} & attachment \\
	das Foto & photo & \Red{die Blog} & blog \\
	\Red{die Suche} & search & das Profil & profile \\
	\Red{die Seite} & site & aktualisiert & updated \\
	l{\"a}den & to load & \Red{die Taste} & key (on a keyboard) \\
	das Internet & internet & \Blue{der Kommentar} & comment \\
	das WLAN & Wi-Fi & das Passwort & password \\
	\Red{die Internetseite} & webpage & das Programm & program \\
	schicken & to send & l{\"a}den \rule{1cm}{0.4pt} hoch & to upload \rule{1cm}{0.4pt} \\
	\Red{die E-mail} & e-mail & l{\"a}den \rule{1cm}{0.4pt} herunter & to download \rule{1cm}{0.4pt} \\
	das Netzwerk & network & l{\"o}schen & to delete \\
	\Red{die Suchmaschine} & search engine \\ 	
\end{tabular}\end{center}

\begin{itemize}
  \item  Ich habe dir etwas geschickt. \\ I have sent you something.
  \item  Drucken Sie es aus. \\ Print it!
  \item  Sie hat ein Bild geteilt. \\ She has shared a picture.
  \item  Was sind die sozialen Fragen? \\ What are the social questions?
\end{itemize}
