\subsection{Past Perfect}

The past perfect is used to describe past events, more specifically events that happened \textit{way} back in the past or any time before another event in the past.

\begin{itemize}
  \item  Strong verbs add the prefix ge-, change the stem vowel or the entire stem, and add the suffix -t, -et or -en, e.g. nennen (to call) becomes ge-nann-t, sein becomes ge-wes-en, sprechen (to speak/to talk) becomes ge-sproch-en. These forms are not quite predictable. You need to memorize them.
\end{itemize}

\begin{center}\begin{tabular}{r|l}
  \textbf{Pr{\"a}sens} & \textbf{Perfekt}  \\
	\hline
	ich laufe (I run) & ich war gelaufen (I had run)  \\
	du l{\"a}ufst (you run) & du warst gelaufen (you had run)  \\ 
	er/sie/es l{\"a}uft (he/she/it runs) & er/sie/es war gelaufen (he/she/it had run) \\
	wir laufen (we run) & wir waren gelaufen (we had run) \\
	ihr lauft (you run) & ihr wart gelaufen (you had run) \\
	sie/Sie laufen (they/you run) & sie/Sie waren gelaufen (they/you had run) \\
\end{tabular}\end{center}

\begin{itemize}
  \item  "Ich hatte ihn schon gesehen." gegen "als er mich sah" \\
  "I had already seen him" versus "when we saw me"
  \item  Nachdem ich Deutschland verlassen hatte, war ich traurig. \\
  After I had left Germany, I was sad.
  \item  Wie hatte sie es genannt? \\ 
  What had she called it?
  \item  Er hatte mich oft seinen besten Sch{\"u}ler genannt. \\
  He had often called me his best student.
\end{itemize}


\pagebreak
\subsection{Education}

\begin{itemize}
  \item  A Student is a university student and a Sch{\"u}ler is a pupil/student at a primary, secondary or high school. Students attending other types of schools such as language or dancing schools may also be called Sch{\"u}ler.
  \item  Careful: a Hochschule is not a high school. It can be a Universit{\"a}t or a Fachhochschule. A Universit{\"a}t is a full research university and a Fachhochschule is a university that focuses on teaching professional skills and does not have the right to run doctoral programs.
  \item  In German, the word Gymnasium refers to a university prep-school. The German for a sports gym is Turnhalle (used by schools and sports clubs) or Fitnessstudio (commercial).
\end{itemize}

\begin{center}\begin{tabular}{r|l||r|l}
  \textbf{Deutsch} & \textbf{English} & \textbf{Deutsch} & \textbf{English} \\
	\hline
	das Leser & reader & \Red{die Klass} & class \\
	\Blue{der Stift} & pen & \Red{die Lehre} & lesson \\
	\Red{die Bildung} & education & das Institut & institute \\
	\Red{die Ausbildung} & training/apprenticeship & \Red{die Akademie} & academy \\
	das Training & training (for animals or sports) & \Red{die Hochschule} & college \\
	\Red{die Note(n)} & note(s) & \Blue{der Kindergarten} & kindergarten \\
	\Red{die Grundschule} & elementary school & \Red{die Pr{\"u}fung} & exam \\
	studieren & to study & \Red{die Erziehung} & education/parenting \\
	das Studium & studies & \Blue{der Fachbereich} & department \\
	\Red{die Forschung} & research & \Red{die Weiterbildung} & advanced/continuing education \\
	\Blue{der Kurs} & course & \Red{die Uni} & university \\
\end{tabular}\end{center}

\begin{itemize}
  \item  Sie macht {\"U}bungen. \\ 
  She exercises.
  \item  Nach den Tests werden wir mehr wissen. \\
  After the tests we will know more.
  \item  Das Seminar macht Spa{\ss}. \\
  The seminar is fun.
  \item  Heute ist kein Unterricht. \\
  There is no class today.
  \item  Erziehung beginnt zu Hause. \\
  Education begins at home.
  \item  Ich {\"u}berlege es mir. \\
  I think about it.
\end{itemize}


\pagebreak
\subsection{Future Perfect}

The future perfect talks about actions that will have been completed in the future. It's used pretty much like the English future perfect, but it's formed slightly differently.  The future perfect consists of the future tense of the auxiliary verb haben or sein, and the past participle of the main verb. 

\begin{itemize}
  \item  Die T{\"u}r wird geschlossen sein. \\
  The door will be closed.
  \item  Wirst du schon gegessen haben? \\
  Will you have already eaten?
  \item  Sie werden gegangen sein. \\
  They will have left.
  \item  Was wird er gesagt haben? \\
  What will be have said?
  \item  Wohin wird sie gegangen sein? \\
  Where will she have gone?
\end{itemize}


\pagebreak
\subsection{Phrases 2}

\begin{itemize}
  \item  Keine Ahnung \\
  No idea
  \item  Verzeihung \\
  Pardon
  \item  Naja, das ist ziemlich schnell. \\
  Well, that is quite fast.
\end{itemize}


\pagebreak
\subsection{Science}

\begin{center}\begin{tabular}{r|l||r|l}
  \textbf{Deutsch} & \textbf{English} & \textbf{Deutsch} & \textbf{English} \\
	\hline
	das Wissen & knowledge & das Element & element \\
	\Red{die Wissenschaft} & science & \Red{die Statistik} & statistics \\
	\Red{die Technologie} & technology & \Red{die Maschine} & machine \\
	\Red{die Theorie} & theory & \Red{die Technik} & technique \\
	\Red{die Definition} & definition & \Red{die Biologie} & biology \\
	\Red{die Kenntnis(se)} & skill(s) & \Blue{der Wissenschaftler}, \Red{die Wissenschaftlerin} & scientist \\
	\Red{die Erfindung} & invention & das Praktikum & internship \\
	\Red{die Energie} & energy & das Lehrbuch & textbook \\
	\Red{die Temperatur} & temperature & \Red{die Chemie} & chemistry ("HEE-me") \\
	\Red{die Physik} & physics & \Red{die Atmosph{\"a}re} & atmosphere \\
	\Blue{der Motor} & motor & \Red{die Analyse} & analysis \\
	achtung! & watch out! & das Gas & gas \\
	\Red{die Strahlung} & radiation & \Blue{der Kunststoff} & plastic \\
	messen & to measure \\
\end{tabular}\end{center}

\begin{itemize}
  \item  Diese Erfindung ist hilfreich. \\
  This invention is helpful.
  \item  Ich arbeite am Motor. \\
  I am working on the motor.
  \item  Was \textbf{misst} die Gruppe? \\
  What is the group measuring?
  \item  Wie steht es mit der Physik? \\
  What about physics?
  \item  Ich habe die Statistik bei mir. \\
  I have the statistics with me.
  \item  Ich habe Angst vor dieser Methode. \\
  I am afraid of this method.
  \item  Der Nachweis fehlt. \\
  The proof (or certificate) is missing.
\end{itemize}


\pagebreak
\subsection{Reflexive Verbs}

German reflexive verbs require the pronoun sich, which is not needed in English. "Er rasiert sich." = "He is shaving."

\begin{itemize}
  \item  Wir befinden uns in einem Wald. \\
  We find ourselves in a forest.
  \item  Aber die Suche lohnt sich. \\
  But the search is worth it.
  \item  Er f{\"u}hlt sich Wohl. \\
  He feels well.
  \item  Sie befindet sich in dem Schlafzimmer. \\
  She is in the bedroom.
  \item  Wir werden uns melden. \\
  We will be in touch.
  \item  Setzen Sie sich! \\
  Sit down!
  \item  Warum freuen wir uns? \\
  Why are we happy?
  \item  Alle freuen sich. \\
  All rejoice.
  \item  Er freut sich dar{\"u}ber. \\
  He is happy about it.
  \item  Er interessiert sich nicht f{\"u}r uns. \\
  He is not interested in us.
  \item  Er ergab sich. \\
  He surrendered.
  \item  Es ergab sich einfach so. \\
  It just came about like that.
  \item  Sie ergab sich nach dem Gespr{\"a}ch. \\
  She surrendered after the conversation.
  \item  Man sieht sich. \\
  See you around.
  \item  Er sorgt sich. \\
  He worries.
  \item  Wir holen uns pizza. \\
  We are going to get pizza.
  \item  Wir werden uns morgen das Hotel anschauen. \\
  We will examine the hotel tomorrow. 
  \item  Ich w{\"u}nsche mir eine Blume. \\
  I wish for a flower.
  \item  Ich erinnere mich nicht! \\
  I don't remember!
  \item  Wir waschen uns die H{\"a}nde mit Seife. \\
  We are washing our hands with soap.
  \item  Ich werde mich anmelden. \\
  I will sign up.
  \item  Wir werden uns eintragen. \\
  We will sign in.
  \item  Ich habe mich verlaufen. \\
  I am lost.
\end{itemize}


\pagebreak
\subsection{Communication 2}

\begin{center}\begin{tabular}{r|l||r|l}
  \textbf{Deutsch} & \textbf{English} & \textbf{Deutsch} & \textbf{English} \\
	\hline
	\Blue{der Brief} & letter & \Red{die Postleitzahl} & post [ZIP] code \\
	\Red{die Adresse} & address & \Red{die Best{\"a}tigung} & confirmation \\
	\Red{die Post} & post office & \Red{die Begr{\"u}ndung} & justification \\
	\Blue{Gr{\"u}se,} & ``Sincerely,'' (to end a letter) & das Thema & topic \\
	\Red{die Notiz} & note & ver{\"o}ffentlichen & to publish \\
	\Red{die Anrede} & salutation & das Radio & radio \\
	\Red{die Einladung} & invitation & \Blue{der Sender} & transmitter \\
	\Red{die Briefmark(en)} & [postage] stamp(s) & \Blue{der Lautsprecher} & (loud)speaker \\
	\Blue{der Briefkasten} & mailbox & \Red{die Anmeldung} & registration/enrollment \\
	\Red{die Postkarte} & postcard & \Red{die Mitteilung} & message \\
	\Blue{der Kontakt(e)} & contact(s) & das Taschenbuch & paperback [book] \\
	senden & to send \\
\end{tabular}\end{center}

\begin{itemize}
  \item  \Blue{Verlag}, Ort und Jahr \\
  Publisher, place and year
  \item  Heute gibt es keine guten \Red{Meldungen} in der Zeitung. \\
  There is no good news in the newspaper today.
\end{itemize}


\pagebreak
\subsection{Business 1}

\begin{center}\begin{tabular}{r|l||r|l}
  \textbf{Deutsch} & \textbf{English} & \textbf{Deutsch} & \textbf{English} \\
	\hline
	das B{\"u}ro & office & \Blue{der Beitrag} & offering \\
	\Red{die Fabrik} & factory & das Management & management \\
	\Blue{der Inhaber} & owner & \Red{die Miete} & rent \\
	\Red{die Mitgliedschaft} & membership & das Dokument(e) & document(s) \\
	\Blue{der Arbeitserlaubnis} & work permit & \Blue{der K{\"a}ufer} & buyer \\
	das Unternehmen & businesses & \Blue{der Verbraucher} & consumer \\
	\Blue{der Kollege} & colleague/coworker & \Blue{der Gewinn} & profit/prize \\
	bieten \rule{1cm}{0.4pt} an & to offer \rule{1cm}{0.4pt} & \rule{1cm}{0.4pt} nimmt zu & \rule{1cm}{0.4pt} is growing \\
	machen \rule{1cm}{0.4pt} Bestellungen & to place \rule{1cm}{0.4pt} orders &  \Red{die Bewerbung} & application \\
	\Red{die Aktie} & share/stock & \Blue{der Job} & job \\
	das Gebot & offer & Gro{\ss}artig! & Terrific! \\
	\Red{die Beratung} & advice/counseling & \Blue{der Anzug} & suit \\
	\Blue{der Wettbewerb} & competition & \Red{die Karriere} & career \\
	das Projekt & project \\
\end{tabular}\end{center}

\begin{itemize}
  \item  Sie ist Mitgleid dieser \Red{Organisation}. \\
  She is a member of this organization.
  \item  Es l{\"a}uft ein Wettbewerb \\
  There is a competition.
  \item  Kauf es nicht! \\
  Do not buy it!
  \item  Der Kauf lohnt sich. \\
  The purchase is worth it.
  \item  Die \Blue{Werte} sind zu alt. \\
  The values are too old.
  \item  Er meint die \Red{Kleinanzeigen}. \\
  He refers to the classified ads.
\end{itemize}


\pagebreak
\subsection{Language}

\begin{center}\begin{tabular}{r|l||r|l}
  \textbf{Deutsch} & \textbf{English} & \textbf{Deutsch} & \textbf{English} \\
	\hline
	\Red{die Sprache} & language & das Konzept & concept \\
	\Red{die Idee} & idea & \Red{die Erl{\"a}rung} & statement \\
	das W{\"o}rterbuch & dictionary (``word book'') & \Blue{der Bereicht} & report \\
	\Red{die Beschreibung} & description & das Kapitel & chapter \\
	\Red{die Schift} & writing & das Franz{\"o}sisch & French (language) \\
	\Red{die Geschichte} & tale & \Blue{der Satz} & sentence \\
	\Blue{der Text(e)} & text & das Verzeichnis & directory \\
	das Handbuch & handbook & \Red{die Zusammenfassung} & summary (``together'' + ``version'') \\
	\Red{die Zustimmung} & consent & \Red{die Unterhaltung} & conversation \\
	\Red{die Meinung} & opinion \\
\end{tabular}\end{center}

\begin{itemize}
  \item  Diese \Red{Bedeutung} ist mir unbekannt. \\
  This meaning is unfamiliar to me.
  \item  Nur ein Wort \\
  Only one word
  \item  Er hat die Geschichte erz{\"a}hlt. \\
  He has told the tale.
  \item  Kein Zeichen davon? \\
  No sign of it?
  \item  Ich las die \Blue{Titel}. \\
  I read the title.
  \item  Deutsch ist schwer. \\
  German is difficult.
  \item  Englisch \textbf{ist} eine internationale Sprache geworden. \\
  English has become an international language.
  \item  Ich will Englisch lernen. \\
  I want to learn English.
  \item  Ist diese \Red{{\"U}bersetzung} richtig? \\
  Is this translation correct?
  \item  Was \textbf{bedeutet} dieses Wort? \\
  What does this word mean?
  \item  Sie ist im \textbf{Begriff} zu gehen. \\
  She is about to leave (perhaps ``She is on the verge of leaving.'')
\end{itemize}


\pagebreak
\subsection{Abstract Objects 1}

\begin{center}\begin{tabular}{r|l||r|l}
  \textbf{Deutsch} & \textbf{English} & \textbf{Deutsch} & \textbf{English} \\
	\hline
	das Problem & problem & \Red{die Empfehlung} & recommendation \\
	\Blue{der Hinweis(e)} & hint(s)/clue(s) & \Red{die Kompetenz} & competence \\
	\Red{die L{\"o}sung} & solution & das Ding & thing/object \\
	\Red{die {\"A}nderung} & change/amendment & das Ergebnis(se) & result(s) \\
	\Red{due Form} & form/shape & das Engagement [French] & involvement/engagement \\
	das Ziel & objective & \Red{die Entwicklung} & development \\
	das Verhalten & behavior & \Red{die Qualit{\"a}t} & quality \\
	\Red{die Auswahl} & selection & das Ma{\ss}e & measurements \\
	\Blue{der Druck} & pressure & \Red{die Gelegenheit} & occasion \\
	\Blue{der Nutzen} & usage/benefit & \Red{die Lage} & position \\ 
	\Red{die Zusammenarbeit} & collaboration (``together'' + ``work'') \\
\end{tabular}\end{center}

\begin{itemize}
  \item  Auf diese \textbf{Weise} k{\"o}nnen wir helfen. \\
  That is how we can help.
  \item  Das Wort ist nicht l{\"a}nger in \textbf{Gebrauch}. \\
  That word is no longer in usage.
  \item  Diess Wort ist noch in Gebrauch. \\
  This word is still in use.
  \item  Druck das aus. \\
  Print that out.
  \item  Wir k{\"o}nnen auf die gleiche Weise zur{\"u}ck. \\
  We can go back the same way.
  \item  Auf seine Weise \\
  in his own way
  \item  Mehr Augen, mehr \Red{Sicherheit} \\
  More eyes, more safety [German idiom]
  \item  Das ist uns \textbf{wertvoll}. \\
  That is valuable to us.
  \item  Der Bahnhof ist in der \Blue{N{\"a}he}. \\
  The train station is nearby.
  \item  Einen \Blue{Versuch} ist es Wert. \\
  An attempt is worth it.
\end{itemize}


\pagebreak
\subsection{Animals 2}

\begin{center}\begin{tabular}{r|l||r|l}
  \textbf{Deutsch} & \textbf{English} & \textbf{Deutsch} & \textbf{English} \\
	\hline
	das Schaf & sheep & \Blue{der Zoo} & zoo \\
	\Blue{der Schmetterling} & butterfly & \Red{die Schildkr{\"o}te} & turtle \\
	das Eichh{\"o}rnchen & squirrel & \Blue{der Affe} & monkey \\
	bei{\ss}en & to bite & \Blue{der Wal} & whale \\
	\Blue{der Elefant} & elephant & das Huhn & chicken \\
	\Blue{der L{\"o}we} & lion & \Blue{der Delfin} & dolphin \\
	\Blue{der Eisb{\"a}r} & polar bear (literally ``ice'' + ``bear'') & \Blue{der Wolf} & wolf \\
	\Blue{der Pinguin} & penguin \\
\end{tabular}\end{center}

\begin{itemize}
  \item  Der \Blue{Fuchs} hat die \Red{Gans} gestolen. \\
  The fox has stolen the goose.
  \item  Die \Red{Eulen} fligen heute tief. \\
  The owls fly low today.
  \item  Bist du ein S{\"a}ugetier? \\
  Are you a mammal?
  \item  Das Pferd steht neben den Elefanten. \\
  The horse is standing next to the elephants.
  \item  Ich will ein Kamel zu Weihnachten. \\
  I want a camel for Christmas.
  \item  Der \Blue{Hamster} ist verr{\"u}ckt! \\
  The hamster is crazy!
  \item  Ich will eine \Red{Giraffe}, kein Zebra! \\
  I want a giraffe, not a zebra!
  \item  Wann reitest du wieder? \\
  When are you riding again?
  \item  Die \Red{Wespe} sticht den Koch. \\
  The wasp stings the cook.
  \item  \Blue{Tiger} haben \Red{Streifen}. \\
  Tigers have stripes (here, both ``Tiger'' and ``Streifen'' are the same for singular and plural forms).
  \item  \Red{K{\"u}he} und Schafe haben zwei H{\"o}rner. \\
  Cows and sheep have two horns.
  \item  Ist diese \Red{Schlange} gef{\"a}hrlich? \\
  Is this snake dangerous?
  \item  Dieser \Blue{Hai} ist harmlos. \\
  This shark is harmless.
\end{itemize}


\pagebreak
\subsection{Present 3}

\begin{center}\begin{tabular}{r|l||r|l}
  \textbf{Deutsch} & \textbf{English} & \textbf{Deutsch} & \textbf{English} \\
	\hline
	reden & to talk & erreichen & to reach/accomplish \\
	bitten & to request & erh{\"o}hen & to raise \\
	springen & to jump & erscheinen & to appear \\
	melden & to report & erleben & to experience \\
	sichern & to secure & ermitteln & to investigate \\
	zitieren & to cite & bekommen & to get \\
	pr{\"a}sentieren & to present & beantwortet & to respond (to) \\
	definieren & to define & berechnen & to calculate \\
	basieren & to be based (on) & betrifft & to affect \\
	kontactieren & to contact & berichten & to report \\
	wechseln & to exchange & verhindern & to prevent \\
	bewerten & to evaluate & vergleichen & to compare \\
	dienen & to serve & ver{\"a}ndern & to change \\
	verdeint & to earn & vermeiden & to avoid \\
	f{\"o}rdern & to support & entwickelt & to develop \\
	unterst{\"u}tzt & to support & triten & to kick/operate \\
	{\"u}berpr{\"u}fen die Daten & to check & stimmen & to be correct \\
	erwarten & to expect & brennen & to burn \\
	erhalten & to receive & teilen & to share \\
	erm{\"o}glichen & to make possible \\
\end{tabular}\end{center}

\begin{itemize}
  \item  Wir beginnen zu lernen. \\
  We are beginning to learn.
  \item  Es ist einfach f{\"u}r mich dieses Buch zu lesen. \\
  It is easy for me to read this book.
  \item  \textbf{Nennen} Sie einen. \\
  Name one.
  \item  Sie \textbf{merken} nichts. \\
  They did not notice anything.
  \item  Es ist Zeit das zu lernen. \\
  It is time to learn that.
  \item  Wo \textbf{fangen} wir an? \\
  Where do we begin?
  \item  Sie begann zu telefonieren. \\
  She began to talk on the phone.
  \item  \textbf{Markieren} Sie es null! \\
  Mark it 'zero'!
  \item  Ich hoffe, dass ich es tun kann. \\
  I hope that I can do it.
  \item  \textbf{Achte} auf deine Ern{\"a}hrung! \\
  \textbf{Pay attention} to your diet!
  \item  Sofort \textbf{wenden}! \\
  \textbf{Turn around} immediately!
  \item  Sie \textbf{bewegen} ihn nach links. \\
  They are \textbf{moving} him to the left.
  \item  Wir vergessen zu schnell. \\
  We forget too quickly.
  \item  Das stimmt! \\
  That is correct!
  \item  Lass uns {\"A}pfel essen. \\
  Let us eat apples.
  \item  Vielleicht wird er ein guter Lehrer. \\
  Perhaps he will be a good teacher.
\end{itemize}


\pagebreak
\subsection{Body 2}

Note that for body parts, some plural forms in German are the same as the singular words.

\begin{center}\begin{tabular}{r|l||r|l}
  \textbf{Deutsch} & \textbf{English} & \textbf{Deutsch} & \textbf{English} \\
	\hline
	\Red{die Lung(e)}  & lung(s) & \Blue{der Daumen} & thumb(s) \\
	\Red{die Zunge} & tongue & das Knie & knee(s) \\
	\Red{die Lippe(n)} & lip(s) & \Blue{der Zeh(en)}, \Red{die Zehe(n)} & toe(s) \\
	das Gehirn & brain & \Blue{der Ellbogen} & elbow(s) \\
	das Kinn & chin & \Blue{die H{\"u}fte} & hip \\
	\Red{die Stirn} & forehead/brow & \Red{die Frese(n)} & heel(s) \\
	dick & fat & \Red{die Leber} & liver \\
	d{\"u}nn & thin & \Blue{der Muskel} & muscle \\
	\Blue{der Bauch} & belly/abdomen & \Blue{der Knochen} & bone(s) \\
	das Handgelenk & wrist (``hand'' + ``joint'') & \Blue{der Oberschenkel} & thigh \\
	\Blue{der Kn{\"o}chel}& ankle & \Red{die Haut} & skin \\
\end{tabular}\end{center}

Note that \Red{die Brust} translates to the chest or breast exterior, while \Blue{der Brustkorb} translations to the internal chest, ribcage, or thorax (literally ``breast'' + ``basket'').

\begin{itemize}
  \item  Faustregel \\
  rule of thumb (literally: ``rule of fist'')
  \item  Ich dr{\"u}cke dir die \Blue{Daumen}! \\
  I will keep my fingers crossed for you!
  \item  Dein \Blue{Kn{\"o}chel} ist gr{\"u}n. \\
  Your knuckle is bruised [i.e. Germans say ``green'' for `bruised'].
  \item  \Red{Leber} und intestine sind Organe. \\
  Liver and intestines are organs.
  \item  Wie viele \Blue{Knochen} hat eine Katze? \\
  How many bones does a cat have?
\end{itemize}


\pagebreak
\subsection{Future 3}

\begin{itemize}
  \item  Ich werde wandern gehen. \\
  I will go hiking.
  \item  Nein, du wirst das Geld zahlen. \\
  No, you will pay the money.
  \item  Er wird die Texte vergleichen. \\
  He will compare the texts.
  \item  Der Koch wird die K{\"u}che \textbf{vergr{\"o}{\ss}ern}. \\
  The cook will \textbf{enlarge} the kitchen.
  \item  Ich werde Sie \textbf{durchsuchen}. \\
  I will \textbf{search} you.
  \item  Kannst du den Brief morgen \textbf{abgeben}? \\
  Can you \textbf{hand in} the letter tomorrow?
  \item  Die Zentrale wird es verhindern. \\
  The head office will prevent it.
  \item  Wer wird es \textbf{einrichten}? \\
  Who will \textbf{arrange} it?
  \item  Wir werden das Auto bewegen. \\
  We will move the car.
  \item  Das werden wir \textbf{beachten}. \\
  We will \textbf{consider} that.
  \item  Wir werden es als Unfall melden. \\
  We will report it as an accident.
  \item  Wirst du mir deinen Namen nennen? \\
  Will you tell me your name?
  \item  Das Dorf wird brennen! \\
  The village will burn!
  \item  Wirst du die Klinik f{\"o}rdern? \\
  Will you promote the clinic?
  \item  Es wird reichen. \\
  It will be enough.
  \item  Dieses Hotel werde ich nicht mehr buchen. \\
  I will not book this Hotel anymore.
  \item  Wir werden am Sonntag dar{\"u}ber \textbf{diskutieren}. \\
  We will \textbf{discuss} that on Sunday.
\end{itemize}


\pagebreak
\subsection{Spiritual}

\begin{center}\begin{tabular}{r|l||r|l}
  \textbf{Deutsch} & \textbf{English} & \textbf{Deutsch} & \textbf{English} \\
	\hline
	\Red{die Seele} & soul & \Red{die Hoffnung}& hope \\
	\Blue{der Wunder} & miracle & meditieren & to meditate \\
	das Gef{\"u}hl & feeling & \Red{die Wahrheit} & truth \\
	\Red{die Spiritualit{\"a}t} & spirituality & wunderbar & wonderful \\
	\Blue{der Geist} & ghost \\
\end{tabular}\end{center}

\begin{itemize}
  \item  Der sechste \Blue{Sinn} \\
  the sixth \Blue{sense}
  \item  Ich sehe tote Menschen. \\
  I see dead people.
  \item  Du bist gl{\"u}cklich. \\
  You are happy.
  \item  Du bist mein Schicksal. \\
  You are my destiny.
  \item  Sie ist wundersch{\"o}n. \\
  She is simply wonderful.
  \item  Bist du im \textbf{Gleichgewicht}? \\
  Are you in \textbf{balance}?
  \item  Das ist mein \textbf{Leid}! \\
  That is my \textbf{song}!
\end{itemize}


\pagebreak
\subsection{Verbs:  Conditional}

There are several possibilities to express the notion of would in German. We concentrate on the most common and simple one: w{\"u}rden + the infinitive (base form) of a verb.

\begin{center}\begin{tabular}{l|l}
  ich w{\"u}rde spielen & wir w{\"u}rden spielen \\
  \hline
  du w{\"u}rdest spielen & ihr w{\"u}rdet spielen \\
  \hline
  er/sie/es w{\"u}rde spielen & sie/Sie w{\"u}rden spielen \\
\end{tabular}\end{center}

The second possibility is to conjugate the verb directly. In modern language, this form is only common for a few frequent verbs such as sein and haben.

\begin{center}\begin{tabular}{l|l||l|l}
  ich w{\"a}re & wir w{\"a}ren & ich h{\"a}tte & wir h{\"a}tten \\
  \hline
  du w{\"a}rst & ihr w{\"a}rtet & du h{\"a}ttst & ihr h{\"a}ttet \\
  \hline
  er/sie/es w{\"a}ren & sie/Sie w{\"a}ren & er/sie/es h{\"a}tten & sie/Sie h{\"a}tten \\
\end{tabular}\end{center}

\begin{itemize}
  \item  Das w{\"u}rde alle unsere Probleme l{\"o}sen. \\
  That would solve all of our problems.
  \item  \textbf{Falls} ja, wie? \\
  \textbf{If} yes, how?
  \item  Er hat mir gesagt, dass er es nicht tun w{\"u}rde. \\
  He has told me that he would not do it.
  \item  W{\"u}rde er heute so handeln? \\
  Would he act like that today?
  \item  Wir w{\"u}rden es nie erfahren. \\
  We would never experience it.
  \item  Was k{\"o}nnten wir heute Abend essen? \\
  What could we eat tonight?
  \item  So, da \textbf{w{\"a}ren} wir. \\
  Well, here we are.
  \item  Das w{\"a}re gut. \\
  That would be good.
  \item  Wir h{\"a}tten gerne Wasser. \\
  We would like to have water.
  \item  W{\"u}rdest du f{\"u}r mich anfragen? \\
  Would you ask on my behalf?
  \item  Der Stuhl w{\"u}rde langsam brennen. \\
  The chair would burn slowly.
  \item  W{\"u}rdestdu darauf achten? \\
  Would you pay attention to that?
  \item  D{\"u}rfte ich es lesen? \\
  May I read it?
  \item  Niemand w{\"u}rde mich einstellen. \\
  Nobody would hire me.
\end{itemize}


\pagebreak
\subsection{Math}

\begin{itemize}
  \item  Zwei \textbf{plus} zwei ist vier.
  \item  F{\"u}nf \textbf{minus} vier ist eins.
  \item  Vier minus zwei \textbf{gleich} zwei.
  \item  Zw{\"o}lf \textbf{geteilt durch} vier gleich drei.
  \item  Vier plus vier \textit{macht} acht.
  \item  F{\"u}nf und drei \textit{ergibt} acht.
\end{itemize}


\pagebreak
\subsection{Banking}

\begin{center}\begin{tabular}{r|l||r|l}
  \textbf{Deutsch} & \textbf{English} & \textbf{Deutsch} & \textbf{English} \\
	\hline
	\Blue{der Betrag} & amount & \Blue{die Zinsen} & interest \\
	\Red{die M{\"u}nze} & coin & \Red{die Finanzierung} & credit \\
	\Red{die Rechnung} & check (end of meal) & das Konto & account \\
\end{tabular}\end{center}

\begin{itemize}
  \item  Was ist f{\"u}r \Red{die Zahlung} verantwortlich? \\
  Who is responsible for the \Red{payment}?
  \item  Akzeptieren Sie \Red{Kreditkarten}? \\
  Do you accept \Red{credit cards}?
  \item  Es gibt eine \Red{Frist}. \\
  There is a \Red{deadline}.
\end{itemize}


\pagebreak
\subsection{Abstract Objects 2}

\begin{center}\begin{tabular}{r|l||r|l}
  \textbf{Deutsch} & \textbf{English} & \textbf{Deutsch} & \textbf{English} \\
	\hline
	\Red{die Planung} & plan & \Red{die Anleitung} & guide \\
	\Red{die Basis} & basis & \Blue{der Verlauf} & course/process \\
	\Red{die Version} & version & \Red{die Verwendung} & usage \\
	\Red{die Bestimmung} & destiny & \Red{die Rolle} & role \\
	\Red{die Verbesserung} & improvement & \Red{die Spalte} & column/crack \\
	\Red{die Erfahrung} & experience & \Red{die Einf{\"u}hrung} & introduction \\
	\Blue{der Unterschied} & difference & das Gewicht & weight \\
	\Red{die Liste} & list & \Red{die Stufe} & step \\
	das Original & original & \Red{die Wirkung} & effect \\
	\Red{die Kraft} & energy/strength/force & \Red{die Party(s)} & party \\
	\Red{die Kategorie} & category & das Feld & field \\
	\Red{die Unterst{\"u}tzung} & support & \Red{die Einzelheit} & detail \\
	\Red{die Eigenschaft} & characteristic & \Red{die Referenz} & reference \\
	\Blue{der Zugang} & access & \Red{die Kombination} & combination \\
	das Mittel & means/average/remedy & \Blue{der Kreis} & circle/circuit \\
	\Red{die L{\"a}nge} & length & \Blue{der Einsatz} & dedication \\
	\Red{die H{\"o}he} & height/altitude & \Red{die Umsetzung} & implementation \\
	\Red{die Position} & position & \Blue{der Blick} & glance/look/view \\
	\Blue{der Hintergrund} & background & \Blue{der {\"U}berblick} & overview \\
\end{tabular}\end{center}

\begin{itemize}
  \item  Das ist eine gro{\ss}e \Red{Menge} Energie. \\
  That is a large \Red{amount} of energy.
  \item  Die \Red{Vorschl{\"a}ge} sind alt, glaube ich. \\
  I believe the \Red{recommendations} are old.
  \item  Darf ich nach ihrem Gewicht fragen? \\
  May I ask you for your weight?
  \item  Sie sind in gutem \Blue{Zustand}. \\
  They are in good \Blue{shape}.
\end{itemize}


\pagebreak
\subsection{Conditional Perfect}

Conditional Perfect works just as normal Perfect, but uses the conditional form of "haben" instead. So, "Ich habe ihn gesehen" becomes "Ich h{\"a}tte ihn gesehen".  Be aware that in some verbs, such as behalten, verlassen, erfahren, the Participle looks like the Infinitive. Don't let that confuse you, always use the Participle!

\begin{itemize}
  \item  Er h{\"a}tte euch beraten. \\
  He would have advised you.
  \item  Wir h{\"a}tten alles erfahren. \\
  We would have found out everything.
  \item  Du h{\"a}ttest ihn verlassen. \\
  You would have left him.
  \item  Wir h{\"a}tten Sie gerne beraten. \\
  We would have gladly advised you.
  \item  H{\"a}tte er den Hund behalten? \\
  Would he have kept the dog?
  \item  Wir dachten, du h{\"a}ttest uns verlassen. \\
  We thought you would have left us.
  \item  Wir h{\"a}tten es von ihr erfahren. \\
  We would have found it out from her.
  \item  Ohne die Zeitung h{\"a}tten wir es nie erfahren. \\
  Without the newspaper we never would have found out about it.
  \item  Ich h{\"a}tte etwas gesagt. \\
  I would have said something.
  \item  Das w{\"a}re gut gewesen. \\
  That would have been good.
  \item  Sie w{\"a}re gerne gekommen. \\
  She would have liked to have come [innuendo].
  \item  Ich dachte, Sie w{\"a}ren das gewesen. \\
  I thought it had been you.
  \item  Wir w{\"a}ren gerne dabei gewesen. \\
  We would have liked to have been there.
\end{itemize}