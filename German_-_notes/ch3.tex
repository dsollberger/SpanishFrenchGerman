\subsection{Possessive Pronouns}

\subsubsection{Personal Pronouns in the Nominative Case}

A pronoun is a word that represents a noun, like ``er'' does for ``der Mann''" In the nominative case, the personal pronouns are simply the grammatical persons you already know: ``ich'', ``du'', ``er/sie/es'', ``wir'', ``ihr'', ``sie'', and ``Sie''.

\subsubsection{Demonstrative Pronouns in the Nominative Case}

The demonstrative pronouns in English are: this, that, these, and those. In German, the demonstrative pronouns in the nominative case are the same as the definite articles. That means, ``der'', ``die'', and ``das" can also mean ``that (one)'' or ``this (one)'' depending on the gender of the respective noun, and ``die'' can mean ``these'' or ``those''. For example, if you talk about a certain dog, you could say ``Der ist schwarz'' (that one is black).

\begin{center}\begin{tabular}{c|c}
  mein/meine & unser/unsere \\
	\hline
	dein/deine & euer/euere \\
	\hline
	sein/seine/ihr/ihre & seins \\
\end{tabular}\end{center}

\vspace{0.5in}

\begin{center}
  \includegraphics[bb=0 0 640 480]{images/Wassermelone.jpg}
\end{center}


\pagebreak
\subsection{Nominative Pronouns}

\begin{center}\begin{tabular}{l|l}
  \textbf{Deutsch} & \textbf{English} \\
	\hline
	beide/beides/beiden & both \\
	dieser/diese/dies & these \\
	jeder/jede/jedes & each one \\
	manche & some \\
	viele & a lot of \\
	alle & everybody \\
	alles & everything \\
	viel & a lot \\
	niemand & nobody \\
	jemand & anyone \\
	etwas & some \\
	paar & a few \\
\end{tabular}\end{center}

It's ``beide'' when it's `both' in the nominative and accusative without an article like the sentence above, ``Beide m{\"o}gen Kaffee'' or ``Ich mag beide''.

It's ``beiden'' in the dative case when there's no article as in ``Mit beiden H{\"a}nden'' or ``Gib beiden Menschen Kaffee'', etc.

It's also ``beiden'' when it follow the plural definite article as in ``Die beiden sind groß''. It's like that in all cases: Nominative: die beiden Accusative: die beiden Dative: den beiden Genitive: der beiden

And then there's ``beides''

``beide'' is used in the plural, for countable objects:

``Soll ich Mama oder Papa holen?'' - ``Beide sollen kommen!'' ``Das Radio und das Grammophon standen im Freien, beide wurden nass vom Regen.''

``beides'' is used in the singular, for uncountable objects or abstract objects:

``Magst du Ketchup oder Mayo auf die Pommes?'' - ``Beides muss drauf!'' ``Was machst du lieber: Geschirr sp{\"u}len oder W{\"a}sche b{\"u}geln?'' - ``Das ist beides {\"a}tzend!''

\pagebreak
\subsection{Negatives}

\begin{center}\begin{tabular}{l|l}
  \textbf{Deutsch} & \textbf{English} \\
	\hline
	nicht & not (temporary condition) \\
	kein & not (permanent condition) \\
	keine & not one (feminine) \\
	keinen & not any \\
	keiner & nobody \\
	\Red{Die Steuer} & tax \\
	nie & never \\
	niemals & never ever (i.e. with emphasis) \\
	nichts & nothing \\
\end{tabular}\end{center}


\pagebreak
\subsection{Adverbs 1}

\begin{center}\begin{tabular}{l|l}
  \textbf{Deutsch} & \textbf{English} \\
	\hline
	auch & too \\
	zu & too \\
	so & so \\
	gern & like \\
	gerne & like (either form is accepted) \\
	wirklich & really \\
	noch & still \\
	noch eine & another \\
	nur & only \\
	schon & already \\
	immer & always \\
	genug & enough \\
	in Ordnung & alright \\
	zusammen & together \\
	dran & turn (as in ``It is my turn'') \\
	vielleicht & might [be] \\
	wieder & again \\
	alleine & alone \\
\end{tabular}\end{center}

Examples:
\begin{itemize}
  \item  Ich lerne gern.
  \item  Ich trinke gerne Wein.
\end{itemize}


\pagebreak
\subsection{Places 1}

\begin{center}\begin{tabular}{l|l}
  \textbf{Deutsch} & \textbf{English} \\
	\hline
	das Geb{\"a}ude & building \\
	das Haus (die H{\"a}user) & house (houses) \\
	\Blue{der Markt} & market \\
	\Red{die Schule (die Schulen)} & school (schools) \\
	\Blue{der G{\"a}rten} & garden \\
	\Red{die Ecke} & corner \\
	\Blue{der Bahnhof (die Bahnh{\"o}fe)} & train station (train stations) \\
	das Restaurant (die Restaurants) & restaurant (restaurants) \\
	\Red{die Bibliothek} & library \\
	das Schloss (die Schlosses) & castle (castles) \\
	\Red{die Bank} & bank \\
	bauen & to build \\
	das Land (die L{\"a}nder) & country (countries) \\
	das Hotel & hotel \\
	\Red{die Stra{\ss}e} & street \\
	\Red{die Stadt} & city \\
	das Dorf (die D{\"o}rfer) & village (villages) \\
	\Red{die Br{\"u}cke} & bridge \\
	\Blue{der Park (die Parks)} & park (parks) [noun] \\
	das Zimmer (die Zimmer) & room (rooms) [as in bedroom, living room, etc.] \\
\end{tabular}\end{center}


\pagebreak
\subsection{Stuff}

German is well known for its very long words that can be made up on the go by concatenating existing words. In this skill you will learn one very simple and commonly used way of forming compounds: adding ``-zeug'' (``stuff'') to existing words.

\begin{center}\begin{tabular}{l|l}
  \textbf{Deutsch} & \textbf{English} \\
	\hline
	das Zeug & stuff \\
	das Fahrzeug (die Fahrzeuge) & vehicle (vehicles) \\
	das Flugzeug (die Flugzeuge) & airplane (airplanes) \\
	das Feuerzeug (die Feuerzeuge) & lighter (lighters) \\
	das Spielzeug (die Spielzeuge) & toy (toys) \\
	das Werkzeug (die Werkzeuge) & tool (tools) \\
\end{tabular}\end{center}


\pagebreak
\subsection{Accusative Pronouns}

\subsubsection{Personal Pronouns in the Accusative Case}

Aside from the nominative case, most of the German pronouns are declined in each of the four cases. Like in English, when the subject becomes the object, the pronoun changes. For instance, ``ich'' changes to ``mich'' (accusative object) as in ``Ich sehe mich'' (I see me).

In the accusative case of the third person pronouns, only the masculine gender shows the change, thus neither the feminine ``sie'' nor the neuter ``es'' change. For example, ``Er/Sie/Es mag ihn/sie/es'' (He/She/It likes him/her/it).

\begin{center}\begin{tabular}{l|l}
  \textbf{Nominative} & \textbf{Accusative} \\
	\hline
	ich (I) & mich (me) \\
	du (you, singular informal) & dich (you, singular informal) \\
	\Blue{er (he)} & \Blue{ihn (him)} \\
	sie (she) & sie (her) \\
	es (it) & es (it) \\
	wir (we) & uns (us) \\
	ihr (you, plural informal) & euch (you, plural informal) \\
	sie (they) & sie (them) \\
	Sie (you, formal) & Sie (you, formal) \\
\end{tabular}\end{center}

\subsubsection{Demonstrative Pronouns in the Accusative Case}

Similarly, only the masculine gender shows the change in the demonstrative pronouns: ``der'' (for ``that one'') changes to ``den'', but ``die'' and ``das'' (for ``that one'') remain the same.

The demonstrative pronouns in the accusative case are thus: ``den'' = that one (masculine), ``die'' = that one (feminine), ``das'' = that one (neuter), and for the plural, ``die'' = ``these''. Take this example: ``Er isst den'' is ``He is eating that one (masculine)''; ``Er isst die'' and ``Er isst das'' are both ``He is eating that one'', but for the other two genders.


\pagebreak
\subsection{Household 1}

\begin{center}\begin{tabular}{l|l}
  \textbf{Deutsch} & \textbf{English} \\
	\hline
	{\"o}ffnen & to open \\
	\Red{die Wand} & wall \\
	\Red{die Decke} & blanket (or ceiling) \\
	das Fenster & window \\
	\Red{die T{\"u}r} & door \\
	\Blue{der Zaun (die Z{\"a}une)} & fence \\
	\Red{die Treppe} & staircase \\
	\Red{die Wohnung} & apartment \\
	das Dach & roof \\
	\Blue{der Balkon} & balcony \\
	\Blue{der Schl{\"u}ssel} & key (same word for ``keys'') \\
	\Red{die K{\"u}che} & kitchen \\
	\Blue{der Keller} & basement \\
	das Wohnzimmer & living room \\
	\Blue{der Tisch} & table \\
	\Blue{der Stuhl}& chair \\
	\Red{die Schrank} & wardrobe/locker/cabinet \\
	das Sofa & sofa \\
	das M{\"o}bel & furniture \\
	das Schlafzimmer & bedroom \\
	das Bett & bed \\
	\Blue{der Teppich} & carpet/rug \\
	\Red{die Lampe} & lamp \\
	das Licht & light \\
	\Red{die Steckdose} & power outlet \\
	\Blue{der Ladeger{\"a}t} & charger \\
\end{tabular}\end{center}


\pagebreak
\subsection{Conjunctions}

A conjunction like ``wenn'' (when) or ``jedoch'' (however) connects two parts of a sentence together. In German, conjunctions do not change with the case (i.e. they are not declinable).

Subordinating conjunctions combine an independent clause with a dependent clause; the dependent clause cannot stand on its own and its word order will be different than if it did. For instance, in ``Er ist hungrig, weil er nichts aß'' (he is hungry, because he ate nothing), the clause starting with ``weil'' is the dependent clause, which would be ordered as ``er aß nichts'' (he ate nothing) if it stood by itself.

Coordinating conjunctions form a group of coordinators (like ``und'' = and; ``aber'' = but), which combine two items of equal importance; here, each clause can stand on its own and the word order does not change.

Lastly, correlative conjunctions work in pairs to join sentence parts of equal importance. For instance, ``entweder...oder'' (either...or) is such a pair and can be used like this: ``Der Schuh ist entweder blau oder rot'' (this shoe is either blue or red).

\begin{center}\begin{tabular}{l|l}
  \textbf{Deutsch} & \textbf{English} \\
	\hline
	entweder & either \\
	oder & or \\
	aber & but \\
	denn & because (a comma precedes ``denn'' as a coodinating conjunction) \\
	doch & on the contrary \\
	wenn & if \\
	weil & because (a comma precedes ``weil'' as a subordinating conjunction) \\
	dass & that \\
	da & since (a comma precedes ``da'' as a subordinating conjunction) \\
	obwohl & although (a comma precedes ``obwohl'' as a subordinating conjunction) \\
	solange & as long as \\
	sobald & as soon as \\
	sondern & but rather (a comma precedes ``sondern'' as a coodinating conjunction) \\
\end{tabular}\end{center}


\pagebreak
\subsection{People 1}

\begin{center}\begin{tabular}{l|l}
  \textbf{Deutsch} & \textbf{English} \\
	\hline
	\Red{die Person} & person \\
	\Blue{der Freund} & friend (male, or boyfriend) \\
	\Red{die Freundin} & friend (female, or girlfriend) \\
	\Blue{der Name} & name \\
	\Blue{der Vorname} & first name \\
	das Baby (Babys) & baby (babies) \\
	\Blue{der Mensch} & man (closer to mankind) \\
	\Red{die Leute} & people \\
	\Blue{der Herr} & gentleman \\
	\Red{die Dame} & lady \\
	\Blue{der Erwachsene} & adult \\
	\Blue{der B{\"u}rger} & citizen \\
	das Publikum & audience \\
	\Blue{der T{\"u}rk} & Turk \\
	\Red{die Gruppe} & group/band \\
	\Blue{der Gast} & guest \\
	\Blue{der Vegetarier} & vegetarians (same for plural) \\
	das Geschlecht & gender \\
	\Blue{der Gegner} & opponent/rival/adversary \\
	das Mitglied (die Mitglieder) & member (members) \\
	\Blue{der Anf{\"a}nger} & beginner \\
	\Blue{der Besucher} & visitor (same for plural) \\
	\Blue{der Fan (die Fans)} & fan (supporters) \\
	\Blue{der Feind} & enemy \\
	\Blue{der Nachbar} & neighbor \\
\end{tabular}\end{center}


\pagebreak
\subsection{Questions 2}

\subsubsection{Asking a Question in German With a W-Word}

Six W-questions - Wer (Who), Was (What), Wo (Where), Wann (When), Warum (Why) and Wie (How) - can be asked in German to elicit more than yes/no answers. Two of the six adverbs are declineable (i.e. change with the case), whereas four are not.

\subsubsection{Wer (Who)}

Wer is declinable and needs to adjust to the four cases. The adjustment depends on what the question is targeting.

    If you ask for the subject of a sentence (i.e. the nominative object), wer (who) remains as is: Wer sitzt da? (Who is sitting there?).
    If you ask for the direct (accusative) object in a sentence, wer changes to wen (who/whom). As a mnemonic, notice how wen sounds similar to den in den Apfel. Wen siehst du? (Whom do you see?) - Ich sehe den Sohn (I see the son).
    If you ask for the indirect object, wer changes to wem (who/to whom) and adjusts to the dative case. You could ask Wem hast du den Apfel gegeben? (To whom did you give the apple?) and the answer could be Dem Mann (the man). Notice again how the declined form of wer (wem) sounds like the definite article of all masculine and neuter nouns in the dative case (like dem Mann or dem Kind).
    Lastly, asking about ownership (genitive case), changes wer to wessen (whose). Wessen Schuhe sind das? (Whose shoes are these?) - Das sind die Schuhe des Jungen (These are the boy’s shoes). And notice once again how wessen (of the) and des (of the) include a lot of s-sounds.

\subsubsection{Was (What)}

Similar to the changes made to wer, was will decline depending on the four cases.

    For both the nominative and accusative cases, was remains the same. It is common to ask Wer oder was? (who or what?), if you want to know more about the nominative object and do not know if it is a person (who) or a thing (what). You ask Wen oder was? (who/whom or what?), if you want to know more about the accusative object.
    Was changes to wessen for questions about the genitive object as in Wessen ist sie schuldig? (What is she guilty of?).
    For the dative, was changes to a compount of wo(r) + preposition. For instance, if the verb takes the German preposition an (on/about) as in an etwas denken, you would ask Woran denkt er? (About what is he thinking?). Likewise, hingehen is a verb composed of gehen + hin (go + to) and you would ask Wohin geht sie? (To where is she going?).

\subsubsection{Wo (Where)}

In German, you can inquire about locations in several ways. Wo (where) is the general question word, but if you are asking for a direction in which someone or something is moving, you may use wohin (where to). Look at: Wo ist mein Schuh? (Where is my shoe?) and Wohin kommt dieser Wein? (Where does this wine go?). Furthermore, Wohin is separable into Wo + hin. For example, Wo ist mein Schuh hin? (Where did my shoe go?).

Note that the sound of Wer is similar to Where and that of Wo to Who, but they must not be confused. In other words: the two German questions words Wer (Who) and Wo (Where) are false cognates to English. They mean the opposite of what an English speaker would think.

\subsubsection{Wann (When)}

Wann (when) does not change depending on the case. Wann can be used with conjunctions such as seit (since) or bis (till): Seit wann haben Sie für Herrn Müller gearbeitet? (Since when have you been working for Mr. Müller?) and Bis wann geht der Film? (Till when does the movie last?).

\subsubsection{Warum (Why)}

Warum (why) is also not declinable. Wieso and Weshalb can be used instead of Warum. For an example, take Warum ist das Auto so alt? = Wieso ist das Auto so alt? = Weshalb ist das Auto so alt? (Why is that car so old?).

\subsubsection{Vocabulary}

\begin{center}\begin{tabular}{l|l}
  \textbf{Deutsch} & \textbf{English} \\
	\hline
	was & what (probably for actions) \\
	wo & where \\
	wohin & where to \\
	woher & where ... from \\
	wann & when \\
	warum & why \\
	wieso & why (alternative) \\
	wof{\"u}r & what for \\
	womit & with what \\
	wer & who \\
	welche & which (feminine, singular object) \\
	welcher & which (masculine, singular object) \\
	welches & which (neuter, singular object) \\
	welchen & which (multiple objects) \\
	wessen & whose \\
	wie & what (probably for objects) \\
	wor{\"u}ber & about \\
	wie viel & how much \\
	wie viele & how many \\
	\Red{die Frage} & question (noun) \\
	fragen & to question (verb) \\
	\Red{die Antwort} & answer (noun) \\
	antworten & to answer (verb) \\
\end{tabular}\end{center}


\pagebreak
\subsection{Family 1}

Just like in English, there are informal and formal words for ``mother'', ``father'', ``grandmother'', and ``grandfather''. Note that in German, the difference between formal and informal is a lot more pronounced than in English. The informal terms are pretty much only used within your own family.

\begin{center}\begin{tabular}{l|l}
  \textbf{Formal} & \textbf{informal} \\
	\hline
	\Red{die Mutter (mother)} & \Red{die Mama (mom)} \\
	\Blue{der Vater (father)} & \Blue{der Papa (dad)} \\
	\Red{die Gro{\ss}mutter (grandmother)} & \Red{die Oma (grandma)} \\
	\Blue{der Gro{\ss}vater (grandfather)} & \Blue{der Opa (grandpa)} \\
\end{tabular}\end{center}


\begin{center}\begin{tabular}{l|l}
  \textbf{Deutsch} & \textbf{English} \\
	\hline
	\Red{die Schwester (die Schwestern)} & sister (sisters) \\
	\Blue{der Bruder (die Br{\"u}der)} & brother (brothers) \\
	\Red{die Totcher} & daughter \\
	\Blue{der Sohn} & son \\
	\Red{die Famile} & family \\
	die Eltern & parents (no singular form) \\
	das Geschwister & sibling (same for plural) \\
	\Blue{der Partner} & partner (no plural form) \\
	\Red{die Beziehung} & relationship \\
	die Gro{\ss}eltern & grandparents (no singular form) \\
	\Blue{der Enkel} & grandson \\
	\Red{die Enkelin} & granddaughter \\
\end{tabular}\end{center}
