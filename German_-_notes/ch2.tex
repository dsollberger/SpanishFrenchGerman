\subsection{Plurals}

[already explaned back in `Basics 2']


\pagebreak
\subsection{Adjectives}

\subsubsection{Predicate adjectives}

Predicate adjectives, i.e. adjectives that don't precede a noun, are not inflected.
\begin{itemize}
  \item  Der Mann ist gro{\ss}.
	\item  Die M{\"a}nner sind gro{\ss}.
	\item  Die Frau ist gro{\ss}.
	\item  Die Frauen sind gro{\ss}.
	\item  Das Haus ist gro{\ss}.
	\item  Die H{\"a}user sind gro{\ss}.
\end{itemize}
As you can see, the adjective remains in the base form, regardless of number and gender.

\subsubsection{Vocabulary}

\begin{center}\begin{tabular}{l|l}
  \textbf{Deutsch} & \textbf{English} \\
	\hline
	perfekt & perfect \\
	schlecht & bad (poor) \\
	frei & free \\
	normal & normal \\
	toll & great \\
	leicht & light \\
	laut & loud \\
	leise & quiet \\
	lang & long \\
	schwach & weak \\
	klar & clear \\
	klein & small \\
	rund & round \\
	gro{\ss} & big \\
	ruhig & calm \\
\end{tabular}\end{center}


\pagebreak
\subsection{Negation}

There are different ways to negate expressions in German (much like in English you can use ``no'' in some cases, and ``does not'' in others). The German adverb ``nicht'' (not) is used very often, but sometimes you need to use ``kein'' (not a).

\subsubsection{Nicht}

Use ``nicht'' in the following five situations:
\begin{enumerate}
  \item  Negating a noun that has a definite article like ``der Raum'' (the room) in ``Der Architekt mag den Raum nicht'' (the architect does not like the room).
	\item  Negating a noun that has a possessive pronoun like ``sein Glas'' (his glass) in ``Der Autor sucht sein Glas nicht.'' (the writer is not looking for his glass).
	\item  Negating the verb: ``Sie trinken nicht'' (They/You do not drink).
	\item  Negating an adverb or adverbial phrase. For instance, ``Mein Mann isst nicht immer'' (my husband does not eat at all times).
	\item  Negating an adjective that is used with ``sein'' (to be): ``Du bist nicht hungrig'' (you are not hungry).
\end{enumerate}

Adverbs go in different places in different languages. You cannot simply place the German adverb ``nicht'' where you would put "not" in English.  The German ``nicht'' will precede adjectives and adverbs as in ``Das Fr�hst�ck ist nicht schlecht'' (the breakfast is not bad) and ``Das Hemd ist nicht ganz blau'' (the shirt is not entirely blue).

For verbs, ``nicht'' can either precede or follow the verb, depending the type of verb. Typically, ``nicht'' comes after conjugated verbs as in ``Die Maus isst nicht'' (the mouse does not eat). In conversational German, the perfect (``Ich habe gegessen'' = ``I have eaten'') is often used to express simple past occurrences (``I ate''). If such statements are negated, ``nicht'' will come before the participle at the end of the sentence: ``Ich habe nicht gegessen'' (I did not eat/I have not eaten).

Finally, ``nicht'' also tends to come at the end of sentences (after direct objects like ``mir'' = ``me'', or after yes/no questions if there is just one conjugated verb). For example, ``Die Lehrerin hilft mir nicht'' (The teacher does not help me) and ``Hat er den Ball nicht?'' (Does he not have the ball?)

\subsubsection{Kein}

Simply put, ``kein'' is composed of ``k + ein'' and placed where the indefinite article would be in a sentence. For instance, look at the positive and negative statement about each noun: ``ein Mann'' (a man) versus ``kein Mann'' (not a/not one man), and ``eine Frau'' versus ``keine Frau''.

``Kein'' is also used for negating nouns that have no article: ``Man hat Brot'' (one has bread) versus ``Man hat kein Brot'' (one has no bread).

\subsubsection{Nicht versus Nichts}

``Nicht'' is an adverb and is useful for negations. On the other hand, ``nichts'' (nothing/anything) is a pronoun and its meaning is different from that of ``nicht''. Using ``nicht'' simply negates a fact, and is less overarching than ``nichts''. For example, ``Der Sch�ler lernt nicht'' (the student does not learn) is less extreme than ``Der Sch�ler lernt nichts'' (the student does not learn anything).

The word ``nichts'' can also be a noun if capitalized (``das Nichts'' = nothingness).

\subsubsection{Vocabulary}

\begin{center}\begin{tabular}{l|l}
  \textbf{Deutsch} & \textbf{English} \\
	\hline
	nicht & not \\
	stark & strong \\
	traurig & sad \\
	gesund & healthy \\
	einfach & simple \\
	lustig & funny \\
	fertig & ready \\
\end{tabular}\end{center}


\pagebreak
\subsection{Questions and Statements}

Questions can be asked by switching the subject and verb. For instance, ``Du verstehst das.'' (You understand this) becomes ``Verstehst du das?'' (Do you understand this?). These kinds of questions will generally just elicit yes/no answers. In English, the main verb ``to be'' follows the same principle. ``I am hungry.'' becomes ``Am I hungry?''. In German, all verbs follow this principle. There's no do-support.

\subsubsection{Vocabulary}

\begin{center}\begin{tabular}{l|l}
  \textbf{Deutsch} & \textbf{English} \\
	\hline
	neu & new \\
	schnell & quickly \\
	langsam & slow \\
	sch{\"o}n & beautiful \\
	wichtig & important \\
	teuer & expensive \\
	weit & far \\
	m{\"u}de & tired \\
	kalt & cold \\
	schwer & difficult \\
	richtig & correct \\
	alt & old \\
	jung & young \\
	schmutzig & dirty \\
	sauber & clean \\
	hoch & high \\
	tief & deep \\
	warm & warm \\
\end{tabular}\end{center}


\pagebreak
\subsection{Present 1}

\subsubsection{How do you `like' things in German?}

Use the verb m{\"o}gen to express that you like something or someone, and use the adverb gern(e) to express that you like doing something.\footnote{What is the difference between gern and gerne?  They are just variations of the same word.  There is no difference in terms of meaning or style.  You can use whichever you like best.}  M{\"o}gen is used for things, animals, and people:
\begin{itemize}
  \item  Ich mag Bier (I like beer)
	\item  Sie mag Katzen (She likes cats)
	\item  Wir m{\"o}gen dich (We like you)
	\item  Ihr m{\"o}gt B{\"u}cher (You like books)
\end{itemize}

M{\"o}gen cannot be followed by another verb.\footnote{The subjunctive form (m{\"o}chten) can be followed by a verb, but ``Ich m{\"o}chte Fu{\ss}ball spielen'' translates as ``I would like to play soccer'' (not ``I like playing soccer'').}  Also, m{\"o}gen is conjugated irregularly:

\begin{center}\begin{tabular}{l|l}
  ich mag & wir m{\"o}gen \\
	\hline
	du magst & ihr m{\"o}gt \\
	\hline
	er/sie/es mag & sie/Sie m{\"o}gen \\
\end{tabular}\end{center}

Gern(e) is used for verbs or activities:
\begin{itemize}
  \item  Ich trinke gern(e) Bier (I like to drink beer)
	\item  Er spielt gern(e) Fu{\ss}ball (He likes to play soccer)
	\item  Wir lesen gern(e) B{\"u}cher (We like to read books)
	\item  Sie schreibt gern(e) Briefe (She likes to write letters)
\end{itemize}

\subsubsection{Vocabulary}

\begin{center}\begin{tabular}{l|l}
  \textbf{Deutsch} & \textbf{English} \\
	\hline
	wollen & to want \\
	machen & to make \\
	spielen & to play \\
	l{\"a}ufen & to walk \\
	gehen & to go \\
	schl{\"a}fen & to sleep \\
	lernen & to learn \\
	lesen & to read \\
	schreiben & to write \\
	sehen & to see \\
	h{\"o}ren & to hear \\
	kennen & to know \\
	bringen & to bring/deliver \\
	fahren & to drive \\
	rennen & to run \\
	denken & to think \\
	schwimmen & to swim \\
	beginnen & to begin \\
	bezahlen & to pay \\
	reicht & is enough\footnote{That is, no additional linking verb is needed with ``reicht''} \\
	w{\"a}schen & to wash \\
	brauchen & to need \\
\end{tabular}\end{center}


\pagebreak
\subsection{Clothing}

\subsubsection{Kleider - dresses or clothes?}

Das Kleid means the dress, and die Kleider means the dresses, but the plural die Kleider can also mean clothes or clothing. In most cases, clothing (or clothes) translates to Kleidung (usually uncountable), but It's important to be aware that Kleider can be used in that sense as well.

\subsubsection{Hose or Hosen?}

Both Hose and Hosen translate to pants (trousers in British English), but they're not interchangeable. The singular Hose refers to one pair of pants, and the plural Hosen refers to multiple pairs of pants.

\subsubsection{Vocabulary}

\begin{center}\begin{tabular}{l|l}
  \textbf{Deutsch} & \textbf{English} \\
	\hline
	tr{\"a}gen & to wear \\
	das Kleid & dress \\
	\Blue{der Rock} & skirt \\
	\Blue{der Hut} & hat \\
	\Blue{die h{\"u}ten} & hats \\
	\Blue{der Schuh} & shoe \\
	\Blue{die Schuhe} & shoes \\
	die Kleidung & clothing \\
	\Red{die Hose} & pants (single pair) \\
	\Red{die Hosen} & pants (multiple pairs) \\
	das Hemd & shirt \\
	\Blue{der Mantel} & coat \\
	\Blue{die M{\"a}ntel} & coats \\
	\Red{die Jacke} & jacket \\
	\Red{die Tasche} & bag \\
	\Blue{der Knopf} & button \\
	\Blue{der Schmuck} & jewelry \\
	\Red{die Kosmetik} & cosmetics \\
	\Blue{der Ring} & ring \\
	passt & fits \\
	\Blue{der Fleck} & stain \\
\end{tabular}\end{center}


\pagebreak
\subsection{Nature}

\subsubsection{Vocabulary}

\begin{center}\begin{tabular}{l|l}
  \textbf{Deutsch} & \textbf{English} \\
	\hline
	\Red{die Natur} & nature \\
	\Red{die Erde} & earth \\
	\Blue{der Berg} & mountain \\
	\Blue{die Berge} & mountains \\
	\Blue{der Baum} & tree \\
	\Blue{die Ba{\"u}me} & trees \\
	\Red{die Luft} & air \\
	\Blue{der Wind} & wind \\
	das Feuer & fire \\
	\Red{die Blum} & flower \\
	das meer & sea \\
	die meere & seas \\
	\Blue{der Moon} & Moon \\
	\Blue{die monds} & moons \\
	\Blue{der Himmel} & sky \\
	\Red{die Sonne} & Sun \\
	\Blue{der Stern} & star \\
	f{\"a}llt (fallen) & falls \\
	lebe & alive \\
\end{tabular}\end{center}