\subsection{Accusative Prepositions}

Dative prepositions always trigger the dative case.  Here they are: 
\begin{center}
  aus, au{\ss}er, bei, gegen{\"u}ber, mit, nach, seit, von, zu
\end{center}

Accusative prepositions always trigger the accusative case.  Here they are: 
\begin{center}
  bis, durch, f{\"u}r, gegen, ohne, um
\end{center}

\subsubsection{Two-Way Prepositions}

Two-way prepositions take the dative case or the accusative case depending on the context.  If there's movement from one place to another, use the accusative case.  If there's no movement or if there's movement within a certain place, use the dative case:
\begin{center}
  an, auf, entlang, hinter, in, neben, {\"u}ber, unter, vor, zwischen
\end{center}

\begin{itemize}
  \item  No movement $\rightarrow$ dative: \\ Ich bin in einem Haus (I am in a house)
	\item  Movement within a certain place $\rightarrow$ dative: \\ Ich laufe in einem Wald (I am running in [within] a forest)
	\item  Movement from one place to another $\rightarrow$ accusative: \\ Ich gehe in ein Haus (I am walking into a house) 
\end{itemize}

\subsubsection{zu Hause and nach Hause}

zu Hause means at home, and nach Hause means home (homewards, not at home). The -e at the end of zu Hause and nach Hause is an archaic dative ending, which is no longer used in modern German, but survived in certain fixed expressions.

\begin{itemize}
  \item  Ich bin zu Hause (I am at home)
	\item  Ich gehe nach Hause (I am walking home)
\end{itemize}

\begin{center}\begin{tabular}{r|l}
  \textbf{Deutsch} & \textbf{English} \\
	\hline
	ohne & without \\
	%um & 
	es geht um & it is about \\
	gegen & against (recall:  Gegner == opponent) \\
	durch & through \\
	f{\"u}r & for \\
	entlang & along (but usually placed at the end of a clause) \\
\end{tabular}\end{center}


\pagebreak
\subsection{Numbers}

\begin{center}\begin{tabular}{r|l}
  \textbf{Deutsch} & \textbf{English} \\
	\hline
	\Red{die Nummer} & number \\
	es gibt & there is (there are) \\
	\Red{die Zahl} & number/figure (kind of like a data statistic) \\
	z{\"a}hlen & to count \\
	\Red{die Summe} & sum \\
	\Red{die H{\"a}lfte} & half \\
	\Red{das Dutzend} & dozen \\ 
	\Red{die Telefonnummer} & telephone number \\
	weniger & less \\
	mehr & more \\
\end{tabular}\end{center}

\vspace{0.5in}
\begin{center}\begin{tabular}{l|l||l|l}
  1 & eins & 11 & elf \\
	2 & zwei & 12 & zw{\"o}lf \\
	3 & drei & 13 & dreizehn \\
	4 & vier & 14 & vierzehn \\
	5 & f{\"u}nf & 15 & f{\"u}nfzehn \\
	6 & sechs & 16 & \Red{sech}zehn \\
	7 & sieben & 17 & \Red{sieb}zehn \\
	8 & acht & 18 & achtzehn \\
	9 & neun & 19 & neunzehn \\
	10 & zehn & 20 & zwanzig \\
\end{tabular}\end{center}


\pagebreak
\subsection{Food 2}

\begin{center}\begin{tabular}{r|l||r|l}
  \textbf{Deutsch} & \textbf{English} & \textbf{Deutsch} & \textbf{English} \\
	\hline
	das Fr{\"u}hst{\"u}ck & breakfast & \Red{die Zitrone} & lemon \\
	fr{\"u}hst{\"u}cken & to have breakfast & das Getr{\"a}nk & beverage \\
	\Red{die Butter} & butter & \Red{die Speisekarte} & menu \\
	\Red{die Marmelade} & jelly & \Red{die Vorspeise} & appetizer \\
	das M{\"u}sli & cereal (muesli) & \Blue{der Hauptgericht} & main course \\
	\Red{die N{\"u}sse} & nuts & \Blue{der Salat} & salad \\
	\Blue{der Honig} & honey & \Blue{der Knoblauch} & garlic \\
	das Mittagessen & lunch & \Red{die Zwiebel} & onion \\
	essen zu Mittag & eating lunch & scharf & hot/spicy/pungent \\
	salzig & salty & \Blue{der Senf} & mustard \\
	das Rezept & recipe & \Blue{der Nachtisch (die Nachtischs)} & dessert (desserts) \\
	\Red{die Tomate} & tomato & das Abendessen & dinner \\
	\Blue{der Pilz (die Pilze)} & mushroom (mushrooms) & essen zu Abend & eating dinner \\
	das H{\"a}hnchen & chicken & \Blue{der L{\"o}ffel} & spoon \\
	schmeckten & to taste & \Red{die Gabel (die Gabeln)} & fork (forks) \\
	kochten & to cook & das Messer & knife \\
	sauer & sour & \Red{die Bohne} & bean \\
	\Red{die Kuche} & cake (no umlaut!) & \Blue{der Empf{\"a}nger} & recipient \\
\end{tabular}\end{center}

%\begin{center}\begin{tabular}{r|l}
  %\textbf{Deutsch} & \textbf{English} \\
	%\hline
	%das Fr{\"u}hst{\"u}ck & breakfast \\
	%fr{\"u}hst{\"u}ckten & to have breakfast \\
	%\Red{die Butter} & butter \\
	%\Red{die Marmelade} & jelly \\
	%das M{\"u}sli & cereal (muesli) \\
	%\Red{die N{\"u}sse} & nuts \\
	%\Blue{der Honig} & honey \\
	%das Mittagessen & lunch \\
	%essen zu Mittag & eating lunch \\
	%salzig & salty \\
	%das Rezept & recipe \\
	%\Red{die Tomate} & tomato \\
	%\Blue{der Pilz (die Pilze)} & mushroom (mushrooms) \\
	%das H{\"a}hnchen & chicken \\
	%schmeckten & to taste \\
	%kochten & to cook \\
	%sauer & sour \\
	%\Red{die Zitrone} & lemon \\
	%das Getr{\"a}nk & beverage \\
	%\Red{die Speisekarte} & menu \\
	%\Red{die Vorspeise} & appetizer \\
	%\Blue{der Hauptgericht} & main course \\
	%\Blue{der Salat} & salad \\
	%\Blue{der Knoblauch} & garlic \\
	%\Red{die Zwiebel} & onion \\
	%scharf & hot (spicy) \\
	%\Blue{der Senf} & mustard \\
	%\Blue{der Nachtisch (die Nachtischs)} & dessert (desserts) \\
	%das Abendessen & dinner \\
	%essen zu Abend & eating dinner \\
	%\Blue{der L{\"o}ffel} & spoon \\
	%\Red{die Gabel (die Gabeln)} & fork (forks) \\
	%das Messer & knife \\
%\end{tabular}\end{center}


\pagebreak
\subsection{Dative Case}

The indirect object in a sentence is called the dative object. The indirect object is the receiver of the direct (accusative) object. For example, ``Frau'' is the indirect (dative) object in ``Das M{\"a}dchen gibt einer Frau den Apfel'' (A girl gives the apple to a woman).

The dative is also used for certain dative verbs such as ``danken'' (to thank) and ``antworten'' (to answer) and with dative prepositions such as ``von'' (by/of) and ``mit'' (with). For example, ``Ich danke dem Koch'' (I thank the cook) or ``Wir spielen mit der Katze'' (We play with the cat).

This case is known as the ``Wem-Fall'' (with whom-case), because to identify the word in the dative case, you have to ask ``With/to whom ...?''

Note that the dative changes all articles for the words, the plural and pronouns. For example, even though ``Frau'' is a feminine noun, it will take the masculine article here to indicate the dative: "Ich danke der Frau" (I thank the woman).

\begin{center}\begin{tabular}{c||c|c|c|c}
  \textbf{Case} & \textbf{Masculine} & \textbf{Feminine} & \textbf{Neuter} & \textbf{Plural} \\
	\hline
	Nominative & \Blue{der} & \Red{die} & das & die \\
	Accusative & \Blue{den} & \Red{die} & das & die \\
	Dative & \Blue{dem} & \Blue{der} & dem & den \\
\end{tabular}\end{center}

\begin{center}\begin{tabular}{c||c|c|c}
  \textbf{Case} & \textbf{Masculine} & \textbf{Feminine} & \textbf{Neuter} \\
	\hline
	Nominative & \Blue{ein} & \Red{eine} & ein \\
	Accusative & \Blue{einen} & \Red{eine} & ein \\
	Dative & \Blue{einem} & \Red{einer} & einem \\
\end{tabular}\end{center}

Some masculine nouns add an -en or -n ending in the dative and in all other cases besides the nominative. For example in the dative, it is ``dem Jungen'' (the boy).

\subsubsection{German Plurals:  The Dative Case}

There are some exceptions when it comes to pluralizing nouns in the dative case.
\begin{itemize}
  \item  As mentioned before, for most German one-syllable nouns, the -e ending will be needed in their plural form. However, in the dative case, the noun always adds an -en ending (and there may be umlaut changes). For ``the hands'', in the dative case it is ``den H{\"a}nden'' and for ``the dogs'' it is ``den Hunden''.
	\item  For most German masculine or neuter nouns, the plural will end in -er with the exception of the dative case: they will end in -ern in the dative case. There may also be umlaut changes. For example, for ``the books'' it is ``den B�chern''. An example sentence would be ``Der Junge lernt mit den B�chern'' (The boy is learning with the books). Or for ``the children'', this would mean ``den Kindern''.
	\item  Whereas most neuter or masculine nouns ending in -chen, -lein, -el, or -er, require no change of the noun in the plural, they end in -n in the dative case. There may be umlaut changes. For example, for ``the windows'' it is ``den Fenstern'' for the dative plural. An example sentence would be: ``Es funktioniert mit den Fenstern'' (It works with the windows). For ``the mothers'', it is ``den M�ttern as in: ``Ich spreche mit den M�ttern'' (I talk with the mothers).
\end{itemize}

\begin{center}\begin{tabular}{r|l}
  \textbf{Deutsch} & \textbf{English} \\
	\hline
	gibten (wir geben) & to give (we give [irregular situation]) \\
	\Red{die Frauen} & women/wives \\
	aus & from \\
	\Blue{der M{\"a}nner} & men/husbands \\
	zeigten (wir zeigen) & to show (we give [irregular situation]) \\
	die kindern & children \\
	sagten & to tell/say \\
\end{tabular}\end{center}


\pagebreak
\subsection{Money}

\textbf{Euro or Euros?}

In German, the singular is Euro and the plural is usually Euro as well. As a rule of thumb, use Euro when talking about a specific amount, e.g. 200 Euro. In some contexts, the form Euros is used as well. For instance, you can say Euros to refer to individual euro coins, an unquantified amount of euros, or euros as opposed to a different currency, e.g. Ich habe hundert Schweizer Franken, aber keine Euros (I have a hundred Swiss francs but no euros). At the end of the day, many native speakers use either plural form regardless of context.

In English, either plural form is perfectly fine. The plural form euro tends to be preferred in the Republic of Ireland, and the plural form euros tends to preferred pretty much anywhere else. Originally, the plural form euro was supposed to be used in official EU documents, but that's no longer the case.

\begin{center}\begin{tabular}{r|l}
  \textbf{Deutsch} & \textbf{English} \\
	\hline
	das Geld (die Gelder) & money \\
	kosten & to cost \\
	das Euro & European Union currency \\
	\Blue{der Cent} & cent \\
	kauften & to buy \\
	\Blue{der Preis} & price \\
	gewinen & to win \\
	\Blue{der Dollar} & dollar \\
	reich & rich/wealthy \\
	besitzen & to own \\
	das Geldautomat & ATM \\
	{\"u}berweisen & to transfer \\
\end{tabular}\end{center}


\pagebreak
\subsection{Dative Pronouns}

Many words change in the dative case. For the third person pronouns, the following are different from the nominative case: the masculine pronoun is ``ihm'' (to him), the feminine is ``ihr'' (to her), the neuter is ``ihm'' (to it), and the plural is ``ihnen'' (to them).

\begin{center}\begin{tabular}{c|c|c}
  \textbf{Nominative} & \textbf{Accusative} & \textbf{Dative} \\
	\hline
	ich & mich & mir \\
	du & dich & dir \\
	er/sie/es & \Blue{ihn}/sie/es & ihm/ihr/ihm \\
	wir & uns & uns \\
	ihr & euch & euch \\
	sie & sie & ihnen \\
	Sie & Sie & Ihnen \\
\end{tabular}\end{center}

This explains why when thanking a female person it is only correct to say ``Ich danke ihr'' (I thank her) and not ``Ich danke sie'' (I thank she).

All four instances of demonstrative pronouns (the three genders and the plural) change in the dative case. For the masculine, the pronoun is ``dem'' (to/with that), for the feminine it is ``der'' (to/with that) and for the neuter it is ``dem'' (to/with that); for the plural it is ``denen'' (to/with them).

\begin{center}\begin{tabular}{r|l}
  \textbf{Deutsch} & \textbf{English} \\
	\hline
	meinem/meiner/meinen & my (then corresponds to object's gender) \\
	das Trinkgeld & tip (at a restaurant) \\
	helfen (er/sie/es hilft) & to help \\
	danken & to thank \\
	geh{\"o}ren & to belong to \\
	folgen & to follow \\
	deinem/deiner/deinen & your (then corresponds to object's gender) \\
	eurem/eurer/euren & your (for direct objects?) \\
	unserem/unserer/unseren & our (then corresponds to object's gender) \\
	seinem/seiner/seinen & their (then corresponds to object's gender) \\
	beiden & both \\
	dieser & these \\
	vielen & many \\
	welchem & which (in a question) \\
	Es geht \rule{1.0cm}{0.4pt} gut & \rule{1.0cm}{0.4pt} is/are well \\
\end{tabular}\end{center}


\pagebreak
\subsection{Family 2}

\begin{center}\begin{tabular}{r|l}
  \textbf{Deutsch} & \textbf{English} \\
	\hline
	\Red{die Tante} & aunt \\
	\Blue{der Onkel} & uncle \\
	\Red{die Nichte} & niece \\
	\Blue{der Neffe} & nephew \\
	die Verwandte & relative (same for plural) \\
	Cousins (Cousinsen) & cousin (pronouns match to gender) \\
	Zwillinge & twin (same for plural?) \\
	\Blue{der Urenkel} & great-grandson \\
	\Red{die Urgro{\ss}mutter} & great-grandmother \\
	\Red{die Partnerschaft} & partnership \\
	\Red{die Hochzeit} & marriage \\
	verheiratet & married \\
	schwanger & pregnant \\
	\Blue{der Halbbruder} & half brother \\
\end{tabular}\end{center}


\pagebreak
\subsection{Dative Prepositions}

\begin{center}\begin{tabular}{r|l}
  \textbf{Deutsch} & \textbf{English} \\
	\hline
	seit & since \\
	von & from \\
	mit & with \\
	bei & by \\
	nach & toward (a specific place) \\
	zu & toward (something, but not toward a specific place) \\
\end{tabular}\end{center}


\pagebreak
\subsection{Body}

\begin{center}\begin{tabular}{r|l}
  \textbf{Deutsch} & \textbf{English} \\
	\hline
	\Blue{der K{\"o}rper} & body \\
	\Blue{der Kopf} & head/mind \\
	das Haare & hair \\
	das Auge & eye \\
	das Ohr (die Ohres) & ear (ears) \\
	\Red{die Nase} & nose \\
	dr{\"u}cken & to shake \\
	\Blue{der Mund} & mouth \\
	\Blue{der Zahn (die Zahne)} & tooth (teeth) \\
	\Blue{der Hals (die Halse)} & neck/throat (necks/throats) \\
	\Blue{der Arm (die Arme)} & arm (arms) \\
	\Red{die Hand} & hand \\
	\Blue{der Finger (die Fingern)} & finger (fingers) \\
	\Red{die Schulter} & shoulder \\
	\Blue{der R{\"u}cken (die R{\"u}ckens)} & back (backs) \\
	\Red{die Brust (die Br{\"u}ste)} & breast (breasts) \\
	\Blue{der Magen} & stomach \\
	das Herz & heart \\
	das Bein (die Beines) & leg (legs) \\
	\Red{die Haut} & skin \\
	\Blue{der Fu{\ss} (die F{\"u}{\ss}e)} & foot (feet) \\
	das Blut & blood \\
	das Gesicht (die Gesichts) & face (faces) \\
\end{tabular}\end{center}


\pagebreak
\subsection{You (Formal)}

There are three ways of saying ``you'' in German. In English, however, ``you'' can be either singular or plural and no distinction is made between formal and informal.

In German, if you are familiar with someone, you use ``du'' (which is called ``duzen'). For example, if you talk to your mother, you would say: ``Hast du jetzt Zeit, Mama?'' (Do you have time now, Mommy?). But if you are not familiar with someone or still wish to stay formal and express respect, you use ``Sie`` (so-called ``siezen''). For example, you would always address your professor like this: ``Haben Sie jetzt Zeit, Herr Smith?'' (Do you have time now, Mr. Smith?) The person who is addressed with a ``Sie'' has to offer you a ``du'' before you can use it. 

You can distinguish the formal ``Sie'' from the plural ``sie'' (they) because the formal ``Sie'' will always be capitalized, but it will remain ambiguous at the beginning of written sentences. For instance, ``Sie sind sch�n.'' can either refer to a beautiful individual or a beautiful group of people. The verbs for ``sie'' (they) and ``Sie'' (you) are conjugated the same. On Duolingo, either should be accepted unless the context suggests otherwise. In real life, there's always context. Don't worry about misunderstandings.

Fortunately, the verb for ``sie'' (she) is different. ``Sie ist sch�n.'' only translates to ``She is beautiful.'' There's no ambiguity.

Lastly, the German ``ihr'' is the informal plural of ``you,'' like in ``Tom und Sam, habt ihr Zeit?'' (Tom and Sam, do you have time?). Duolingo accepts ``you all'' and ``you guys'' for ``ihr'' but not for the more formal ``Sie''.


\pagebreak
\subsection{Travel}

The word Sehensw{\"u}rdigkeit (sight, as in sightseeing) is made up of several meaningful parts:

\begin{center}\begin{tabular}{c|l}
  \textbf{Part} & \textbf{Meaning} \\
	\hline
	sehen & to see \\
	-s- & connecting element \\
	w{\"u}rdig & to be worthy \\
	-keit & noun suffix \\
\end{tabular}\end{center}

Literally Sehensw{\"u}rdigkeit means \textit{something which is worthy to see}.  The connecting element -s- is used to link words together.  
The ending -keit turns an adjective into a noun.  Often the ending of a compound noun is a good indicator for the gender of the noun. For example, if a noun ends in -keit, chances are high that it is feminine (die).

\begin{center}\begin{tabular}{r|l}
  \textbf{Deutsch} & \textbf{English} \\
	\hline
	{\"O}sterreich & Austria \\
	das Auto (die Autos) & car (cars) \\
	Wien & Vienna \\
	\Blue{der Zug (die Z{\"u}ge)} & train (trains) \\
	\Blue{der Bus (die Busse)} & bus (buses) \\
	\Blue{der Urlaub (die Urlaube)} & vacation (vacations) \\
	\Blue{der Pass} & passport \\
	\Blue{der Visum} & visa \\
	Bayern & Bavaria \\
	Afrika & Africa \\
	\Red{die Reise} & trip (vacation) \\
	\Blue{der Zoll (die Z{\"o}lle)} & customs office \\
	das Boot (die Boote) & boat (boats) \\
	\Blue{der Flug(e)} & flight(s) \\
	\Blue{der Taxi} & taxi \\
	das Motorrad & motorcycle \\
	das Fahrrad (die Fahrr{\"a}der) & bicycle(s) \\
	Frankreich & France \\
	Gro{\ss}britannien & Great Britain \\
	\Blue{der Weg} & track (as in a track at a train station) \\
	\Blue{der Stadtplan (die St{\"a}dteplane)} & street map(s) \\
	\Blue{der Tourismus} & tourism \\
	\Blue{der Reisef{\"u}hrer} & travel guide \\
\end{tabular}\end{center}

\begin{center}\begin{tabular}{r|l}
  \textbf{Deutsch} & \textbf{English} \\
	\hline	
	\Blue{der Weg (die Weges)} & method/track/way \\
	Spanien & Spain \\
	das Abenteuer & adventure \\
	\Blue{der Mietw{\"a}gen} & rental car \\
	\Red{die Streck} & route \\
	\Blue{der Verkehr} & traffic \\
	Italien & Italy \\
	\Red{die Bahn} & train system \\
	die Ferien & holidays (neuter case) \\
	\Blue{der Besuch (die besuches)} & visit(s) \\
	\Red{die tour} & tour \\
	wandren & to emigrate (from) \\
	Schweden & Sweden \\
	\Red{die Bushaltesstelle} & bus stop \\
	fliegen & to fly \\
	\Blue{der Reisef{\"u}hrer} & tour guide \\
	\Red{die F{\"a}hre (die F{\"a}hren)} & ferry \\
	Versp{\"a}tung & delay (use with the verb \textit{haben}) \\
\end{tabular}\end{center}


\subsubsection{Nationality}
\begin{itemize}
  \item  Schweiz (country) = Switzerland
  \item  \Blue{der Schweizer (singular, masculine) = the Swiss (man)}
	\item  \Blue{die Schweizer (plural, masculine) = the Swiss (men)}
	\item  \Red{die Schweizerin (singular, feminine) = the Swiss (woman)}
	\item  \Red{die Schweizerinnen (plural, feminine) = the Swiss (women)}
	\item  \Red{die Schweizer Bev�lkerung (singular, feminine) = the Swiss population}
\end{itemize}


\pagebreak
\subsection{Some}

\begin{center}\begin{tabular}{r|l}
  \textbf{Deutsch} & \textbf{English} \\
	\hline	
	irgendwie & somehow \\
	irgendwo & somewhere \\
	irgenwas & something \\
	irgendwer & someone \\
	irgendwann & sometime \\
\end{tabular}\end{center}


\pagebreak
\subsection{Shopping}

\begin{center}\begin{tabular}{r|l}
  \textbf{Deutsch} & \textbf{English} \\
	\hline	
	einkaufen & shopping (use with ``to go'' verb `gehen') \\
	\Red{die B{\"a}ckerei} & bakery \\
	\Blue{der Marktplatz} & market place \\
	\Blue{der Laden (die L{\"a}den)} & shop \\
	das Gesch{\"a}ft (die Gesch{\"a}fts) & store \\
	verkaufen & to sell \\
	Kunde/Kundin/Kunden & customer (by gender) \\
	\Blue{der Supermarkt} & supermarket \\
	das einkaufesw{\"a}gen & shopping cart \\
	\Red{die Kasse} & check out (similar to cashier) \\
	\Red{die T{\"u}te (die T{\"u}ten)} & bag(s) \\
	gratis & free of charge \\
	billig & cheap \\
	\Blue{der Gutschein} & coupon/voucher \\
	das eink{\"a}ufezentren & shopping mall \\
	\Red{die Apotheke} & pharmacy \\
	das Sonderangebot & special offer \\
\end{tabular}\end{center}

\pagebreak
\subsection{Numbers 2}

\begin{center}\begin{tabular}{r|l||r|l}
  1 & eins & 21 & einundzwanzig (``one and twenty'') \\
	2 & zwei & 22 & zweiundzwanzig \\
	3 & drei & 23 & dreiundzwanzig \\
	4 & vier & 24 & vierundzwanzig \\
	5 & f{\"u}nf & 25 & f{\"u}nfundzwanzig \\
	6 & sechs & 26 & \Red{sech}undzwanzig \\
	7 & sieben & 27 & \Red{sieb}undzwanzig \\
	8 & acht & 28 & achtundzwanzig \\
	9 & neun & 29 & neunundzwanzig \\
	10 & zehn & 30 & drei{\ss}ig \\
\end{tabular}\end{center}

%\vspace{0.5in}

\begin{center}\begin{tabular}{r|l}
  30 & drei{\ss}ig \\
	40 & vierzig \\
	50 & f{\"u}nfzig \\
	60 & sechzig \\
	70 & siebzig \\
	80 & achtzig \\
	90 & neunzig \\
	100 & (ein)hundert \\
	200 & zweihundert \\
	300 & dreihundert \\
	1000 & (ein)tausend \\
	2000 & zweitausend \\
	1,000,000 & \Red{eine Million} \\
	1,000,000,000 & \Red{eine Millarde} \\
	1,000,000,000,000 & \Red{eine Billion} \\
\end{tabular}\end{center}

That is, ``billion'' and ``trillion'' are translated quite differently in German!

\begin{center}\begin{tabular}{r|l}
  \textbf{Deutsch} & \textbf{English} \\
	\hline	
	prozent & percent \\
\end{tabular}\end{center}


\pagebreak
\subsection{Colors}

Adjectives are only inflected when they come before a noun.
\begin{itemize}
  \item  Der K{\"a}se ist alt.
	\item  Das ist ein alter K{\"a}se.
\end{itemize}

\subsubsection{Declension classes}

Strong inflection is used:
\begin{itemize}
  \item  When no article is used
	\item  When a quantity is indicated by:
	  \begin{itemize}
		  \item  etwas (some; somewhat), mehr (more)
			\item  wenig- (few), viel- (much; many), mehrer- (several; many), einig- (some)
			\item  a number (greater than one, i.e. with no endings)
			\item  non inflectable phrases: ein paar (a couple; a few), ein bisschen (a bit; a little bit)
		\end{itemize}
\end{itemize}
The adjective endings are the same as the definite article endings, apart from the adjectival ending "-en" in the masculine and neuter genitive singular.

\begin{center}\begin{tabular}{l||c|c|c|c}
  ~ & \textbf{Masculine} & \textbf{Neuter} & \textbf{Feminine} & \textbf{Plural} \\
	\hline
	Nominative & alter & altes & alte & alte \\
	Accusative & alten & altes & alte & alte \\
	Dative & altem & altem & alter & alten \\
	Genitive & alten & alten & alter & alter \\
\end{tabular}\end{center}

\vspace{0.5in}

Mixed inflection is used after:
\begin{itemize}
  \item  indefinite articles ein-, kein-
	\item  possessive determiners mein-, dein-, sein- etc.
\end{itemize}
Nominative and accusative singular endings follow the definite article; all other forms end with "-en".

\begin{center}\begin{tabular}{l||c|c|c|c}
  ~ & \textbf{Masculine} & \textbf{Neuter} & \textbf{Feminine} & \textbf{Plural} \\
	\hline
	Nominative & alter & altes & alte & alten \\
	Accusative & alten & altes & alte & alten \\
	Dative & alten & alten & alten & alten \\
	Genitive & alten & alten & alten & alten \\
\end{tabular}\end{center}

\pagebreak
Weak inflection is used after:
\begin{itemize}
  \item  definite articles (der, die, das, etc)
	\item  derselb- (the same), derjenig- (the one)
	\item  dies- (this/that), jen- (that), jeglich- (any), jed- (every), which decline like the definite article.
	\item  manch- (some), solch- (such), welch- (which), which decline like the definite article.
	\item  alle (all)
\end{itemize}
Five endings in the nominative and accusative cases end with -e, all others with -en.

\begin{center}\begin{tabular}{l||c|c|c|c}
  ~ & \textbf{Masculine} & \textbf{Neuter} & \textbf{Feminine} & \textbf{Plural} \\
	\hline
	Nominative & alte & alte & alte & alten \\
	Accusative & alten & alte & alte & alten \\
	Dative & alten & alten & alten & alten \\
	Genitive & alten & alten & alten & alten \\
\end{tabular}\end{center}

\begin{center}\begin{tabular}{r|l}
  \textbf{Deutsch} & \textbf{English} \\
	\hline	
	\Red{die Farbe} & color \\
	rot & red \\
	gr{\"u}n & green \\
	blau & blue/drunk \\
	gelb & yellow \\
	schwarz & black/illicit \\
	wei{\ss} & white \\
	braun & brown \\
	grau & gray \\
	bunt & colorful \\
	pink & hot pink \\
	rosa & (toned down) pink \\
\end{tabular}\end{center}

Capitalize colors when used as a noun---e.g. Ich mag Gr{\"u}n.

\pagebreak
\subsection{Imperative}

The imperative mood is used to express commands.  There are three different forms.

The first one is used to address one person informally. It is formed by dropping the infinitive ending -en and adding -e. More often than not, this -e ending is dropped, especially in spoken German. This form of the imperative does not include a personal pronoun.

The second one is used to address more than one person informally. It uses the same conjugation as the regular ihr form of the present tense. This form of the imperative does not include a personal pronoun.

The third one is used to address one or more people formally. It uses the same conjugation as the regular Sie form of the present tense. The formal imperative is the only form to include the personal pronoun (Sie). Note that the word order is reversed. The verb always precedes the pronoun. It essentially looks like a question.  For some examples:

\begin{itemize}
  \item  Trink(e) es! = Drink it! (informal, addressing one person)
  \item  Trinkt es! = Drink it! (informal, addressing more than one person)
  \item  Trinken Sie es! = Drink it! (formal, addressing one or more people)
  \item  Sei kein Baby! = Don't be a baby!
  \item  Bleib bei mir! = Stay with me!
\end{itemize}

Some verbs have irregular imperative forms:

\begin{tabular}{l|l|l|l}
  \textbf{infinitive} & \textbf{informal singular} & \textbf{informal plural} & \textbf{formal} \\
  \hline
  lesen (to read) & lies & lest & lesen Sie \\
  geben (to give) & gib & gebt & geben Sie \\
  nehmen (to take) & nimm & nehmt & nehmen Sie \\
  sein (to be) & sei & seid & seien Sie \\
\end{tabular}

\begin{center}\begin{tabular}{r|l}
  \textbf{Deutsch} & \textbf{English} \\
	\hline	
	rufen & to call/shout \\
	lassen & to leave/let \\
	handlen & to act/handle/trade \\
	bleiben & to stay \\
\end{tabular}\end{center}

\pagebreak
\subsection{Occupation}

A Student is a university student and a Sch{\"u}ler is a pupil/student at a primary, secondary or high school. Students attending other types of schools such as language or dancing schools may also be called Sch{\"u}ler.

\subsubsection{Dropping articles}

When talking about your or someone else's profession in sentences such as I'm a teacher or She's a judge, German speakers usually drop the indefinite article (ein/eine). It sounds more natural to say Ich bin Lehrer and Sie ist Richterin than Ich bin ein Lehrer and Sie ist eine Richterin. This rule also applies to students.

If you add an adjective, you can't drop the article. Er ist ein schlechter Arzt (He's a bad doctor) is correct, but Er ist schlechter Arzt is not.

Also note that you can't drop the definite article (der/die/das).

\subsubsection{Male and female variants}

The grammatical gender usually matches the biological sex of the person you're referring to, i.e. the word that refers to a male baker is grammatically masculine, and the word that refers to a female baker is grammatically feminine. In the vast majority of cases, the female variant is formed by simply adding the suffix -in to the male variant, e.g. der B{\"a}cker becomes die B{\"a}ckerin and der Sch{\"u}ler (the pupil) becomes die Sch{\"u}lerin.

The plural of the female variant is formed by adding the suffing -innen to the singular of the male variant, e.g. "die B{\"a}ckerinnen" and "die Sch�lerinnen".

Keep in mind that, in some cases, the plural comes with an umlauted stem vowel. This applies to the female variant as well, e.g. "der Koch" becomes "die K{\"o}che" and "die K{\"o}chin" becomes "die K{\"o}chinnen".

Occupations are described by gender.  For example:
\begin{itemize}
  \item  \Blue{der Lehrer}: male teacher
	\item  \Red{die Lehrerin}:  female teacher
	\item  \Blue{die Lehrer}: multiple male teachers (or a mixture)
	\item  \Red{die Lehrerinnen}:  multiple female teachers
\end{itemize}

\begin{center}\begin{tabular}{r|l}
  \textbf{Deutsch} & \textbf{English} \\
	\hline	
	\Blue{der Berufe (die Berufes)} & job(s) (as in profession or career) \\
	\Blue{der Arzt} , \Red{die {\"A}rztinn} & doctor \\
	\Blue{der B{\"a}cker}, \Red{die B{\"a}ckerin} & baker \\
	\Blue{der Koch}, \Red{die K{\"o}chin} & cook/chef \\
	\Blue{der Sch{\"u}ler}, \Red{die Sch{\"u}lerin} & student/pupil \\
	\Blue{der Professor}, \Red{die Professorin} & professor \\
	\Blue{der Student}, \Red{die Studentin} & university student \\
	\Blue{der Gesch{\"a}ftsf{\"u}hrer}, \Red{die Gesch{\"a}ftsf{\"u}hrerin} & manager/CEO \\
	\Blue{der Arbeit} & job (as in an assignment) \\
	\Blue{der Arbeitgeber}, \Red{die Arbeitgeberin} & employer \\
	\Blue{der Arbeitnehmer}, \Red{die Arbeitnehmerin} & employee \\
	\Blue{der Mitarbeiter}, \Red{die Mitarbeiterin} & employee \\
	\Blue{der Autor}, \Red{die Autorin} & author \\
	\Blue{der Meister}, \Red{die Meisterin} & foreman/master \\
	\Blue{der B{\"u}rgermeister}, \Red{die B{\"u}rgermeisterin} & mayor \\
	\Blue{der Chef}, \Red{die Chefin} & boss \\
	\Blue{der Trainer}, \Red{die Trainerin} & coach/trainer \\
	\Blue{der Senior}, \Red{die Seniorin} & senior citizen \\
	\Blue{der H{\"a}ndler}, \Red{die H{\"a}ndlerin} & merchant/dealer/trader \\
	\Blue{der Bauer}, \Red{die Bauerin} & farmer \\
	\Blue{der Fahrer}, \Red{die Fahrerin} & driver \\
	\Blue{der Verk{\"a}ufer}, \Red{die Verk{\"a}uferin} & salesperson \\
	\Blue{der Achitekt}, \Red{die Achitektin} & architect \\
	\Red{die Bedienung} & service \\
	\Red{die Feuerwehren} & fire department \\
\end{tabular}\end{center}


\pagebreak
\subsection{Prepositions}

\begin{center}\begin{tabular}{r|l}
  \textbf{Deutsch} & \textbf{English} \\
	\hline	
	legen & to put \\
	sitzen & to sit \\
	in & in \\
	auf & on top of (colloquially: ``on'') \\
	{\"u}ber & across/about \\
	unter & under/among \\
	zwischen & between \\
	vor & before (sometimes: in front of) \\
	hinter & behind \\
	neben & next to \\
	%ab & 
	ab und zu & once in a while \\
	w{\"a}hrend & during \\
	am & at (more specifically:  for geographic locations) \\
	an & of \\
	kommen an & to arrive (i.e. conjugate `kommen') \\
	%ums & 
	wegen & because \\
	einschlie{\ss}lich & including \\
	au{\ss}er & except \\
\end{tabular}\end{center}

\pagebreak
\subsubsection{Contractions}

Some prepositions and articles can be contracted.

\begin{center}\begin{tabular}{r|c|l}
  \textbf{preposition + article} & \textbf{contraction} & \textbf{English} \\
	\hline
	an + das & ans & of the \\
	an + dem & am & of the \\
  auf + das & aufs & on top of the \\
  bei + dem & beim & by the \\
  in + das & ins & in the \\
  in + dem & im & in the \\
  hinter + das & hinters & behind the \\
  �ber + das & {\"u}bers & across the \\
  unter + das & unters & under the \\
  von + dem & vom & of the \\
  vor + das & vors & before the \\
  zu + dem & zum & toward the \\
  zu + der & zur & toward the \\
\end{tabular}\end{center}


\pagebreak
\subsection{Materials}

\begin{center}\begin{tabular}{r|l}
  \textbf{Deutsch} & \textbf{English} \\
	\hline	
	ist aus & is made of \\
	\Red{die Plastik} & plaster sculpture \\
	das Plastik & plastic (material) \\
	das Glas & glass \\
	\Blue{der Sand} & sand \\
	\Blue{der Stein} & rock/stone \\
	\Red{die Wolle} & wool \\
	\Blue{der Beton} & concrete \\
	\Red{die Pappe} & cardboard \\
	\Red{die Papier} & paper \\
	das Leder & leather \\
	das Holz (die H{\"o}lzer) & wood \\
	\Red{die Baumwolle} & cotton (``tree wool'') \\
	das Metall & metal \\
	das Gold & gold \\
	das Silber (die Silbers) & silver \\
	das Kupfer & copper \\
	das Eisen & iron \\
	\Blue{der Stahl} & steel \\
	\Red{die Mauer} & wall (outdoors) \\
\end{tabular}\end{center}

\begin{center}
  \includegraphics[scale = 0.67, bb=0 0 640 480]{images/Metall.jpg}
\end{center}


\pagebreak
\subsection{Ordinal Numbers}

German ordinal numbers are pretty regular. The general rule is:
\begin{center}\begin{tabular}{r|l}
  $1 - 19$ & -te \\
	$> 19$ & -ste \\
\end{tabular}\end{center}

There are some irregular situations:
\begin{center}\begin{tabular}{r|l}
  1st & erste \\
	3rd & dritte \\
	7th & siebte \\
	8th & achte \\
\end{tabular}\end{center}

Ordinal numbers behave like adjectives, so their endings will change accordingly:
\begin{itemize}
  \item  Er kennt den ersten S{\"a}nger.
	\item  Er ist am sechsten August geboren.
	\item  Ich bin seine tausendste Lehrerin.
\end{itemize}

\begin{center}\begin{tabular}{r|l}
  \textbf{Deutsch} & \textbf{English} \\
	\hline	
	{\"a}hnlich & similar \\
	Mathematik & mathematics \\
\end{tabular}\end{center}