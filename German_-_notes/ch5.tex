\subsection{Comparison}

\begin{center}\begin{tabular}{r|l}
  \textbf{Deutsch} & \textbf{English} \\
	\hline
	gr{\"o}{\ss}e & bigger/older \\
	kleine & smaller/younger \\
	gro{\ss}e & very \\
	\Red{die Gr{\"o}{\ss}e} & size \\
	langsamer & more slowly \\
	sch{\"o}ner & more beautiful \\
	leichter & easier/lighter \\
	wichtiger & more important \\
	h{\"o}her & higher \\
	besser & better \\
	l{\"a}nger & longer \\
	st{\"a}rker & stronger \\
	teurer & more expensive \\
	als & than \\
	\Red{die Schnecke} & snail \\
	schnelles & fast \\
	schneller & faster \\
\end{tabular}\end{center}

``Easier way to know adjective endings (my teacher side is coming out)! I have 3 rules for being able to add (or recognize) the correct ending when an adjective precedes the noun.
\begin{itemize}
  \item  Big 3 get an -e (der, die, das) der alte Mann, das kleine Kind, die sch�ne Frau
	\item  Changin' gets -en (plural and case changes) den alten Mann (accusative), der sch�nen Frau (dative), die kleinen Kinder (plural)
	\item  No 'the'? Adjective takes over (no 'der' word or just an 'ein') Kaltes Wetter gef�llt mir nicht (das Wetter). Ein guter Mann ist schwer zu finden (der Mann).
\end{itemize}

Now the only tricky part is knowing which 'the' word your noun has'' --- Duolingo user jess1camar1e


\pagebreak
\subsection{Qualifiers}

\begin{center}\begin{tabular}{r|l}
  \textbf{Deutsch} & \textbf{English} \\
	\hline
	gute & good \\
	ganz & entirely \\
	sehr & very \\
	ziemlich & pretty (adverb) \\
	super & great/brilliant/super \\
	eher & rather \\
	beste & best \\
	gew{\"o}hnlich & normally \\
	normalerweise & usually \\
\end{tabular}\end{center}

Perhaps sehr $>$ ziemlich $>$ eher.


\pagebreak
\subsection{House 2}

\begin{center}\begin{tabular}{r|l||r|l}
  \textbf{Deutsch} & \textbf{English} & \textbf{Deutsch} & \textbf{English} \\
	\hline
	das Glas (die Gl{\"a}ser) & glass(es) & das Tor (die Tore) & goal/gate \\
	das Ger{\"a}t (die Ger{\"a}te) & utensil/gadget/instrument/device & \Red{die Sch{\"u}ssel} & bowl/dish \\
	\Red{die Haushaltger{\"a}te} & household appliances & \Blue{der Teller} & plate (same for plural) \\
	\Red{die Heizung} & radiator & \Blue{der K{\"u}hlschrank} & refrigerator (``chill'' + ``cabinet''?) \\
	\Red{die Reinigung} & cleaning & \Blue{der Karton} & box \\
	\Blue{der Schlaf} & sleep & \Blue{der Rucksack} & backpack \\
	\Blue{der Umzug (die Umz{\"u}ge)} & relocation/move/procession & \Red{die Flasche} & bottle \\
	voll & full & \Blue{der Rasierer} & razor (same for plural) \\
	leer & empty & das Bad & bath/bathroom \\
	\Blue{der Haushalt} & household & \Red{die Seife} & soap \\
	\Red{die Zahn{\"u}rste} & toothbrush & \Blue{der Spiegel} & mirror \\
	\Red{die Zahnpasta} & toothpaste & das Handtuche & towel (``hand'' + ``cloth'') \\
	\Blue{der Raum (die R{\"a}ume)} & room/area/space & \Red{die Dusche} & shower \\
	\Red{die Leiter} & ladder & das Duschgel & body wash \\
	\Red{die Batterie} & battery & duschen & to take a shower \\
	\Red{die Toilette} & bathroom & das Shampoo & shampoo \\
	\Red{die Tasse} & cup \\
\end{tabular}\end{center}

Note that while das Handtuch = the towel, das Tuch = the cloth.  A Handtuch is a towel, not a hand towel. Of course, a towel can be a hand towel, but this does not mean that the two words are interchangeable. A pet can be a dog, but this does not mean that the words pet and dog are interchangeable.

\begin{center}
  \includegraphics[bb=0 0 640 480]{images/Spiegel.jpg}
	
	\vspace{0.5in}
	
	Spiegel = mirror
\end{center}


\pagebreak
\subsection{Dates 1}

\begin{center}\begin{tabular}{r|l}
  \textbf{Deutsch} & \textbf{English} \\
	\hline
	\Blue{der Tag (die Tage)} & day(s) \\
	\Blue{der Montag (die Montage) [Mo]} & Monday(s) \\
	\Blue{der Dienstag [Di]} & Tuesday \\
	\Blue{der Mittwoch [Mi]} & Wednesday \\
	\Blue{der Donnerstage (die Donnerstages) [Do]} & Thursday(s) (``thunder'' + ``day'') \\
	\Blue{der Freitag [Fr]} &  Friday \\
	\Blue{der Samstag/Sonnabend [Sa]} & Saturday \\
	\Blue{der Sonntag [So]} & Sunday \\
	\Red{die Woche} & week \\
	bis & up to \\
	w{\"o}chentlich & weekly \\
	das Wochenende & weekend \\
	\Blue{der Werktag} & workday (weekday) \\
	\Red{die Zukunft} & future \\
	t{\"a}glich & daily \\
	\Blue{der Alltag (die Alltages)} & daily routine(s) \\
	\Red{die Vergangenheit} & past/history/past tense \\
	\Blue{der Anfang (die Anf{\"a}nge)} & beginning(s) \\
	das Ende & end \\
	bald & soon \\
	danach & then/afterwards \\
	sp{\"a}testens & at the latest \\
	endlich & finally \\
	inzwischen & by now, up to now, in the meantime \\
\end{tabular}\end{center}

Recall:  ``Bis bald!'' == ``See you soon!''


\pagebreak
\subsection{Adjectives:  Predicative 2}

\begin{center}\begin{tabular}{r|l||r|l}
  \textbf{Deutsch} & \textbf{English} & \textbf{Deutsch} & \textbf{English} \\
	\hline
	komplett & complete & privat & private \\
	fit & fit & sinnvoll & sensible/useful/reasonable \\
	regional & regional/local & knapp & scarce/meager/short \\
	pl{\"o}tzlich & suddenly & verantwortlich & responsible \\
	automatisch & automatic & bereit & ready \\
	hilfreich & helpful & egal & does not matter (regardless) \\
	aktiv & active & offen & open \\
	relativ & relative & germainsam & together \\
	allgemein & general/universal & ausgezeichnet & excellent \\
	extrem & extreme & zust{\"a}ndig & responsible (interchangeable with ``verantwortlich'') \\
	tats{\"a}chlich & indeed & komisch & funny/strange (either way) \\
	original & original & genau & exactly \\
	seltsam & strange/odd/peculiar & fest & firm/hard/tight \\
	wert & worthy & weich & soft \\
	falsch & false/wrong & hart & hard/harshly \\
	beliebt & popular & kaputt & torn/damaged/broken \\
	Echt? & Really? (idiom) \\
\end{tabular}\end{center}

``Gemeinsam sind wir st{\"a}rker`` is the statement ``We are stronger together'' while ``Sind wir st{\"a}rker gemeinsam?'' is a question.  Both settings have the verb first:  ``sind wir''.


\pagebreak
\subsection{Location}

When talking about locations in English, you can use here, there, this, and that to express that something is close or far away. In German the word da is commonly used when talking about locations. The good thing about da is, you don't have to worry about the distance! It can mean anything close or far away.  Let's look at a few examples:  ``Wir sind da'' (We are here/there) or ``Da ist ein Apfel (Here/There is an apple)''.  With hier (here), da (here/there), and dort (there) you can be more specific about the distance.

In colloquial language, you can combine all of them with articles, and use them similar to this and that:  das hier, das da, das dort.  To refer to one specific thing, you can put a noun between the article and hier/da/dort.  For example:  ``Der Apfel da ist gro{\ss}'' (That apple is big) or ``Die Katzen da sind s{\"u}{\ss}'' (Those cats are cute).  Some people might add dr{\"u}ben (over there):  ``Der Apfel da dr�ben ist gro�'' (That apple over there is big) or ``Die Katzen dort dr�ben sind s{\"u}{\ss}'' (Those cats over there are cute).

\begin{center}\begin{tabular}{r|l}
  \textbf{Deutsch} & \textbf{English} \\
	\hline
	oben & upstairs \\
	unten & downstairs \\
	vorne & in front (or ``over there'') \\
	hinten & in the back \\
	nebenan & next door \\
	drinnnen & inside/indoors \\
	drau{\ss}en & outside/outdoors \\
	innen & on the inside (e.g. cooking meat) \\
	au{\ss}en & on the outside \\
	{\"u}berall & everywhere \\
\end{tabular}\end{center}


\pagebreak
\subsection{Adjectives:  Predicative 3}

\begin{center}\begin{tabular}{r|l}
  \textbf{Deutsch} & \textbf{English} \\
	\hline
	notwendig & necessary \\
	verf{\"u}gbar & available \\
	selbstverst{\"a}ndlich & self-evident (``self'' + ``understand'' + adverb form) \\
	eindeutig & clear/unambiguous/definite \\
	begeistert & excited (``be'' + ``spirited'') \\
	zufrieden & content \\
	sichtbar & visible \\
	unsichtbar & invisible \\
	abh{\"a}ngig & dependent \\
	unabh{\"a}ngig & independent \\
	bekannt & known/familiar \\
	unbekannt & unknown \\
	n{\"u}tzlich & useful \\
	m{\"o}glich & possible \\
	unm{\"o}glich & impossible \\
	wahrscheinlich & likely/possibly (``true'' + ``appearance'' + adverb form) \\
	praktisch & practically/convenient \\
	pers{\"o}nlich & personal \\
	positiv & positive \\
	negativ & negative \\
	individuell & distinct/individual \\
	international & international \\
	verpflichtet & obliged (obligated to do something) \\
	kostenlos & free (of charge) \\
	deutlich & clearly \\
\end{tabular}\end{center}


\pagebreak
\subsection{Places 2}

\begin{center}\begin{tabular}{r|l}
  \textbf{Deutsch} & \textbf{English} \\
	\hline
	wohnen & to live \\
	\Blue{der Ort} & town/site/place \\
	\Red{die Kneipe} & pub/bar \\
	\Blue{der Platz} & place/square/yard \\
	\Red{die Pension} & guest house (or bed and breakfast, or inn) \\
	\Red{die Unterkunft} & lodging \\
	\Blue{der Bereich} & field/area \\
	\Blue{der Flughafen} & airport \\
	das Bundesland & province/state (federal state) \\
	\Red{die Region} & region \\
	\Blue{der Bezirk} & district \\
	\Blue{der Standort} & location (industrial site) \\
	gegen{\"u}ber & across the street \\
	\Blue{der Grund} & ground/land/bottom \\
	das Grundst{\"u}ck & property/plot \\
	\Red{die Zentrale} & headquarters \\
	das Zentrum & center (of town) \\
	\Red{die Halle} & hall \\
	\Red{die Fl{\"a}che} & expanse/surface/area \\
	\Blue{der Hof} & courtyard \\
	\Red{die Innenstadt} & downtown (city center) \\
	das Innere & inside \\
	\Red{die Umgebung} & environment \\
	Ausland & abroad (in a foreign country) \Red{Capitalized??} \\
	Europa & Europe \\
	das Ferienhaus & vacation home \\
	\Red{die Heimat} & home/homeland \\
	\Red{die Insel} & island \\
	\Red{die Hauptstadt} & capital city \\
\end{tabular}\end{center}


\pagebreak
\subsection{Medical}

\begin{center}\begin{tabular}{r|l}
  \textbf{Deutsch} & \textbf{English} \\
	\hline
	krank & ill \\
	\Blue{der Patient}, \Red{die Patientin} & patient \\
	\Red{die Medikamente} & drugs/medications \\
	das Pflaster & band-aid/plaster \\
	bluten & to bleed \\
	\Blue{der Rollstuhl} & wheelchair \\
	\Red{die Praxis} & [medical] practice \\
	\Red{die Gesundheit} & health \\
	\Red{die Untersuchung} & examination/study/investigation \\
	das Formular & form \\
	\Red{die Krankheit} & disease/illness/sickness \\
	\Red{die Medizin} & medicine \\
	das Krankenhaus & hospital \\
	\Blue{der Krankenwagen} & ambulance \\
	\Red{die Krankenversicherung} & health insurance (``sickness'' + ``insurance'') \\
	\Blue{der Unfall (die Unf{\"a}lle)} & accident(s) \\
	\Blue{der Notfall} & emergency \\
	das Opfer & victim \\
	\Red{die Klinik} & clinic \\
	\Red{die Therapie} & therapy \\
	\Blue{der Alkohol} & alcohol \\
	\Red{die Di{\"a}t} & diet \\
	\Red{die Ern{\"a}hrung} & nutrition/nourishment \\
	\Blue{der Zahnarzt}, \Red{die Zahnarztin} & dentist \\
\end{tabular}\end{center}


\pagebreak
\subsection{Present 2}

\begin{center}  k{\"o}nnen (to be able to) \end{center}

\begin{center}\begin{tabular}{l|l|l}
  \textbf{English person} & \textbf{ending} & \textbf{German example} \\
	\hline
	I & -e & ich kann \\
	\hline
	you (singular informal) & -st & du kann\Red{st} \\
	\hline
	he/she/it & -t & er/sie/es kann \\
	\hline
	we & -en & wir k{\"o}nn\Red{en} \\
	\hline
	you (plural informal) & -t & ihr k{\"o}nn\Red{t} \\
	\hline
	you (formal) & -en & Sie k{\"o}nn\Red{en} \\
	\hline
	they & -en & sie k{\"o}nn\Red{en} \\
\end{tabular}\end{center}

\begin{center}\begin{tabular}{r|l}
  \textbf{Deutsch} & \textbf{English} \\
	\hline
	m{\"o}chten & to like \\
	arbeiten & to work \\
	sammelen & to collect \\
	singen & to sing \\
	schwimmen & to swim \\
	hoffen & to hope/expect \\
	kennen & to know [i.e. to know a person] \\
	wissen (ich wei{\ss}) & to know [i.e. to know a fact] (I know) \\
	suchen & to search [for] \\
	gefallen & to like/please (umlaut on second-person or third-person singular forms) \\
	gehen & to go \\
	passieren & to happen \\
	finden & to find \\
	{\"a}nderen & to change \\
	erlauben & to allow/let/permit \\
	treffen (du trifft) & to meet \\
	glauben & to believe \\
	benutzen & to use \\
	warten & to wait \\
	Warten Sie! & Wait! \\
\end{tabular}\end{center}
\pagebreak
\begin{center}\begin{tabular}{r|l}
  \textbf{Deutsch} & \textbf{English} \\
	\hline
	liegt & there is \\
	stehen & to stand (to point out location) \\
	scheinen & to shine \\
	ersetzen & to replace \\
	nehmen & to take \\
	schlie{\ss}en & to close \\
	stelle & to put \\
	bestellen & to order [food] \\
	meinen & to mean [something] \\
	funktionieren & to work/function \\
	erkennen & to recognize \\
	pr{\"u}fen & to test/examine \\
	halten & to hold (umlaut on second-person or third-person singular forms) \\
	{\"u}bernimmen & to take over \\
	abonnieren & to subscribe \\
	spazieren & to walk \\
	bleiben & to remain/stay \\
	schauen & to look \\
	steigen & to get on/off/in [something] \\
	%erweitern & 
	fehlen & to be missing \\
	stiehlen & to steal \\
	akzeptieren & to accept \\
	verpassen & to miss [something] \\
\end{tabular}\end{center}


\pagebreak
\subsection{Dates 2}

\Blue{Der M{\"a}rz} ist ein \Blue{Monat}.  \Blue{Die Monate} sind
\begin{center}
  \Blue{Januar, Februar, M{\"a}rz, April, Mai, Juni,} \\ \Blue{Juli, August, September, Oktober, November, und Dezember}
\end{center}

\Blue{Der Fr{\"u}hling} ist ein \Blue{Jahreszeit}.  \Blue{Die Jahreszeiten} sind
\begin{center}
  \Blue{Fr{\"u}hling, Sommer, Herbst, und Winter}
\end{center}

\begin{center}\begin{tabular}{r|l}
  \textbf{Deutsch} & \textbf{English} \\
	\hline
	monatlich & monthly \\
	\Blue{der Spargel} & asparagus \\
	das Jahr (die Jahre) & year(s) \\
	j{\"a}hrlich & yearly \\
	das quartal & quarter \\
	hei{\ss} & hot/burning/passionate \\
	\Red{die Daten} & data \\
	das Datum & date \\
	\Blue{der Kalender} & calendar \\
	k{\"u}hl & cool/chilly \\
	\Red{die Saison} & season (not just for ``Spring, Summer, Fall, or Winter'') \\
	letzte & last (i.e. describing the last item of a list) \\
	das Weihnachten & Christmas \\
	Schluss & over (as in no longer existing, e.g. a breakup) \\
	das Alter & age \\
	\Blue{der Geburtstag} & birthday (``birth'' + ``day'') \\
	\Red{die Phase} & phase \\
	das Jahrhundert & century \\
	enden & to end/finish \\
	vorbei & over/passed/finished/ended \\
\end{tabular}\end{center}


\pagebreak
\subsection{People 2}

\begin{center}\begin{tabular}{r|l}
  \textbf{Deutsch} & \textbf{English} \\
	\hline
	\Red{die Gemeinde} & community/parish/congregation \\
	\Blue{der Verein (die Vereine)} & club(s) \\
	\Red{die {\"O}ffentlichkeit} & public sphere (usually for politics) \\
	\Red{die Verbindung} & connection/relationship \\
	das Verh{\"a}ltnis (die Verh{\"a}ltnisse) & affair(s)/relationship(s) \\
	\Blue{der Nutzer} & user \\
	\Red{die Bev{\"o}lkerung} & population \\
	\Red{die Jungend} & youth \\
	\Red{die Mitgliedschaft} & membership \\
	\Blue{der Einwohner} & inhabitants/residents (written as singular in German) \\
	das Paar & pair \\
\end{tabular}\end{center}


\pagebreak
\subsection{Future}

The future tense consists of a conjugated form of werden in the present tense and an infinitive (the base form of the verb).  Depending on the context, ``ich werde spielen'' translates to ``I will play'' or "I am going to play". In German, there is no distinction between `will' and `going to'.  German normally uses the present tense to indicate the future. For example, ``ich gehe morgen ins Kino'' translates to ``I will go to the movies tomorrow''.

\begin{center}  werden (to do [in the future]) \end{center}

\begin{center}\begin{tabular}{l|l|l}
  \textbf{English person} & \textbf{ending} & \textbf{German example} \\
	\hline
	I & -e & ich werde \\
	\hline
	you (singular informal) & -st & du wirst \\
	\hline
	he/she/it & -t & er/sie/es wird \\
	\hline
	we & -en & wir werden \\
	\hline
	you (plural informal) & -t & ihr werdet \\
	\hline
	you (formal) & -en & Sie werden \\
	\hline
	they & -en & sie werden \\
\end{tabular}\end{center}

\begin{center}\begin{tabular}{r|l}
  \textbf{Deutsch} & \textbf{English} \\
	\hline
	reden & to talk \\
	warten & to wait \\
	werden & [can also mean] going to \\
	bleiben & to stay \\
	vergessen & to forget \\
	lieben & to love \\
	%kosten & to cost \\
	wissen & to know \\
	%folgen & to follow \\
	testen & to test \\
	rufen & to call \\
	l{\"o}sen & solve/resolve/loosen \\
	%sprechen & to speak \\
	handeln & haggle/trade/take action \\
	beraten & advise/counsel \\
	sch{\"u}tzen & to protect \\
	bieten & to offer \\
	merken & to remember \\
	erkl{\"a}ren & to explain \\
\end{tabular}\end{center}


\pagebreak
\subsection{Feelings}

\begin{center}\begin{tabular}{r|l||r|l}
  \textbf{Deutsch} & \textbf{English} & \textbf{Deutsch} & \textbf{English} \\
	\hline
	\Red{die Liebe} & love & interessant & interesting \\
	\Blue{der Traum} & dream & langweilig & boring \\
	tr{\"a}umen & to dream & \Blue{der {\"A}rger} & anger \\
	Ich habe \Red{keine Lust}. & I don't feel like it. & Ich mag \rule{1cm}{0.4pt} lieber & I prefer \rule{1cm}{0.4pt} \\
	\Red{die Not} & distress/need/trouble & Ehrlich? & Really? \\
	\Red{die Freude} ist gro{\ss}. & There is a lot of joy. & witzig & funny \\
	\Blue{der Wunsch} & wish & dumm & dumb/stupid/dense \\
	lieben & to love & Ganz sch{\"o}n schlau! & Pretty clever! \\
	Bitte zeige Verst{\"a}ndnis & Please show understanding & sein & to be \\
	\Blue{der Spa{\ss}} & fun & stolz & proud \\
	Es ist mein \Blue{Ernst} & I am serious & \Blue{der Liebling} & darling \\
	im \Blue{Ernst} & seriously & total & really (as an adverb) \\
	\Blue{der Humor} & humor (sense of humor) & b{\"o}se & mad/evil \\
	Hast du Angst? & Are you afraid? & nett & nice \\
	\Blue{der Eindruck} & impression & tapfer & brave \\
	unheimlich & scary/eerie/uncanny & \Red{die Schuld} & fault/blame/mortgage \\
	\Red{die Gedanken} & thoughts & lachen & to laugh \\
	\Red{die Ruhe} & rest & schlimm & severe/bad/serious \\
	\Blue{der Witz} & joke & gar nicht & not at all \\
	hassen & to hate & Zum Gl{\"u}ck & fortunately \\
\end{tabular}\end{center}


\pagebreak
\subsection{Time}

\begin{center}\begin{tabular}{r|l||r|l}
  \textbf{Deutsch} & \textbf{English} & \textbf{Deutsch} & \textbf{English} \\
	\hline
	\hline
	\Blue{der Termin} &  appointment & \Red{die Uhrzeit} & time (``clock'' + ``show'') \\
	steht fest & is set & \Red{die Stunde} & hour \\
	Moment & ``Hold on.'' or ``Just a moment.'' & \Red{die Minute} & minute \\
	dann & then & \Red{die Sekunde} & second \\
	jetzt & now & sp{\"a}ter & later \\
	heute & today & \Blue{der Zeitpunkt} & point in time \\
	morgen & tomorrow & halb & half \\
	\Blue{der Morgen} & morning & viertel & quarter \\
	\Blue{der Mittag} & noon (``middle'' + ``day'') & nachts & nights \\
	heute Abend & this evening & sofort & immediately/right away \\
	\Red{die Nacht (die N{\"a}chte)} & night(s) & etwa & about \\
	fr{\"u}h & early & fast & almost \\
	sp{\"a}t & late & gerade & currently \\
	\Blue{der Zeitraum} & [time] period & \Red{die Dauer} & duration \\
	einen Augenblick Zeit & a moment & lange & a long time \\
	\Red{die Uhr} & clock & damals & [back] then \\
	\Red{die Mitternacht} & midnight \\
\end{tabular}\end{center}

\begin{itemize}
  \item  Es ist halb acht. \\
  It is half past seven.
  \item  Ich bin nicht von gestern \\
  I was not born yesterday.
  \item  Das dauert. \\
  That takes a while.
\end{itemize}


\pagebreak
\subsection{Frequency}

\begin{center}\begin{tabular}{r|l}
  \textbf{Deutsch} & \textbf{English}  \\
	\hline
	mehr & more \\
	oft & often \\
	als & as \\
	manchmal & sometimes \\
	weniger/wenige & few (by gender) \\
	gro{\ss}eren & bigger/larger (by gender) \\
	zahlreiche & numerous/many \\
	h{\"a}ufig & frequent/often/common \\
	das letzte Mal & the last time \\
	selten & rare/infrequent/seldom \\
	ein bisschen & a little [bit of] \\
	ob & whether/if/though (conjunction) \\
	meist & usually \\
	haben keinerlei & does not have (has absolutely no) \\
	je \rule{1cm}{0.4pt}, je \rule{1cm}{0.4pt} & the \rule{1cm}{0.4pt}, the \rule{1cm}{0.4pt} \\
	je \rule{1cm}{0.4pt}, desto \rule{1cm}{0.4pt} & the \rule{1cm}{0.4pt}, the \rule{1cm}{0.4pt} \\
	meisten & most of \\
\end{tabular}\end{center}


\pagebreak
\subsection{Verbs:  Modal}

\begin{center}  k{\"o}nnen (``could'' is a conditional verb) \end{center}

\begin{center}\begin{tabular}{l|l|l}
  \textbf{English person} & \textbf{ending} & \textbf{German example} \\
	\hline
	I & -te & ich k{\"o}nn\Red{te} \\
	\hline
	you (singular informal) & -test & du k{\"o}nn\Red{test} \\
	\hline
	he/she/it & -te & er/sie/es k{\"o}nn\Red{te} \\
	\hline
	we & -ten & wir k{\"o}nn\Red{ten} \\
	\hline
	you (plural informal) & -tet & k{\"o}nn\Red{tet} \\
	\hline
	you (formal) & -ten & Sie k{\"o}nn\Red{ten} \\
	\hline
	they & -ten & sie k{\"o}nn\Red{ten} \\
\end{tabular}\end{center}

\begin{center}  sollen (``should'' is a conditional verb) \end{center}

\begin{center}\begin{tabular}{l|l|l}
  \textbf{English person} & \textbf{ending} & \textbf{German example} \\
	\hline
	I & -te & ich soll\Red{te} \\
	\hline
	you (singular informal) & -test & du soll\Red{test} \\
	\hline
	he/she/it & -te & er/sie/es soll\Red{te} \\
	\hline
	we & -ten & wir soll\Red{ten} \\
	\hline
	you (plural informal) & -tet & soll\Red{tet} \\
	\hline
	you (formal) & -ten & Sie soll\Red{ten} \\
	\hline
	they & -ten & sie soll\Red{ten} \\
\end{tabular}\end{center}

\begin{center}  m{\"o}gen (``Do \rule{1cm}{0.4pt} want \rule{1cm}{0.4pt}?'' is a conditional question) \end{center}

\begin{center}\begin{tabular}{l|l|l}
  \textbf{English person} & \textbf{ending} & \textbf{German example} \\
	\hline
	I & -te & ich m{\"o}ch\Red{te} \\
	\hline
	you (singular informal) & -test & du m{\"o}ch\Red{test} \\
	\hline
	he/she/it & -te & er/sie/es m{\"o}ch\Red{te} \\
	\hline
	we & -ten & wir m{\"o}ch\Red{ten} \\
	\hline
	you (plural informal) & -tet & m{\"o}ch\Red{tet} \\
	\hline
	you (formal) & -ten & Sie m{\"o}ch\Red{ten} \\
	\hline
	they & -ten & sie m{\"o}ch\Red{ten} \\
\end{tabular}\end{center}

\begin{center}\begin{tabular}{r|l}
  \textbf{Deutsch} & \textbf{English}  \\
	\hline
	%essen & to eat \\
	wollen & to want \\
	kannen & can (to be able to) \\
	%gehen & to go \\
	mussen & must (makes a verb imperative: ``Du musst essen'') \\
	%schlafen & to sleep \\
	%machen & to make \\
	%schwimmen & to swim \\
	finden & to find (e.g. ``Ich kann es nicht finden'') \\
	%h{\"o}ren & to hear \\
	%lesen & to read \\
	%magen & to like \\
	%lernen & to learn \\
	d{\"u}rfen & may (is a conditional verb) \\
	%sehen & to see \\
	%zeigen & to show \\
	%sprechen & to speak \\
	%bezahlen & to pay \\
	best{\"a}tigen & to confirm \\
	%kochen & to cook \\
	benutzen & to use (e.g. ``Kann ich bitte ihre Toilette benutzen?'') \\
	%helfen & to help (e.g. ``Wie kann ich helfen?'') \\
	reden & to speak \\
	wiederholen & to repeat (e.g. ``Kannst du das bitte weiderholen?'') \\
	mieten & to rent \\
	wecken & to wake up \\
	entscheiden & to decide \\
	Man & people/man/one (i.e. a generalized subject of a sentence) \\
	aussagen & to testify \\
	darf & allowed \\
	registrieren & to register \\
	teilnehmen & to participate (competitively or professionally) \\
	mitmachen & to participate (for leisure) \\
	tun & to do \\
	sagen & to say \\
	spazieren & to walk \\
	%fragen & to ask \\
	betrachten & to look \\
	vorstellen & to introduce (``front'' + ``put'') \\
	ersetzen & to replace \\
	(Gl{\"u}ck) w{\"u}nschen & to wish (luck) \\
	%{\"o}ffnen & to open \\
\end{tabular}\end{center}