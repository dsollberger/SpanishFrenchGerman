\subsection{Basics 1}

\subsubsection{Capitalizing nouns}

In German, all nouns are capitalized. For example, ``my name'' is ``mein Name'', and ``the apple'' is ``der Apfel''.  This helps you identify which are the nouns in a sentence.

\subsubsection{Conjugations of the verb sein (to be)}

A few verbs like ``sein'' (to be) are completely irregular, and their conjugations simply need to be memorized:

\begin{center}\begin{tabular}{l|l}
  \textbf{German} & \textbf{English} \\
	\hline
	ich bin & I am \\
	\hline
	du bist & you (singular informal) are \\
	\hline
	er/sie/es ist & he/she/it is \\
	\hline
	wir sind & we are \\
	\hline
	ihr seid & you (plural informal) are \\
	\hline
	sie sind & they are \\
	\hline
	Sie sind & you (formal) are
\end{tabular}\end{center}

\subsubsection{Umlauts}

Umlauts are letters (more specifically vowels) that have two dots above them and appear in some German words like ``M{\"a}dchen''. Literally, ``Umlaut'' means ``around the sound'', because its function is to change how the vowel sounds.

An umlaut can sometimes indicate the plural of a word. For example, the plural of ``Mutter'' (mother) is ``M{\"u}tter''. It might even change the meaning of a word entirely. That's why it's very important not to ignore those little dots.

\subsubsection{No continuous aspect}

In German, there's no continuous aspect, i.e. there are no separate forms for ``I drink'' and ``I am drinking''. There's only one form: Ich trinke.  There's no such thing as Ich bin trinke or Ich bin trinken!  When translating into English, how can I tell whether to use the simple (I drink) or the continuous form (I am drinking)?  Unless the context suggests otherwise, either form should be accepted.

\pagebreak
\subsubsection{Conjugating regular verbs}

Verb conjugation in German is more challenging than in English. To conjugate a regular verb in the present tense, identify the invariant stem of the verb and add the ending corresponding to any of the grammatical persons, which you can simply memorize:

\begin{center}  trinken (to drink) \end{center}

\begin{center}\begin{tabular}{l|l|l}
  \textbf{English person} & \textbf{ending} & \textbf{German example} \\
	\hline
	I & -e & ich trink\Red{e} \\
	\hline
	you (singular informal) & -st & du trink\Red{st} \\
	\hline
	he/she/it & -t & er/sie/es trink\Red{t} \\
	\hline
	we & -en & wir trink\Red{en} \\
	\hline
	you (plural informal) & -t & ihr trink\Red{t} \\
	\hline
	you (formal) & -en & Sie trink\Red{en} \\
	\hline
	they & -en & sie trink\Red{en} \\
\end{tabular}\end{center}

\subsubsection{Vocabulary}

\begin{center}\begin{tabular}{l|l}
  \textbf{German} & \textbf{English} \\
	\hline
  \Blue{der Mann} & man \\
	\Red{die Frau} & woman \\
	\Blue{der Junge} & boy \\
	das M{\"a}chen & girl (neuter gender unfortunately due to the -chen suffix) \\
	\Red{die M{\"a}chen} & girls \\
	\Blue{der Kind} & child \\
	\Blue{der Brot} & bread \\
	\Blue{der Wasser} & water \\
\end{tabular}\end{center}


\pagebreak
\subsection{The}

\subsubsection{Three grammatical genders, three types of nouns}

Nouns in German are either feminine, masculine or neuter. For example, ``Frau'' (woman) is feminine, ``Mann'' (man) is masculine, and ``Kind'' (child) is neuter. The grammatical gender may not match the biological gender: ``M{\"a}chen'' (girl) is a neuter noun.

It is very important to learn every noun along with its gender because parts of German sentences change depending on the gender of their nouns.

Generally speaking, the definite article ``die'' (the) and the indefinite article ``eine'' (a/an) are used for feminine nouns, ``der'' and ``ein'' for masculine nouns, and ``das'' and ``ein'' for neuter nouns. For example, it is ``die Frau,'' ``der Mann,'' and ``das Kind.'' However, later you will see that this changes depending on something called the ``case of the noun.''


\pagebreak
\subsection{Basics 2}

\subsubsection{German plurals - the Nominative Case}

In English, making plurals out of singular nouns is typically as straightforward as adding an "s" or an "es" at the end of the word. In German, the transformation is more complex, and also the articles for each gender change. The following five suggestions can help:
\begin{enumerate}
  \item  -e ending: most German one-syllable nouns will need -e in their plural form. For example, in the nominative case, ``das Brot'' (the bread) becomes ``die Brote,'' and ``das Spiel'' (the game) becomes ``die Spiele.''
	\item  -er ending: most masculine or neuter nouns will need the -er ending, and there may be umlaut changes. For example, in the nominative case ``das Kind'' (the child) becomes ``die Kinder,'' and ``der Mann'' (the man) becomes ``die M{\"a}nner.''
	\item  -n/-en ending: most feminine nouns will take either -n or -en in all four grammatical cases, with no umlaut changes. For example, ``die Frau'' (the woman) becomes ``die Frauen'' and ``die Kartoffel'' becomes ``die Kartoffeln.''
	\item  -s ending: most foreign-origin nouns will take the -s ending for the plural, usually with no umlaut changes. For example: ``der Chef'' (the boss) becomes ``die Chefs.''
	\item  There is no change for most neuter or masculine nouns that contain any of these in the singular: -chen, -lein, -el, or -er. There may be umlaut changes. For example: ``das M{\"a}dchen'' (the girl) becomes ``die M{\"a}dchen,'' and ``die Mutter'' (the mother) becomes ``die M{\"u}tter.''
\end{enumerate}

\subsubsection{German feminine plurals - nouns ending in -in}

Feminine nouns that end in ``-in'' will need ``-nen'' in the plural. For example, ``die K{\"o}chin'' (the female cook) becomes ``die K{\"o}chinnen'' in its plural form.

\subsubsection{ihr vs er}

If you're new to German, ``ihr'' and ``er'' may sound exactly same, but there is actually a difference. ``Ihr'' sounds similar to the English word `ear', and ``er'' sounds similar to the English word `air' (imagine a British/RP accent).

Don't worry if you can't pick up on the difference at first. You may need some more listening practice before you can tell them apart. Also, try using headphones instead of speakers.  Even if this doesn't seem to help, knowing your conjugation tables will greatly reduce the amount of ambiguity.

\subsubsection{Simple German Present Tense}

In English, the present tense can be simple or progressive (as in ``I eat'' or ``I am eating''). Both forms translate to just one German present tense form, because there is no continuous tense in standard German. So, ``she learns'' and ``she is learning'' are both ``sie lernt''


\pagebreak
\subsection{Common Phrases}

\subsubsection{Wie geht's?}

There are many ways to ask someone how he or she is doing. Take ``How are you?,'' ``How do you do?'' and ``How is it going?'' as examples. In German, the common phrase or idiom uses the verb ``gehen'' (go): ``Wie geht es dir?'' (How are you?).

\subsubsection{Willkommen can be a false friend}

In German, ``Willkommen'' means welcome as in ``Welcome to our home'', but it does not mean welcome as in ``Thank you - You're welcome''. The German for the latter is ``Gern geschehen'' or ``Keine Ursache''.

\subsubsection{Vocabulary}

\begin{center}\begin{tabular}{l|l}
  \textbf{German} & \textbf{English} \\
	\hline
	hallo & Hello \\
	tsch{\"u}ss & bye \\
	danke & thanks \\
	bitte & please \\
	ja & yes \\
	nein & no \\
	Guten Morgen & good morning \\
	Guten Tag & good afternoon \\
	Guten Abend & good evening \\
	gern geschehen & you're welcome \\
	Wie geht's? & How are you? \\
	Mir geht's gut & I am good \\
	auf Wiedersehen & goodbye (loosely: ``upon seeing you again'') \\
	bis bald & see you soon \\
	bis morgen & see you tomorrow \\
	bis sp{\"a}ter & see you later \\
	leider & unfortunately \\
	gute Nacht & good night \\
	in ordnung & alright \\
	genau & exactly (i.e. right, absolutely) \\
	Entschuldigung & I am sorry (i.e. excuse me) \\
	Es tut mir leid & I am sorry (i.e. I am sorrowful)
\end{tabular}\end{center}


\pagebreak
\subsection{Accusative Case}

\subsubsection{German Cases}

In English, the words ``he'' and ``I'' can be used as subjects (the ones doing the action in a sentence), and they change to ``him'' and ``me'' when they are objects (the ones the action is applied to). For example, we say ``He likes me'' and ``I like him.'' This is exactly the notion of a `grammatical case:' the same word changes its form depending on its relationship to the verb. In English, only pronouns have cases, but in German most words other than verbs have cases: nouns, pronouns, determiners, adjectives, etc.

Understanding the four German cases is one of the biggest hurdles in learning the language. The good news is that most words change very predictably so you only have to memorize a small set of rules. We'll see more about cases later, but for now you just need to understand the difference between the two simplest cases: nominative and accusative.

The subject of a sentence (the one doing the action) is in the nominative case. So when we say ``Die Frau spielt'' (the woman plays), ``Frau'' is in the nominative.

The accusative object is the thing or person that is directly receiving the action. For example, in ``Der Lehrer sieht den Ball'' (the teacher sees the ball), ``Lehrer'' is the nominative subject and ``Ball'' is the accusative object. Notice that the articles for accusative objects are not the same as the articles in the nominative case: ``the'' is ``der'' in the nominative case and ``den'' in the accusative. The following table shows how the articles change based on these two cases:

\begin{center}\begin{tabular}{r|c|c|c|c}
  \multicolumn{5}{c}{definite article} \\
  Case & Masculine & Feminine  & Neuter & Plural \\
	\hline
	Nominative & \textbf{der} & die & das & die \\
	Accusative & \textbf{den} & die & das & die \\
\end{tabular}\end{center}

\begin{center}\begin{tabular}{r|c|c|c}
  \multicolumn{4}{c}{indefinite article} \\
  Case & Masculine & Feminine  & Neuter \\
	\hline
	Nominative & \textbf{ein} & eine & ein \\
	Accusative & \textbf{einen} & eine & ein \\
\end{tabular}\end{center}

The fact that most words in German are affected by the case explains why the sentence order is more flexible than in English. For example, you can say ``Das M{\"a}dchen hat den Apfel'' (the girl has the apple) or ``Den Apfel hat das M{\"a}dchen.'' In both cases, ``den Apfel'' (the apple) is the accusative object, and ``das M{\"a}dchen'' is the nominative subject.

\pagebreak
\subsubsection{Conjugations of the verb essen (to eat)}

The verb ``essen'' (to eat) is slightly irregular in that the stem vowel changes from e to i in the second (du isst) and third person singular (er/sie/es isst) forms.

\begin{center}  essen (to eat) \end{center}

\begin{center}\begin{tabular}{l|l|l}
  \textbf{English person} & \textbf{ending} & \textbf{German example} \\
	\hline
	I & -e & ich esse \\
	\hline
	you (singular informal) & -st & du \Red{isst} \\
	\hline
	he/she/it & -t & er/sie/es isst \\
	\hline
	we & -en & wir essen \\
	\hline
	you (plural informal) & -t & ihr esst \\
	\hline
	you (formal) & -en & Sie essen \\
	\hline
	they & -en & sie essen \\
\end{tabular}\end{center}

\textbf{How can you hear the difference between isst and ist?}

You can't. ``isst'' and ``ist'' sound exactly the same. In colloquial (rapid) speech, some speakers drop the `t' in ``ist''.  So ``Es ist ein Apfel'' and ``Es isst ein Apfel'' sound the same?  Yes, but you can tell it's ``Es ist ein Apfel'' because ``Es isst ein Apfel'' is ungrammatical. The accusative of ``ein Apfel'' is ``einen Apfel''. Hence, ``It is eating an apple'' translates as ``Es isst einen Apfel.''

\subsubsection{The verb haben (to have)}

In English, you can say ``I'm having bread'' when you really mean that you're eating or about to eat bread. This does not work in German. The verb haben refers to possession only. Hence, the sentence ``Ich habe Brot'' only translates to ``I have bread'', not ``I'm having bread''. Of course, the same applies to drinks. ``Ich habe Wasser'' only translates to ``I have water'', not ``I'm having water''.

\begin{center}  haben (to have) \end{center}

\begin{center}\begin{tabular}{l|l|l}
  \textbf{English person} & \textbf{ending} & \textbf{German example} \\
	\hline
	I & -e & ich habe \\
	\hline
	you (singular informal) & -st & du hast \\
	\hline
	he/she/it & -t & er/sie/es hat \\
	\hline
	we & -en & wir haben \\
	\hline
	you (plural informal) & -t & ihr habt \\
	\hline
	you (formal) & -en & Sie haben \\
	\hline
	they & -en & sie haben \\
\end{tabular}\end{center}


\pagebreak
\subsection{Introductions}

%\subsubsection{Vocabulary}

\begin{center}\begin{tabular}{l|l}
  \textbf{German} & \textbf{English} \\
	\hline
	Hallo, ich hei{\ss}e \rule{1cm}{0.4pt} & Hello, my name is \rule{1cm}{0.4pt} (or ``Hello, I am called \rule{1cm}{0.4pt}'') \\
	Hallo, ich bin \rule{1cm}{0.4pt} & Hello, I am \rule{1cm}{0.4pt} \\
	Sie hei{\ss}e \rule{1cm}{0.4pt} & She is called \rule{1cm}{0.4pt} \\
	aus & from \\
	kommen & to come \\
	Du kommst aus \rule{1cm}{0.4pt} & You are from \rule{1cm}{0.4pt} \\
	Deutsch & German \\
	Deutschland & Germany (more literally:  ``German country'') \\
	sprechen & to speak \\
	verstehen & to understand \\
	Englisch & English \\
\end{tabular}\end{center}


\pagebreak
\subsection{Food}

\subsubsection{Mittagessen - lunch or dinner?}

We're aware that dinner is sometimes used synonymously with lunch, but for the purpose of this course, we're defining Fr{\"u}hst{\"u}ck as breakfast, Mittagessen as lunch, and dinner / supper as Abendessen / Abendbrot.

\subsubsection{Compound words}

A compound word is a word that consists of two or more words. These are written as one word (no spaces).

The gender of a compound noun is always determined by its last element. This shouldn't be too difficult to remember because the last element is always the most important one. All the previous elements merely describe the last element.
\begin{itemize}
  \item  die Autobahn (das Auto + die Bahn)
	\item  der Orangensaft (die Orange + der Saft)
	\item  das Hundefutter (der Hund + das Futter)
\end{itemize}

Sometimes, there's a connecting sound (Fugenlaut) between two elements. For instance, die Orange + der Saft becomes der Orangensaft, der Hund + das Futter becomes das Hundefutter, die Liebe + das Lied becomes das Liebeslied, and der Tag + das Gericht becomes das Tagesgericht.

\subsubsection{Cute like sugar!}

The word s{\"u}{\ss} means sweet when referring to food, and cute when referring to living beings.
\begin{itemize}
  \item  Der Zucker ist s{\"u}{\ss}. (The sugar is sweet.)
	\item  Die Katze ist s{\"u}{\ss}. (The cat is cute.)
\end{itemize}
    

\pagebreak
\subsubsection{Vocabulary}

\begin{center}\begin{tabular}{l|l||l|l}
  \textbf{Deutsch} & \textbf{English} & \textbf{Deutsch} & \textbf{English} \\
	\hline
	\Red{die Suppe} & soup & \Blue{der Fisch} & fish \\
	das Essen & food & \Red{die Pizza} & pizza \\
	Hunger & hunger (use with the `haben` verb) & das Eis & ice cream (or just ice) \\
	\Blue{der Wein} & wine & \Blue{der Tee} & tea \\
	das Bier & beer & \Blue{der Kaffee} & coffee \\
	schmecken & to taste & Durst & thirst (use with the `haben` verb) \\
	\Blue{der Saft} & juice  & \Red{die Orange} & orange (fruit) \\
	\Blue{der Apfil} & apple & das Obst & fruit \\
	\Red{die Banane} & banana & \Blue{der Orangensaft} & orange juice \\
	\Red{die Kartoffel} & potato & \Blue{der K{\"a}se} & cheese \\
	das Ei & egg & das Fleisch & meat \\
	das Gem{\"u}se & vegetables & gut & good \\
	\Blue{der Zucker} & sugar & \Red{die Schokolade} & chocolate \\
	\Red{die Erdbeere} & strawberry & frisch & fresh \\
	lecker & delicious & s{\"u}{\ss} & sweet \\
	das Salz & salt & \Blue{der Reis} & rice \\
	das {\"O}l & salt & und & and \\
	\Red{die Nudeln} & pasta (or noodles) \\
\end{tabular}\end{center}


\pagebreak
\subsection{Animals}

\subsubsection{Essen and Fressen}

Unlike English, German has two similar but different verbs for to eat: essen and fressen. The latter is the standard way of expressing that an animal is eating something. Be careful not to use fressen to refer to humans – this would be a serious insult. Assuming you care about politeness, we will not accept your solutions if you use fressen with human subjects.  The most common way to express that a human being is eating something is the verb essen. It is not wrong to use it for animals as well, so we will accept both solutions. But we strongly recommend you accustom yourself to the distinction between essen and fressen.  Fortunately, both verbs are conjugated very similarly:

\begin{center}\begin{tabular}{r|l}
  \textbf{essen} & \textbf{fressen} (for animals) \\
	\hline
	ich esse  & ich fresse \\
	du isst & du frisst \\
	er/sie/es isst & er/sie/es frisst \\
	wir essen & wir fressen \\
	ihr eest & ihr fresst \\
	sie/Sie essen & sie/Sie fressen \\
\end{tabular}\end{center}

\subsubsection{Vocabulary}

\begin{center}\begin{tabular}{l|l}
  \textbf{Deutsch} & \textbf{English} \\
	\hline
	\Blue{der Hund} & dog \\
  \Red{die Katze} & cat \\
	\Red{die Maus} & mouse \\
	\Blue{der B{\"a}r} & bear \\
	das Tier & animal \\
	das Pferd & horse \\
	\Blue{der Vogel} & bird \\
	\Red{die Ente} & duck \\
	\Red{die Kuh} & cow \\
	das Haustier & pet (literally `house animal') \\
	\Red{die Spinne} & spider \\
	\Red{die Biene} & bee \\
	\Blue{der K{\"a}fer} & beetle \\
	\Red{die Fliege} & fly \\
	das Insekt & insect \\
\end{tabular}\end{center}


%\pagebreak
