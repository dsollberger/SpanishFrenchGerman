\subsection{Prepositions 1}

French prepositions can be difficult because their meanings and uses don't always line up to what you would expect in English.

\subsubsection{De and {\`A}}

The most common French prepositions are \guillemotleft de \guillemotright (``of''/``from'') and \guillemotleft {\`a} \guillemotright (``to''/``at''). These prepositions can be used in many ways. For instance, they may indicate movement or location.

\begin{itemize}
  \item  Nous allons {\`a} Paris. \\ We are going to Paris.
  \item  Il vient de Bordeaux. \\ He is coming from Bordeaux.
  \item  Je suis au restaurant. \\ I am at the restaurant.
\end{itemize}

Notice \guillemotleft au \guillemotright above. De and {\`a} must contract with definite articles whenever they are adjacent.

\begin{center}\begin{tabular}{|c|c|c|}
\hline
\textbf{Definite Article} & \textbf{De} & \textbf{{\`A}} \\ \hline
le                        & du          & au         \\ \hline
la                        & de la       & {\`a} la       \\ \hline
les                       & des         & aux        \\ \hline
\end{tabular}\end{center}

If the contraction is followed by a vowel sound, du and de la both become \guillemotleft de l' \guillemotright and au and {\`a} la both become \guillemotleft {\`a} l' \guillemotright . This change occurs for euphony only; the nouns do not change genders because of it.

\begin{itemize}
  \item  Tu parles {\`a} l'enfant. (Not au) \\ You are speaking to the child.
  \item  La Maison de l'Ours \\ The House of the Bear
  \item  Les copies des livres. \\ The copies of the books.
  \item  Le repas du chien. \\ The dog's meal. (The meal of the dog.)
\end{itemize}

De may be found in numerous fixed expressions, especially in adverbs of quantity like \guillemotleft beaucoup de \guillemotright (``a lot of'').

\begin{itemize}
  \item  Nous avons beaucoup de pommes. \\ We have a lot of apples.
  \item  R{\'e}my a beaucoup d'amis. \\ Remy has a lot of friends.
\end{itemize}

Adding de or {\`a} to the end of certain verbs can change their meanings.

\begin{itemize}
  \item  Penser (``to think''): Je pense que c'est un homme. \\ I think that he is a man.
  \item  Penser {\`a} (``to think about''): Elle pense {\`a} son chien. \\ She's thinking about her dog.
  \item  Penser de (``to have an opinion about''): Que pensez-vous de ce repas ? \\ What do you think of this meal?
\end{itemize}

\subsubsection{Using articles after de}

Most articles can be used immediately after expressions and verbs ending in de, but they must follow contraction and elision rules.

\begin{itemize}
  \item  Elle parle beaucoup des (de + les) p{\^a}tes. \\ She speaks a lot about the pasta.
  \item  Que pensez-vous de la voiture ? \\ What do you think of the car?
  \item  Il a besoin d'un chien. \\ He needs a dog.
\end{itemize}

However, no article that already contains de may follow an expression, negative term, or verb ending in de. This includes the partitives du and de la and the indefinite des. In this situation, the article is removed so that only the naked de remains.

\begin{itemize}
  \item  Elle mange beaucoup de frites. (Not de des) \\ She eats a lot of fries.
  \item  Je n'ai pas de pain. (Not de du) \\ I do not have (any) bread.
  \item  Il a besoin d'argent (Not de de l') \\ He needs (some) money.
\end{itemize}

\subsubsection{Des before adjectives}

When des appears immediately before an adjective, it changes to de. This only occurs with BANGS adjectives, which come before the noun.

\begin{itemize}
  \item  Vous {\^e}tes de jeunes gar{\c c}ons. \\ You are young boys.
  \item  Elle a de petits chiens. \\ She has small dogs.
\end{itemize}

\begin{itemize}
  \item  Merci pour le repas. \\ Thank you for the meal.
  \item  Je bois du jus de pomme. \\ I am drinking apple juice.
\end{itemize}


\pagebreak
\subsection{Numbers 1}

Between z{\'e}ro and 20, most French numbers are constructed similarly to English numbers. The main difference is that French starts using hyphenated compound numbers, at dix-sept (17), while English continues with single-word numbers until 21.


\begin{center}\begin{tabular}{|c|c|c|c|}
\hline
\textbf{Number} & \textbf{Nombre} & \textbf{Number} & \textbf{Nombre} \\ \hline
1               & un              & 11              & onze            \\ \hline
2               & deux            & 12              & douze           \\ \hline
3               & trois           & 13              & treize          \\ \hline
4               & quatre          & 14              & quatorze        \\ \hline
5               & cinq            & 15              & quinze          \\ \hline
6               & six             & 16              & seize           \\ \hline
7               & sept            & 17              & dix-sept        \\ \hline
8               & huit            & 18              & dix-huit        \\ \hline
9               & neuf            & 19              & dix-neuf        \\ \hline
10              & dix             & 20              & vingt           \\ \hline
\end{tabular}\end{center}

\subsubsection{Uses of Un}

The word un (or une in feminine) can be used in a number of ways.  As an indefinite article ("a" or "an"), it is used to modify countable nouns that are unspecified or unknown to the speakers.

\begin{itemize}
  \item  un livre \\ a book
  \item  un {\'e}l{\'e}phant \\ an elephant
\end{itemize}

As a numeral ("one"), it is a kind of adjective.

\begin{itemize}
  \item  J'ai une seule question. \\ I have only one question.
\end{itemize}

As a pronoun (``one''). Like in English, French numbers can be used as pronouns. In general, when you see a preposition like de after a number, that number acts as a pronoun.

\begin{itemize}
  \item  C'est un de mes enfants. \\ He is one of my children.
  \item  Je connais un de ces hommes. \\ I know one of those men.
\end{itemize}

Also, keep in mind that liaisons are forbidden before and after et.


\pagebreak
\subsection{Family}

Adults should use \guillemotleft~ p{\`e}re \guillemotright and \guillemotleft~ m{\`e}re \guillemotright when referring to parents. The juvenile forms, \guillemotleft~ papa \guillemotright and \guillemotleft~ maman\guillemotright , are generally used only by children, much like ``papa'' and ``mama'' or ``daddy'' and ``mommy'' in English.

You learned in \guillemotleft~ {\^E}tre-Avoir \guillemotright that you must often use the impersonal pronoun \guillemotleft~ ce \guillemotright when describing people and things with {\^e}tre. In general, use ce whenever {\^e}tre is followed by any determiner---for instance, an article or a possessive adjective. Remember that ce is invariable, so use c'est for singulars and ce sont for plurals.  This rule applies everywhere, including in questions, inversions, and subordinate clauses.

\begin{itemize}
  \item  C'est un animal ? \\ That's an animal?
  \item  Est-ce votre petit-fils ? \\ Is he your grandson?
  \item  Vous l'aimez parce que c'est votre fils. \\ You love him because he is your son.
\end{itemize}

The personal pronoun il should only be used with {\^e}tre when they're followed by an adjective and/or adverb.

\begin{itemize}
  \item  Il est fort. \\ He is strong.
  \item  Est-elle forte ? \\ Is she strong?
  \item  Est-ce qu'il est content ? \\ Is he happy?
\end{itemize}

In the last example, note that est-ce still appears because est-ce que is a fixed impersonal phrase.

\begin{center}\begin{tabular}{l|l||l|l}
\textbf{Fran{\c c}ais} & \textbf{English} & \textbf{Fran{\c c}ais} & \textbf{English} \\ \hline
\Blue{le p{\`e}re} & father & \Blue{le b{\'e}b{\'e}} & baby \\ 
\Red{le m{\`e}re} & mother & \Blue{le oncle} & uncle \\ 
\Red{le famille} & family & \Red{le tante} & aunt \\ 
\Blue{le fils} & son & \Red{le grand-m{\`e}re} & grandmother \\ 
\Blue{le pr{\'e}nom} & [first] name & \Blue{le grand-p{\`e}re} & grandmother \\ 
\Red{le s{\oe}ur} & sister & \Purple{cousin} & cousin \\ 
\Blue{le fr{\`e}re} & brother & \Blue{le neveu} & nephew \\ 
\Blue{le papa} & daddy & \Blue{le petit-fils} & grandson \\ 
\Red{le maman} & mama & \Blue{le parents} & parents \\ 
\Blue{le mari} & husband & \Blue{le mariage} & marriage \\ 
\Blue{le couple} & couple \\
\end{tabular}\end{center}

Note that ``daughter'' is \guillemotleft~ ma fille \guillemotright and ``wife'' is \guillemotleft~ ma femme \guillemotright .


\pagebreak
\subsection{Possessives 2}

\textbf{Possessive pronouns} replace a possessive adjective + a noun. Like most other pronouns, they agree in gender and number with the noun they replace.

\begin{itemize}
  \item  Est-ce ton chapeau ? \\ Is that your hat?
  \item  Oui, c'est le mien. \\ Yes, it's mine.
\end{itemize}

For one owner, the forms of possessive pronouns follow a simple pattern:

\begin{center}\begin{tabular}{|c|c|c|c|}
\hline
\textbf{Person} & \textbf{English} & \textbf{Masc. Sing.} & \textbf{Fem. Sing.} \\ \hline
1st             & mine             & le mien              & la mienne           \\ \hline
2nd             & yours            & le tien              & la tienne           \\ \hline
3rd             & his/hers         & le sien              & la sienne           \\ \hline
\end{tabular}\end{center}

\begin{itemize}
  \item  J'ai mon livre. As-tu le tien ? \\ I have my book. Do you have yours?
  \item  Ma ceinture est rouge. La sienne est blanche. \\ My belt is red. His (or ``hers'') is white.
\end{itemize}

For multiple owners, the articles vary with gender, but the pronouns do not:

\begin{center}\begin{tabular}{|c|c|c|c|}
\hline
\textbf{Person} & \textbf{English} & \textbf{Sing. Masc.} & \textbf{Sing. Fem.} \\ \hline
1st             & ours             & le n{\^o}tre             & la n{\^o}tre            \\ \hline
2nd             & yours            & le v{\^o}tre             & la v{\^o}tre            \\ \hline
3rd             & theirs           & le leur              & la leur             \\ \hline
\end{tabular}\end{center}

\begin{itemize}
  \item  Vous mangez votre repas et nous mangeons le n{\^o}tre. \\ You eat your meal and we eat ours.
  \item  Vous aimez notre voiture et nous aimons la v{\^o}tre. \\ You like our car and we like yours. 
\end{itemize}

The 2nd-person articles for multiple owners can be used for a single owner when speaking formally.  Notice that you must use c'est with possessive pronouns, not il est, elle est, etc.

\begin{itemize}
  \item  Informal, one owner: C'est le tien.
  \item  Formal, one owner: C'est le v{\^o}tre.
  \item  Multiple owners: C'est le v{\^o}tre.
\end{itemize}

The definite article at the beginning of a possessive pronoun can contract with à or de.

\begin{itemize}
  \item  Tu t{\'e}l{\'e}phones {\`a} ton p{\`e}re et je t{\'e}l{\'e}phone au mien. \\ You are calling your dad and I am calling mine.
  \item  J'aime mon repas. Qu'est-ce que vous pensez du v{\^o}tre ? \\ I like my meal. What do you think of yours?
\end{itemize}


\pagebreak
\subsection{Demonstratives 2}

Ceci (``this'') and cela (``that'') are the formal versions of the indefinite demonstrative pronoun {\c c}a (``this'' or ``that''). These are used when pointing something out, referring to something indefinite (like an idea), or referring back to something already mentioned.

\begin{itemize}
  \item  Je connais cela. \\ I know about that.
  \item  Je veux ceci. \\ I want this.
\end{itemize}

Ceci is usually only used when making a distinction between ``this'' and ``that''. Otherwise, cela is preferred in writing and {\c c}a is preferred in speech.  Remember that ce can only be used with {\^e}tre, including devoir {\^e}tre and pouvoir {\^e}tre.

\begin{itemize}
  \item  C'est un tr{\`e}s bon vin ! \\ This is a really good wine!
  \item  Ce doit {\^e}tre ton fils. \\ It must be your son. 
\end{itemize}

However, cela and ceci can also be used with {\^e}tre for emphasis.

\begin{itemize}
  \item  C'est le mien. \\ It's mine.
  \item  Non, ceci est le mien. Cela est le tien. \\ No, \textit{this} is mine. \textit{That} is yours.
\end{itemize}

Cela/ceci/{\c c}a should be used with all other verbs.

\begin{itemize}
  \item  Cela arrive souvent. \\ It happens often. / That happens often.
  \item  Ceci contient un bonbon. \\ This contains a candy.
\end{itemize}

\subsubsection{Demonstrative Pronouns}

\textbf{Demonstrative pronouns} (e.g. ``this one'', ``that one'', ``these'', ``those'') replace a demonstrative adjective + noun for the sake of avoiding repetition. Like most other pronouns, they agree in gender and number with the noun they replace.

\begin{center}\begin{tabular}{|c|c|c|}
\hline
\textbf{Type}      & \textbf{Adjective + Noun $\Rightarrow$ Pronoun} & \textbf{English}         \\ \hline
masculine singular & ce + noun $\Rightarrow$ celui                   & the one / this / that    \\ \hline
masculine plural   & ces + noun $\Rightarrow$ ceux                   & the ones / these / those \\ \hline
feminine singular  & cette + noun $\Rightarrow$ celle                & the one / this / that    \\ \hline
feminine plural    & ces + noun $\Rightarrow$ celles                 & the ones / these / those \\ \hline
\end{tabular}\end{center}

Demonstrative pronouns refer to a very specific thing and cannot stand alone. They must be used in one of three constructions.

\begin{enumerate}
  \item  \textbf{Demonstrative pronoun + relative pronoun}  A relative pronoun and dependent clause can follow the demonstrative pronoun. For instance, you can use que when the relative pronoun is the direct object and use qui when it's the subject.
  
  \begin{itemize}
  \item  Celui qui est dans ma poche. \\ The one that is in my pocket.
  \item  Ceux que je connais. \\ The ones that I know. / The ones whom I know.
  \end{itemize}
  
  \item  \textbf{Demonstrative pronoun + preposition}  The preposition de can appear after the demonstrative pronoun to indicate possession.
  
  \begin{itemize}
  \item  {\`A} qui est cette balle ? \\ Whose ball is this?
  \item  C'est celle du chien. \\ It's the dog's. (Literally: ``It is the one of the dog.'')
  \end{itemize}    
  
  \item  \textbf{Demonstrative pronoun + suffix}  [This construction will appear in ``Demonstratives 3''.]
\end{enumerate}

\subsubsection{Examples}

Demonstrative pronouns are often used in comparisons or choices between alternatives.

\begin{itemize}
  \item  Ce tableau est moins beau que celui de Rembrandt. \\ This painting is less beautiful than that by Rembrandt.
  \item  Quelle robe pr{\'e}f{\`e}res-tu ? Celle de Paris ou celle de Tokyo ? \\ Which dress do you prefer? The one from Paris or the one from Tokyo?
\end{itemize}

They can also be used within prepositional phrases. 

\begin{itemize}
  \item  Je pense {\`a} celles qui sont en vacances. \\ I am thinking about the ones who are on vacation.
  \item  Ce repas est pour ceux qui aiment les oignons. \\ This meal is for those who like onions.
\end{itemize}


\pagebreak
\subsection{Dates and Time}

In French, the present tense can often be used to describe something that will happen soon.

\begin{itemize}
  \item  Je vous appelle demain. \\ I [will] call you tomorrow.
  \item  On se voit demain. \\ We [will] see each other tomorrow.
\end{itemize}

This also occurs in English, albeit less frequently.

\begin{itemize}
  \item  {\c C}a commence demain. \\ That begins tomorrow.
\end{itemize}

\subsubsection{Describing Dates}

The most formal way to express a date in French is with c'est. (Never use il est.)

\begin{itemize}
  \item  C'est dimanche. \\ It's Sunday.
\end{itemize}

However, the most common way is to use nous sommes or on est. This construction is idiomatic and does not directly translate to English.

\begin{itemize}
  \item  Nous sommes vendredi. \\ It is Friday.
  \item  Aujourd'hui, on est mardi. \\ Today is Tuesday.
\end{itemize}

Note that while ``today'' is a noun and adverb in English, aujourd'hui cannot be used as a noun to give a date, so you cannot say Aujourd'hui est mardi. However, hier, aujourd'hui, and demain can be used as nouns when qualified by an adjective or another noun.

\begin{itemize}
  \item  Demain est un autre jour. \\ Tomorrow is another day.
  \item  Hier {\'e}tait f{\'e}ri{\'e}. \\ Yesterday was a holiday.
\end{itemize}

This construction can be used to express the month, though you must add en. Months aren't capitalized in French.

\begin{itemize}
  \item  Nous sommes en juillet. \\ It's July.
\end{itemize}

When denoting specific dates, put le and the date before the month. Also, French date abbreviations take the form DD/MM/YY.

\begin{itemize}
  \item  27/11/14: C'est le 27 novembre 2014. \\ It's November 27, 2014.
  \item  02/10: Nous sommes le 2 octobre. \\ It's October 2nd.
\end{itemize}

However, for the first day of the month, you must use the word premier.

\begin{itemize}
  \item  01/04: C'est le premier avril. \\ It's April 1st.
\end{itemize}

To express a relative time in the past, you can use il y a.

\begin{itemize}
  \item  il y a huit jours \\ eight days ago
  \item  il y a deux ans \\ two years ago
\end{itemize}

\subsubsection{Jour or Journ{\'e}e?}

A few words for dates and times have both masculine and feminine forms that are used in different contexts.

\begin{center}\begin{tabular}{|c|c|c|}
\hline
\textbf{English} & \textbf{Masculine} & \textbf{Feminine} \\ \hline
day              & jour               & journ{\'e}e           \\ \hline
morning          & matin              & matin{\'e}e           \\ \hline
evening          & soir               & soir{\'e}e            \\ \hline
year             & an                 & ann{\'e}e             \\ \hline
\end{tabular}\end{center}

Consider the meaning of the whole sentence when deciding between the two. Some pairs are more flexible than others. Jour and journ{\'e}e can often be interchangeable, but matin and matin{\'e}e are very strictly separate.  The masculine forms are used for countable units of time and specific dates or moments. For instance:

\begin{itemize}
  \item  With numerals (except un in some cases).
    \begin{itemize}
      \item  deux ans---two years
      \item  trois jours---three days
    \end{itemize}
  \item  With temporal adverbs (e.g. demain and hier).
    \begin{itemize}
      \item  demain matin---tomorrow morning
      \item  hier soir---yesterday evening / last night
    \end{itemize}
\end{itemize}

The feminine forms are used to express or emphasize a duration or the passing of time. They're also used with most adjectives. For instance:

\begin{itemize}
  \item  When emphasizing a duration.
    \begin{itemize}
      \item  Je vais lire toute la matin{\'e}e.---I am going to read all morning.
      \item  la journ{\'e}e de 8 heures---the 8-hour day
    \end{itemize}
  \item  With adjectives (except tous/chaque/ce).
    \begin{itemize}
      \item  une belle soir{\'e}e---a beautiful evening
      \item  Cette ann{\'e}e est m{\'e}morable.---This year is memorable.
    \end{itemize}
\end{itemize}

Deciding between forms with un depends on whether un acts as a numeral or article. If you can translate un as ``one'' in English, then go with the masculine.  Notice that chaque matin doesn't require an article but tous les matins does. This is because chaque, ce, and articles are all examples of determiners, which are words that give context to nouns. You will learn more about determiners in ``Adjectives 3''.

\begin{itemize}
  \item  Les jours de la semaine sont lundi, mardi, mercredi, jeudi, vendredi, samedi, et dimanche. \\ The days of the week are Monday, Tuesday, Wednesday, Thursday, Friday, Saturday, and Sunday.
  \item  Les douze mois de l'ann{\'e}e sont janvier, f{\'e}vrier, mars, avril, mai, juin, juillet, ao{\^u}t, septembre, octobre, novembre, et d{\'e}cembre. \\ The 12 months of the year are January, February, March, April, May, June, July, August, September, October, November, and December.
  \item  Les saisons de l'ann{\'e}e sont printemps, {\'e}t{\'e}, automne, et hiver. \\ The season of the year are spring, summer, autumn, and winter.
\end{itemize}

\begin{center}\begin{tabular}{l|l||l|l}
\textbf{Fran{\c c}ais} & \textbf{English} & \textbf{Fran{\c c}ais} & \textbf{English} \\ \hline
\Blue{le temps} & time & \Blue{le anniversaire} & birthday \\ 
aujourd'hui & today & \Red{la date} & date \\ 
\Blue{le mois} & month & \Red{la date} & date \\ 
\Blue{le petit d{\'e}jeuner} & breakfast & \Red{la naissance} & birth \\ 
\Blue{le d{\'e}jeuner} & lunch & \Red{la jeunesse} & youth \\ 
\Blue{le midi} & noon & \Blue{le apr{\'e}s-midi} & afternoon \\ 
\Blue{le si{\`e}cle} & century & \Red{la nuit} & night \\ 
\Red{la semaine} & week & \Blue{le minuit} & midnight \\ 
\Red{la dur{\'e}e} & duration & \Blue{le d{\"{\i}}ner} & dinner \\ 
\Blue{le d{\'e}but} & beginning & \Red{la montre} & watch \\ 
\Red{la p{\'e}riode} & period & \Blue{le calendrier} & calendar \\ 
\Red{la minute} & minute & \Blue{le dem}i & half (in measurements) \\ 
\Blue{le hier} & yesterday & \Blue{le quart} & quarter \\ 
\Red{las vacances} & vacation & moins & minus \\ 
\Red{la f{\'e}te} & celebration & \Red{la huere} & hour \\ 
\Blue{le second} & second & \Blue{le moment} & moment \\ 
\end{tabular}\end{center}

\begin{itemize}
  \item  Mon petit fr{\'e}re a douze ans. \\ My little brother is 12 years old.
  \item  Dans un jour. \\ In one day.
  \item  Bonne ann{\'e}e ! \\ Happy New Year!
  \item  Au d{\'e}but c'est difficile.
  \item  Dans la minute qui suit. \\ In the next minute.
  \item  De mars {\`a} mai. \\ From March to May.
  \item  Rendez-vous en octobre. \\ Let's meet in October.
  \item  Il est huit heures et quart. \\ It is a quarter past eight.
  \item  Il est midi moins le quart. \\ It is a quarter before noon.
  \item  Avez-vous l'huere ? \\ Do you have the time?
\end{itemize}


\pagebreak
\subsection{Infinitives}

Verb conjugations are classified in two ways: tense and mood. Tenses reflect a time frame (e.g. present tense), while moods reflect a speaker's attitude. So far, you've mainly used the \textbf{indicative mood} (for facts and certainties), but it is only one of seven moods.

\subsubsection{The Infinitive Mood}

The \textbf{infinitive mood} is an impersonal mood that isn't conjugated nor associated with any subject pronoun. It can be used in a variety of constructions, either with or without prepositions.

\begin{itemize}
  \item  Without prepositions:  Infinitives are often the objects of conjugated semi-auxiliary verbs such as vouloir, pouvoir, and aimer. You learned this in ``Verbs: Present 1''.
    \begin{itemize}
       \item  {\c C}a va venir. \\ It is going to come
       \item  Je veux danser. \\ I want to dance.
       \item  J'aime avoir un chat. \\ I like having a cat.
    \end{itemize}
    Infinitives can also act like nouns and can be used as subjects.
    \begin{itemize}
       \item  Faire du caf{\'e} est facile. \\ Making coffee is easy.
       \item  Cuisiner et nettoyer sont ses responsabilit{\'e}s. \\ Cooking and cleaning are his responsibilities.
    \end{itemize}
    Here, note that French infinitives can often be translated as English gerunds (with an -ing ending), especially when they're subjects.
  \item  After verbs + prepositions:  As you learned previously, some verbs must be followed by a preposition to complete their meaning (e.g. penser {\`a}). An infinitive can be used as an object when it follows such prepositions.
    \begin{itemize}
       \item  Elle parle de cuisiner le poulet. \\ She is talking about cooking the chicken.
       \item  Je pense {\`a} changer de job. \\ I am thinking about changing jobs.
       \item  Je vous remercie de laver les verres. \\ I thank you for washing the glasses.
    \end{itemize}
    Since infinitives can act like nouns, they can follow {\^e}tre + de to describe or define a subject (as a subject complement).
    \begin{itemize}
       \item  Mon travail est de cuisiner. \\ My job is to cook.
       \item  L'objectif est d'apprendre le francais. \\ The goal is to learn French.
    \end{itemize}
    The preposition pour (``for'' or ``in order to'') can come before an infinitive to express the purpose of an action.
    \begin{itemize}
       \item  Je lis pour apprendre. \\ I read [in order] to learn.
       \item  Je viens pour parler. \\ I am coming [in order] to talk.
    \end{itemize}
    Keep in mind that conjugated verbs should never come after prepositions.
  \item  After nouns:  An infinitive can also modify a noun when used with de or {\`a}. It may take practice to decide which preposition should be used, but in general, use de whenever the infinitive has an object.
    \begin{itemize}
       \item  Merci de laver les verres. \\ Thanks for washing the glasses.
       \item  Il prend le temps de manger une pomme. \\ He takes the time to eat an apple.
    \end{itemize}
    Use {\`a} when the verb in the sentence is avoir (with the translation ``to have'').
    \begin{itemize}
       \item  J'ai une d{\'e}cision {\`a} prendre. \\ I have a decision to make.
       \item  Il a un examen {\`a} pr{\'e}parer \\ He has an exam to prepare.
    \end{itemize}
    {\`A} can also be used to indicate the purpose of a noun.
    \begin{itemize}
       \item  une maison {\`a} vendre \\ a house for sale
       \item  l'eau {\`a} boire \\ rinking water
    \end{itemize}
  \item  After adjectives:  Infinitives can be used with the construction il est + adjective + de to create impersonal expressions. Remember from ``Common Phrases'' that an impersonal statement is one with a dummy subject instead of a real one.
    \begin{itemize}
       \item  Il est possible de manger maintenant. \\ It is possible to eat now.
       \item  Il est n{\'e}cessaire de boire de l'eau. \\ It is necessary to drink water.
    \end{itemize}
    However, if the subject il is a real thing instead of just a dummy subject, then you must use {\`a} instead of de.
    \begin{itemize}
       \item  Cette t{\^a}che est facile {\`a} faire. \\ This task is easy to do.
       \item  C'est bon {\`a} savoir. \\ That's good to know.
    \end{itemize}
    To further illustrate the difference, consider these two different translations of ``It is fun to read.'' The first is a general statement, while the second is a statement about a real subject.
    \begin{itemize}
       \item  Il est amusant de lire. (Impersonal) \\ It is fun to read. / Reading is fun.
       \item  Il est amusant {\`a} lire. (Real) \\ It (e.g. a book) is fun to read.
    \end{itemize}
\end{itemize}

\subsubsection{Causative Faire}

Faire often appears before a verb to indicate that the subject causes something to happen instead of performing it. It's often used in relation to foods.

\begin{itemize}
  \item  Il fait bouillir le th{\'e}. \\ He boils the tea.
  \item  J'aime faire griller du poulet. \\ I like grilling chicken.
\end{itemize}

It can also be used to indicate that the subject has directed someone else to perform an action.

\begin{itemize}
  \item  Je le fais r{\'e}parer. \\ I am having it fixed. 
  \item  Je fais partir mon ami. \\ I am making my friend leave.
\end{itemize}


\pagebreak
\subsection{Adverbs 1}

Adverbs are invariable words that can modify verbs, adjectives, other adverbs, and more.  If an adverb modifies a verb, it usually follows right after it.

\begin{itemize}
  \item  Il parle vite. \\ He speaks quickly.
  \item  Elle mange souvent de la soupe. \\ She often eats soup.
  \item  J'aime bien l'hiver. \\ I like the winter.
\end{itemize}

An adverb comes before an adjective or other adverb that it modifies.

\begin{itemize}
  \item  Je suis tr{\`e}s heureux. \\ I am very happy.
  \item  Ma cuill{\`e}re est trop grande ! \\ My spoon is too big!
\end{itemize}

A long adverb that modifies a phrase can usually be relegated to the beginning or end of a sentence.

\begin{itemize}
  \item  Ton fils est un homme maintenant. \\ Your son is a man now.
  \item  G{\'e}n{\'e}ralement, je sais quoi faire. \\ Generally, I know what to do.
\end{itemize}

\subsubsection{Adverbs of Quantity}

Imprecise quantities are expressed using adverbs of quantity, which are usually followed by the preposition de.

\begin{itemize}
  \item  Il a beaucoup de chiens. \\ He has a lot of dogs.
  \item  Il boit trop de bi{\`e}re. \\ He is drinking too much beer.
\end{itemize}

Recall that du, de la, and des cannot be used after expressions ending in de, such as adverbs of quantity. Thus, des does not appear before chiens and de la does not appear before bi{\`e}re. However, other articles can follow adverbs of quantity when the noun is specific.

\begin{itemize}
  \item  Beaucoup des (de + les) amis de mon fr{\`e}re sont l{\`a}. \\ Many of my brother’s friends are here.
  \item  Je veux plus du (de + le) m{\^e}me. \\ I want more of the same.
\end{itemize}

\subsubsection{Comparatives and Superlatives}

The adverbs plus (``more'') and moins (``less'') can be used with the conjunction que in comparisons.

\begin{itemize}
  \item  Ta s{\oe}ur est plus jolie qu'elle. \\ Your sister is prettier than her.
  \item  Ils mangent moins que nous. \\ They are eating less than us.
\end{itemize}

To express equivalence, use aussi...que (``as...as'').

\begin{itemize}
  \item  Je suis aussi timide que mon p{\`e}re. \\ I am as shy as my father.
\end{itemize}

Adding a definite article before plus or moins creates a superlative. The definite article agrees with the noun being modified.

\begin{itemize}
  \item  C'est la plus jolie robe. \\ That's the prettiest dress.
  \item  Le plus grand arbre du monde est l{\`a}. \\ The biggest tree in the world is there.
\end{itemize}

If the adjective should follow the noun, then the definite article must be repeated.

\begin{itemize}
  \item  Je veux acheter le pain le moins cher. \\ I want to buy the least expensive bread.
  \item  C'est le livre le plus difficile {\`a} comprendre. \\ That's the most difficult book to understand.
\end{itemize}

\subsubsection{Bon, Bien, Mauvais, et Mal}

In French, we have to deal with the good (bon and bien), the bad (mauvais and mal), and the ugly (trying to decide which to use). Luckily, in most cases, bon and mauvais are adjectives while bien and mal are adverbs.\footnote{L'anglais, ce n'est jamais que du fran{\c c}ais mal prononc{\'e}. \\ ---Georges Clemenceau, ``English is nothing but mispronounced French.''}

\begin{itemize}
  \item  C'est un bon chanteur. \\ He is a good singer.
  \item  Il chante bien. \\ He sings well.
  \item  Elle est bonne {\'e}tudiante. \\ She's a good student.
  \item  Elle {\'e}tudie bien. \\ She studies well.
  \item  C'est un mauvais homme. \\ He's a bad man.
  \item  Mon fr{\`e}re lit tr{\`e}s mal. \\ My brother reads very badly.
  \item  Tu bois le mauvais vin ! \\ You're drinking the wrong wine!
\end{itemize}

There are also a number of fixed expressions or special usages for bien. You are familiar with some of these from ``Common Phrases''.

\begin{itemize}
  \item  Bien ! \\ Good!
  \item  C'est tr{\`e}s bien ! \\ That's very good!
  \item  Bien s{\^u}r. \\ Of course.
\end{itemize}

Also, remember that aimer normally means ``to love'' when directed at people and animals, but adding bien reduces its meaning to ``to like''.

\begin{itemize}
  \item  Elle l'aime. \\ She loves him.
  \item  J'aime bien mon ami. \\ I like my friend.
\end{itemize}

\begin{center}\begin{tabular}{l|l||l|l}
\textbf{Fran{\c c}ais} & \textbf{English} & \textbf{Fran{\c c}ais} & \textbf{English} \\ \hline
plus & more & assez & enough \\ 
moins & less & beaucoup & a lot \\ 
bien & well & maintenant & now \\ 
l{\`a} & here & vite & quickly \\ 
aussi & also & souvent & often \\ 
mal & bad & tard & late \\ 
tr{\`e}s & very & t{\^o}t & early \\ 
trop & too & voici & Here \\ 
ici & here & voil{\`a} & there \\ 
\end{tabular}\end{center}


\pagebreak
\subsection{Occupations}

Remember that occupations (along with nationalities and religions) can act as adjectives when used with {\^e}tre or devenir, so unlike in English, the French often drop the indefinite article (un, une, etc.) before an occupation.

\begin{itemize}
  \item  Je suis juge. \\ I am a judge.
  \item  Elle va devenir avocate. \\ She is going to become a lawyer.
\end{itemize}

However, if any specification follows the occupation, then the indefinite article must be added.

\begin{itemize}
  \item  Tu es un juge respect{\'e} par tous. \\ You are a judge respected by all.
  \item  Il veut devenir un professeur pour adultes. \\ He wants to become a teacher for adults.
\end{itemize}

Omitting the indefinite article is optional. However, if it's included in the third-person, then you must use c'est or ce sont.

\begin{itemize}
  \item  C'est un juge. \\ He's a judge.
  \item  C'est une dentiste bien connue. \\ She is a well-known dentist.
  \item  Ce sont des journalistes. \\ They are journalists.
\end{itemize}

\subsubsection{Genders in Occupations}

Some occupations have the same form in both masculine and feminine.

\begin{center}\begin{tabular}{l|l||l|l}
\textbf{Fran{\c c}ais} & \textbf{English} & \textbf{Fran{\c c}ais} & \textbf{English} \\ \hline
un/une auteur & an author  & un/une professeur & a teacher \\ 
un/une docteur & a doctor & un/une dentiste & a dentist \\ 
un/une juge & a judge & un/une secr{\'e}taire & a secretary \\ 
un/une journaliste & a journalist & un/une ing{\'e}nieur & an engineer  \\ 
un/une p{\'e}diatre & a pediatrician \\ 
\end{tabular}\end{center}

Other occupations have a feminine form that's derived from the masculine:

\begin{center}\begin{tabular}{|c|c|c|}
\hline
\textbf{Masculine} & \textbf{Feminine} & \textbf{English} \\ \hline
un policier        & une polici{\`e}re     & a police officer \\ \hline
un agriculteur     & une agricultrice  & a farmer         \\ \hline
un avocat          & une avocate       & a lawyer         \\ \hline
un enseignant      & une enseignante   & a teacher        \\ \hline
un serveur         & une serveuse      & a server         \\ \hline
un cuisinier       & une cuisini{\`e}re    & a cook           \\ \hline
un coiffeur        & une coiffeuse     & a hairdresser    \\ \hline
un boulanger       & une boulang{\`e}re    & a baker          \\ \hline
un serveur & une serveuse & a server \\ \hline
un chauffeur & une chauffeuse & a driver \\ \hline
\end{tabular}\end{center}

\begin{center}\begin{tabular}{l|l||l|l}
\textbf{Fran{\c c}ais} & \textbf{English} & \textbf{Fran{\c c}ais} & \textbf{English} \\ \hline
\Blue{le roi} & king & \Blue{le prince} & prince \\ 
\Red{la reine} & queen & \Red{la princesse} & princess \\ 
\Blue{le emploi} & job & \Blue{le soldat} & soldier \\ 
\Blue{le m{\'e}tier} & profession & \Red{la carri{\`e}re} & career \\ 
\Red{la police} & police & \Blue{le chef} & boss \\ 
\Blue{le travail} & work & \Blue{le personnel} & staff \\ 
\end{tabular}\end{center}

\begin{itemize}
  \item  Mon p{\`e}re est {\`a} la retraite. \\ My father is retired.
\end{itemize}


\pagebreak
\subsection{Negation}

A negation changes the meaning of a statement to its negative. Most French negations are constructed out of two words that surround a conjugated verb.

\begin{itemize}
  \item  Je ne comprends pas. \\ I don't understand.
  \item  Il ne parle pas anglais. \\ He doesn't speak English.
\end{itemize}

Note that the particle ne elides before vowel sounds.

\begin{itemize}
  \item  Vous n'avez pas de chien. \\ You don't have a dog.
  \item  Ils n'aiment pas le menu. \\ They don't like the menu.
\end{itemize}


Along with ne...pas, there are a number of other negations you can use.

\begin{itemize}
  \item  Ne...plus: not any more/no more/not any longer/no longer
    \begin{itemize}
      \item  Elle n'a plus de lait. \\ She no longer has milk.
      \item  Il ne peut plus marcher. \\ He can't walk any longer.
    \end{itemize}
  \item  Ne... jamais: not ever/never
    \begin{itemize}
      \item  Je ne sais jamais. \\ I never know.
      \item  Je ne gagne jamais. \\ I don't ever win.
    \end{itemize}
  \item  Ne... rien: not anything/nothing 
    \begin{itemize}
      \item  Je n'ai rien. \\ I have nothing.
      \item  Elles ne voient rien. \\ They don't see anything.
    \end{itemize}
  \item  Ne... personne: not anybody/nobody/not anyone/no one
    \begin{itemize}
      \item  Je ne vois personne. \\ I don't see anybody.
      \item  Il ne veut voir personne. \\ He doesn't want to see anyone.
    \end{itemize}
\end{itemize}

Note that in negations, direct objects preceded by indefinite and partitive articles change to de.

\begin{itemize}
  \item  Elle n'a pas de lait. \\ She doesn't have milk. (Not du lait.)
  \item  Je n'entends plus de bruit. \\ I don't hear a sound anymore. (Not un bruit.)
  \item  Je n'entends plus d'oiseaux. \\ I don't hear birds anymore. (Not des oiseaux.)
\end{itemize}

Since {\^e}tre does not have direct objects, all articles may be used.

\begin{itemize}
  \item  Ce liquide n'est pas du lait. \\ This liquid isn't milk.
  \item  Ce n'est pas un couteau. \\ That's not a knife.
\end{itemize}

\subsubsection{Negative Pronouns and Conjunctions}

In addition to the negative adverbs above, you also have the option of starting a sentence with a negative word, which acts like a masculine subject. Both personne and rien can also be negative subject pronouns if you put ne after them.

\begin{itemize}
  \item  Personne ne means ``nobody''.
    \begin{itemize}
      \item  Personne ne sait. \\ Nobody knows.
      \item  Personne n'aime cela. \\ Nobody likes that.
    \end{itemize}
  \item  Rien ne (``nothing'') is the pronoun version of ne...rien.
    \begin{itemize}
      \item  Rien n'est parfait. \\ Nothing is perfect.
      \item  Rien n'est si dangereux qu'un ignorant ami. (Jean de La Fontaine) \\ Nothing is so dangerous as an ignorant friend.
    \end{itemize}
\end{itemize}

The negative conjunction ni can be used to add something to a negation and is similar to the English ``nor''. Think of it as a negative form of et (``and''). Ni can be used instead of negative adverbs or in addition to them.

\begin{itemize}
  \item  Elle ne conna{\^{\i}}t ni toi ni moi. \\ She knows neither you nor me. (Or "She doesn't know you or me.")
  \item  Je ne veux ni ce repas ni cette boisson. \\ I want neither this meal nor this drink.
  \item  Il ne fait pas chaud ni froid. \\ It is neither hot nor cold.
\end{itemize}

When ni coordinates multiple conjugated verbs, each verb must be preceded by ne.

\begin{itemize}
  \item  Je ne lis pas, ni n'{\'e}cris. \\ I don't read or write.
  \item  Il ne veut ni ne peut manger de la colle. \\ He neither wants nor is able to eat glue.
\end{itemize}

\subsubsection{Word Order}

When the negated verb has a pronoun object, it belongs right after ne.

\begin{itemize}
  \item  Je ne l'aime pas. \\ I don't like it.
  \item  Je n'en ai pas. \\ I don't have any. (Lit. ``I do not have some of it.'')
\end{itemize}

When a negation is used with an inversion (to ask a question), the whole inversion must remain inside the negation.

\begin{itemize}
  \item  Ne comprenez-vous pas ? \\ Don't you understand?
  \item  Pourquoi ne l'as-tu pas ? \\ Why don't you have it?
\end{itemize}

Unconjugated verbs like infinitives must come after the negation.

\begin{itemize}
  \item  Ne pas toucher. \\ Do not touch.
  \item  Elle choisit de ne pas manger. \\ She chooses not to eat.
\end{itemize}

Extra adverbs that modify the verb usually come after the negation. Otherwise, they follow the rules from ``Adverbs 1''.

\begin{itemize}
  \item  On ne marche pas vite. \\ We aren't walking quickly.
  \item  Elle ne vient jamais ici. \\ She never comes here.
\end{itemize}

\subsubsection{Miscellaneous}

In English, two negatives may make a positive, but in French, they usually don't. For instance, consider ne... jamais rien, which is ``never... anything'', not ``never... nothing''

\begin{itemize}
  \item  Ils ne vont jamais rien perdre. \\ They will never lose anything.
  \item  Elle ne mange jamais rien. \\ She never eats anything.
\end{itemize}

The particle ne is often skipped or slurred in casual speech. It's also omitted for short phrases that lack a verb.

\begin{itemize}
  \item  Pas si vite ! \\ Not so fast!
  \item  Pas de probl{\`e}me. \\ No problem.
\end{itemize}

Remember that verbs of appreciation (e.g. aimer) require the definite article in French. Negations are no different.

\begin{itemize}
  \item  Je n'aime pas le poisson. \\ I don't like fish. (Not Je n'aime pas de poisson.)
\end{itemize}


\pagebreak
\subsection{Subordinating Conjunctions}

In ``Conjunctions 1'', you learned about coordinating conjunctions, which link similar elements that have equal importance in a sentence. However, in complex sentences, one clause may be dependent on another.

\begin{itemize}
  \item  Il mange parce qu'il a faim. \\ He eats because he is hungry.
\end{itemize}

In this example, parce qu'il a faim (``because he is hungry'') is a dependent clause because it gives more information about the independent clause il mange (``he eats''). The dependent clause is introduced by parce que, which is a subordinating conjunction. Many subordinating conjunctions end in que.  Unlike coordinating conjunctions, subordinating conjunctions can begin sentences.

\begin{itemize}
  \item  Lorsque le gar{\c c}on mange, la fille mange. \\ When the boy eats, the girl eats.
  \item  Pendant que je lis, il {\'e}crit. \\ While I read, he is writing.
\end{itemize}

\subsubsection{Temporal Conjunctions}

Quand and lorsque both mean ``when'', but they aren't always interchangeable. Both can be used for temporal correlations, but lorsque refers to one particular instance, while quand can refer to one or multiple instances. Quand is also an adverb, so it can be used in questions. When in doubt, use quand.

\begin{itemize}
  \item  Je sortais quand/lorsque tu arrivais. \\ I was leaving when you were arriving.
  \item  Je mange quand j'ai faim. \\ I eat when (whenever) I am hungry.
  \item  Quand mangez-vous ? \\ When do you eat?
\end{itemize}

Alors que, pendant que, and tandis que can indicate simultaneit

\begin{itemize}
  \item  Je mange alors que tu manges. \\ I eat while you eat.
  \item  Pendant que tu bois, je bois. \\ While you drink, I drink.
\end{itemize}

Alors que and tandis que can also indicate a contrast or contradiction, though this is rare for tandis que.

\begin{itemize}
  \item  Elle est grande, alors que je suis petit. \\ She is tall, whereas I am short.
  \item  Je mange alors que je n'ai pas faim. \\ I am eating even though I am not hungry.
\end{itemize}

\subsubsection{Causal Conjunctions}

Parce que, car, and puisque all mean ``because'' and describe some kind of cause-and-effect relationship, but they aren't completely interchangeable.  Parce que is a subordinating conjunction that provides an explanation, motive, or justification.

\begin{itemize}
  \item  Elle lit parce qu'elle a un livre. \\ She is reading because she has a book.
  \item  Parce qu'elle est jeune, elle est jolie. \\ She is pretty because she is young.
\end{itemize}

Car is similar to parce que, but it's a coordinating conjunction and thus cannot begin a sentence or clause.

\begin{itemize}
  \item  Je mange du poulet, car j'aime la viande. \\ I am eating chicken because I like meat.
\end{itemize}

Puisque is a subordinating conjunction that means ``because'' or ``since'' and gives an already-known or obvious reason or justification.

\begin{itemize}
  \item  Puisque il pleut, j'ai un parapluie. \\ Since it's raining, I have an umbrella.
\end{itemize}

\subsubsection{Elisions with Si and Que}

Usually, only one-syllable words ending in -e can be elided, but the main exceptions are elle, si, and words ending in que. However, si only elides before il and ils, so you must write s'il, but cannot write s'elle.

\begin{itemize}
  \item  Un citron, sinon une orange. \\ A lemon, otherwise an orange.
  \item  S'il boit, je mange. \\ If he drinks, I eat.
  \item  D{\`e}s qu'elle parle, j'{\'e}cris. \\ As soon as she speaks, I write.
  \item  Vous mangez autant que vous voulez. \\ You eat as much as you want.
  \item  Je parle pendant que je mange. \\ I speak while I eat.
\end{itemize}


\pagebreak
\subsection{Adverbs 2}

In English, many adverbs are constructed from adjectives by adding ``-ly'' to the end. For instance, ``quick'' becomes ``quickly''. In French, add -ment to feminine adjectives to create adverbs.

\begin{itemize}
  \item  facile (easy) $\rightarrow$ facilement (easily)
  \item  forte (strong) $\rightarrow$ fortement (strongly) 
  \item  grande (great) $\rightarrow$ grandement (greatly)
\end{itemize}

However, if the masculine form of an adjective ends in -nt, do not use the feminine form to construct the adverb, but replace the masculine ending with -mment. 

\begin{itemize}
  \item  constant (constant) $\rightarrow$ constamment (constantly)
  \item  prudent (prudent) $\rightarrow$ prudemment (prudently)
\end{itemize}

\subsubsection{Adverbs with Negations}

In negative clauses, adverbs that would otherwise follow the verb usually appear after the negation.

\begin{itemize}
  \item  Nous ne vivons pas ensemble. \\ We don't live together.
  \item  Ce n'est pas si mauvais. \\ That isn't so bad.
\end{itemize}

\begin{center}\begin{tabular}{l|l||l|l}
\textbf{Fran{\c c}ais} & \textbf{English} & \textbf{Fran{\c c}ais} & \textbf{English} \\ \hline
g{\'e}n{\'e}ralement & generally & devant & in front \\ 
s{\^u}rement & surely & {\`a} peu pr{\`e}s & about [in number] \\ 
lentement & slowly & peut-{\^e}tre & perhaps \\ 
normalment & normally & seulement & only \\ 
rarement & rarely & presque & almost \\ 
suffisamment & sufficiently & toujours & still \\ 
ensemble & together & encore & again \\ 
au moins & at least & parfois & sometimes \\ 
en fait & in fact & pourtant & yet \\ 
tout {`a} fait & absolutely & apr{\`e}s & after \\ 
bref & in short & d{\'e}j{\`a} & already \\ 
\end{tabular}\end{center}

\begin{itemize}
  \item  Il est si grand ! \\ It is so big!
\end{itemize}


\pagebreak
\subsection{Household}

\begin{center}\begin{tabular}{l|l||l|l}
\textbf{Fran{\c c}ais} & \textbf{English} & \textbf{Fran{\c c}ais} & \textbf{English} \\ \hline
\Red{la maison} & house & \Blue{le meuble(s)} & furniture \\ 
\Red{la table} & table & \Red{la piscine} & swimming pool \\ 
\Blue{le lit} & bed & \Red{la douche} & shower \\ 
\Red{la chaise} & chair & \Blue{le escalier} & stair \\ 
\Red{la cuill{\`e}re} & spoon & \Blue{le r{\'e}frig{\'e}rateur} & refrigerator \\ 
\Blue{le couteau (les couteaux)} & knife (knives) & \Blue{le oreiller} & pillow \\ 
\Blue{le assiette} & plate & descend & to descend \\ 
\Blue{le fourchette} & fork & \Red{la t{\'e}l{\'e}vision} & television \\ 
\Red{la verre} & glass (container) & \Red{la prise} & [electrical] outlet \\ 
\Red{la bouteille} & bottle & \Red{la serviette} & napkin \\ 
\Blue{le t{\'e}l{\'e}phone} & telephone & \Red{la horloge} & clock \\ 
\Blue{le bol} & bowl & \Red{la couverture} & blanket \\ 
\Red{la tasse} & cup & \Blue{les rideaux} & curtains \\ 
\Blue{le couvert} & cutlery & \Blue{le outil} & tool \\ 
\Blue{le berceau} & crib & \Red{la baignoire} & bathtub \\ 
\Red{la cuisine} & kitchen & \Blue{le chauffage} & heater \\ 
\Red{la fen{\^e}tre} & window & \Blue{le savon} & pink \\ 
\Blue{le miroir} & mirror & \Blue{le plafond} & ceiling \\ 
\Red{la {\'e}ponge} & sponge & ferme & to close \\ 
\Blue{le canap{\'e}} & sofa & \Blue{le recette} & recipe \\ 
\Red{la lampe} & lamp & \Blue{le shampooing} & shampoo \\ 
\Blue{le bureau} & desk/office & \Red{la poubelle} & bin \\ 
\Red{la porte} & door & nettoyer & to clean \\ 
\Blue{le mur} & wall & \Blue{le drap} & bedsheet \\  
\Blue{le four} & oven & \Blue{le balcon} & balcony \\ 
\Blue{le toit} & roof & \Blue{le jouet} & toy \\ 
\Blue{le tapis} & carpet \\ 
\end{tabular}\end{center}


\pagebreak
\subsection{Objects}
\subsubsection{Cognates}

As you may have noticed, a lot of English vocabulary (vocabulaire) comes from French. This has created many etymological patterns that you can use to your advantage when learning new words. Consider the following suffix patterns:

\begin{itemize}
  \item  -aire $\rightarrow$ -ary
    \begin{itemize}
      \item  ordinaire---ordinary
      \item  dictionnaire---dictionary
    \end{itemize}
  \item  -eur $\rightarrow$ -er
    \begin{itemize}
      \item  chargeur---charger
      \item  serveur---server (waiter) 
    \end{itemize}
  \item  -tion / -sion $\rightarrow$ -tion
    \begin{itemize}
      \item  invitation---invitation
      \item  condition---condition
    \end{itemize}
  \item  -ment (noun) $\rightarrow$ -ment
    \begin{itemize}
      \item  le document---the document
      \item  le gouvernement---the government
    \end{itemize}
  \item  -ment (adverb) $\rightarrow$ -ly
    \begin{itemize}
      \item  probablement---probably
      \item  s{\^u}rement---surely
    \end{itemize}
  \item  -ique $\rightarrow$ -ical
    \begin{itemize}
      \item  logique---logical
      \item  {\'e}lectrique---electrical
    \end{itemize}
  \item  -able $\rightarrow$ -able / -ible
    \begin{itemize}
      \item  responsable---responsible
      \item  indispensable---indispensable
    \end{itemize}
\end{itemize}

\subsubsection{Noun Adjuncts}

Unlike English, French does not have noun adjuncts, which are nouns that modify other nouns. Instead, you must use de or another preposition to make one noun modify another.

\begin{itemize}
  \item  l'album de photos \\ photo album
  \item  la soupe de poulet \\ chicken soup 
  \item  le hockey sur gazon \\ field hockey
\end{itemize}

\begin{center}\begin{tabular}{l|l||l|l}
\textbf{Fran{\c c}ais} & \textbf{English} & \textbf{Fran{\c c}ais} & \textbf{English} \\ \hline
\Red{la chose} & thing & \Blue{le drapeau} & flag \\ 
\Blue{le objet} & object & \Red{la feuille} & leaf \\ 
\Blue{le arme} & weapon & \Blue{le cadeau} & gift \\ 
\Red{la bo{\^{\i}}te} & box & \Red{la valise} & suitcase \\ 
\Red{la croix} & cross & \Blue{le ordinateur} & computer \\ 
\Red{la pi{\`e}ce} & room & \Blue{le clavier} & keyboard \\ 
\Blue{le fil} & thread & \Blue{le adaptateur} & adapter \\ 
\Red{la page} & page & \Blue{le album} & album \\ 
\Red{la carte} & map & \Blue{le plateau} & tray \\ 
\Red{la radio} & radio & \Blue{le {\'e}cran} & screen \\ 
\Red{la caisse} & case (crate) & \Blue{le dictionnaire} & dictionary \\ 
\Blue{le document} & document & \Blue{le magazine} & magazine \\ 
\Red{la brosse} & brush & \Blue{le robot} & robot \\ 
\Blue{le cadre} & frame & \Blue{le stylo} & pen \\ 
\Red{las lunettes} & eyeglasses & \Red{la voiture} & car \\ 
\Blue{le parfum} & perfume & \Blue{le roue} & wheel \\ 
\Red{la photo} & photo & \Red{la enveloppe} & envelope \\ 
\Blue{le dossier} & case (investigation) & allumer & to light \\ 
\Blue{le linge} & laundry & \Red{la allumette} & match (of fire) \\ 
\Blue{le disque} & record & brancher & to connect \\ 
\Red{la paire} & pair & charger & to charge \\ 
\Red{la cl{\'e}} & key & \Blue{le chargeur} & charger \\ 
\Blue{le billet} & ticket & \Blue{le bougie} & candle \\ 
\Red{la poudre} & powder & {\'e}teins & to blow out \\ 
\end{tabular}\end{center}


\pagebreak
\subsection{Adjectives 3}

You learned in ``Basics 1'' that almost all nouns must be preceded by an article. This isn't entirely accurate. Rather, almost all nouns must be preceded by a determiner, which is a word that puts a noun in context. As of this unit, you will have encountered every type of determiner.

\begin{itemize}
  \item  Articles, as in le pantalon (``the pants'').
  \item  Possessive adjectives, as in ton cochon (``your pig'').
  \item  Cardinal numbers, as in deux chevaux (``two horses'').
  \item  Interrogative adjectives, as in quel chat ? (``which cat?'').
  \item  Exclamation adjectives, as in quelle chance ! (``what luck!'').
  \item  Negative adjectives, as in aucune chance (``no chance!'').
  \item  Indefinite adjectives, as in plusieurs jouets (``several toys'').
\end{itemize}

There are very few exceptions to the rule that nouns must have a determiner. A few are verb-based. For instance: names of professions, religions and a few nouns expressing a status with {\^e}tre; names of languages with parler; and most nouns with devenir.

\begin{itemize}
  \item  Je suis m{\'e}decin. \\ I am a doctor.
  \item  Il est bon {\'e}l{\`e}ve. \\ He is a good student.
  \item  Elle est victime de son succ{\`e}s. \\ She is a victim of her own success.
  \item  Paul {\'e}tait t{\'e}moin {\`a} mon mariage. \\ Paul was a witness at my wedding.
  \item  Je parle anglais. \\ I speak English.
  \item  Il devient roi du Nord. \\ He becomes King of the North.
\end{itemize}

Determiners are also omitted after some prepositions.

\begin{itemize}
  \item  Je ne peux pas vivre sans eau. \\ I cannot live without water.
  \item  Nous le transportons par avion. \\ We transport it by aircraft.
  \item  C'est une feuille de papier. \\ This is a sheet of paper.
\end{itemize}

Recall that French does not have noun adjuncts, which are nouns that qualify other nouns. Instead, use de between two nouns to qualify the first one.

\begin{itemize}
  \item  C'est un album de photos. \\ That's a photo album. (Literally, ``album of photos'')
  \item  Je vais {\`a} l'agence de voyage. \\ I am going to the travel agency.
  \item  Il a un couteau de cuisine. \\ He has a kitchen knife.
\end{itemize}

\subsubsection{Indefinite Articles}

Indefinite adjectives like plusieurs, certains, quelques, and chaque references nouns in a non-specific sense, akin to the way indefinite articles reference nouns.

\begin{itemize}
  \item  L'enfant a plusieurs jouets. \\ The child has several toys.
  \item  Certains hommes sont mauvais. \\ Some (or "certain") men are bad.
  \item  J'ai quelques livres. \\ I have a few (or "some") books.
  \item  L'automne est un deuxi{\`e}me printemps o{\`u} chaque feuille est une fleur. \\ Autumn is a second spring where every leaf is a flower. (Albert Camus)
\end{itemize}

\subsubsection{Comparatives and Superlatives}

In ``Adverbs 1'', you learned that you can use plus as a comparative and le/la/les plus as a superlative.

\begin{itemize}
  \item  C'est une plus jolie robe. \\ That's a prettier dress.
  \item  C'est la plus jolie robe. \\ That's the prettiest dress.
\end{itemize}

Bon (``good''), bien (``well''), and mauvais (``bad'') also have comparative and superlative forms, but they're irregular, just like their English counterparts.  To say ``better'' when referring to a noun, you can't just say plus bon. Instead, use meilleur, which is a BANGS adjective with four inflections.

\begin{center}\begin{tabular}{|c|c|c|}
\hline
\textbf{Gender} & \textbf{Singular} & \textbf{Plural} \\ \hline
Masculine       & meilleur          & meilleurs       \\ \hline
Feminine        & meilleure         & meilleures      \\ \hline
\end{tabular}\end{center}

\begin{itemize}
  \item  Elle cherche un meilleur emploi. \\ She is looking for a better job.
  \item  Je veux de meilleures robes. \\ I want better dresses. \footnote{Remember that des becomes de when immediately followed by an adjective.}
\end{itemize}

For the superlative, just add a definite article before the adjective that agrees with it.

\begin{itemize}
  \item  Paul est le meilleur. \\ Paul is the best.
  \item  Ses filles sont les meilleures. \\ Her daughters are the best.
\end{itemize}

When ``better'' modifies an action or state of being, you must use mieux.

\begin{itemize}
  \item  Il parle mieux japonais. \\ He speaks better Japanese.
  \item  {\c C}a va mieux. \\ It is going better.
\end{itemize}

Add a definite article to create a superlative.

\begin{itemize}
  \item  C'est Paul qui cuisine le mieux. \\ It's Paul who cooks the best.
  \item  Il les connait le mieux. \\ He knows them the best.
\end{itemize}

Unlike bon and bien, comparative and superlative forms of mauvais can either be regular (with plus) or irregular (with pire).

\begin{itemize}
  \item  C'est une plus mauvaise situation. \\ That's a worse situation.
  \item  {\c C}a peut être pire. \\ That might be worse.
  \item  Ce sont les pires choix. \\ Those are the worst choices.
\end{itemize}

\begin{center}\begin{tabular}{l|l||l|l||l|l}
\textbf{Fran{\c c}ais} & \textbf{English} & \textbf{Fran{\c c}ais} & \textbf{English}  & \textbf{Fran{\c c}ais} & \textbf{English} \\ \hline
pareil & same & fin & thin & possible & possible \\ 
r{\'e}el & real & efficace & efficient & prudent & careful \\ 
logique & logical & sauvage & savage & ancienne & old (thing) \\ 
{\'e}lectrique & electric & id{\'e}al & ideal & vieil & old (BANGS) \\ 
plat & flat & pr{\'e}cieux & precious & c{\'e}ibataire & single (not married) \\ 
ordinaire & ordinary & puissant & strong & bruyant & noisy \\ 
fixe & fixed & dangereux & dangerous & parfait & perfect \\ 
libre & free & fausse & fake & lourd & heavy \\ 
profond & deep & haut & upstairs & blonde & blonde \\ 
pr{\'e}cis & precise & {\^a}g{\'e}e & old (person) & brun & brunette \\ 
intuile & useless & tout & every & roux & red-headed \\ 
indispensable & indispensable & sp{\'e}cial & special & quatri{\`e}me & fourth \\ 
popularaie & popular & gras & fat & cinqui{\`e}me & fifth \\ 
prive{\'e} & private & sage & wise & rond & round \\ 
complet & complete & mari{\'e}e & married & facile & easy \\ 
solide & solid & tel & such & plusieur & several \\ 
pratique & practical & cru & raw & certain & certain \\ 
c{\'e}l{\`e}bre & famous & terrible & terrible  & quelque & few \\ 
vivante & alive & personnel & personal & chaque & every \\ 
int{\'e}ressant & interesting & incroyable & incredible & liquide & liquid \\ 
secret & secret & ind{\'e}pendant & independent & franc(he) & frank \\ 
ferm{\'e} & closed & normal & normal & enti{\'e}re & entire \\ 
responsable & responsible & magnifique & magnificent \\ 
\end{tabular}\end{center}


\begin{itemize}
  \item  Le g{\^a}teau est tr{\`e}s fin. \\ The cake is top quality.
  \item  J'{\'e}cris tous les jours. \\ I write everyday.
  \item  Elle est brune. \\ She is a brunette.
  \item  Chaque jour je t'aime plus. \\ Every day I love you more.
  \item  Aucun animal ne boit. \\ No animal drinks.
\end{itemize}


\pagebreak
\subsection{Prepositions 2}

\subsubsection{Temporal Prepositions}

Choosing a preposition for time depends on the situation, but multiple choices may be appropriate.

\begin{itemize}
  \item  \textbf{Durations}  
    \begin{itemize}
      \item  Pendant and durant are interchangeable and mean ``during'' or ``for''. These are versatile and can be used for most expressions of duration.
        \begin{itemize}
          \item  Pendant l'{\'e}t{\'e}, il fait chaud. \\ During the summer, it is hot.
          \item  Je veux dormir pendant une semaine ! \\ I want to sleep for a week!
          \item  Elles peuvent rester durant un jour. \\ They can stay for a day.
          \item  Chaque matin, je cours pendant une heure. \\ Every morning, I run for an hour.
        \end{itemize}
       \item  Depuis (``since'' or ``for'') can be used for things that are still happening, and it's usually followed by a start date or a duration. It's tricky because a French present tense verb with depuis often translates to an English present perfect tense verb.
         \begin{itemize}
          \item  Il pleut depuis hier. \\ It has been raining since yesterday.
          \item  Je te connais depuis deux ans. \\ I have known you for two years.
        \end{itemize}
      \item  En (``in'') indicates the length of time an action requires for completion and can be used with any tense.
        \begin{itemize}
          \item  Je peux le finir en deux heures. \\ I can finish it in two hours.
          \item  Elle va lire le livre en une heure. \\ She is going to read the book in an hour.
        \end{itemize}
      \item  Pour (``for'') is the most limited choice and is only used with aller or partir for future events.
        \begin{itemize}
          \item  Il est en vacances pour une semaine. \\ He is on vacation for a week.
          \item  Je vais chez moi pour la nuit. \\ I am going home for the night.
        \end{itemize}
    \end{itemize}
  \item  \textbf{References}
    \begin{itemize}
      \item  Use {\`a} to pinpoint exactly what time of day an event begins or to give the endpoint of a time range in conjunction with de.
        \begin{itemize}
          \item  Le repas commence {\`a} midi. \\ The meal begins at noon.
          \item  La boutique est ouverte de 8.00 {\`a} 17.00. \\ The boutique is open from 8 to 5.
        \end{itemize}
      \item  En can also indicate that an action took place in a particular month, season, or year. The exception is spring, which requires au.
        \begin{itemize}
          \item  Je vais {\`a} Paris en avril. \\ I am going to Paris in April. 
          \item  Je commence {\`a} bronzer en douceur en {\'e}t{\'e}. \\ I begin to gently sunbathe in summer.
          \item  Il va toujours chez lui au printemps. \\ He always goes home in spring.
        \end{itemize}
      \item  Dans also means ``in'', but it gives the amount of time before an action will take place.
        \begin{itemize}
          \item  Elle va revenir dans 15 minutes. \\ She is going to return in 15 minutes.
          \item  Je vais t'appeler dans une demi-heure. \\ I'm going to call you in half an hour.
        \end{itemize}
      \item  However, to give the amount of time needed to perform an action, en will be used.
        \begin{itemize}
          \item  Je peux faire ceci en une heure. \\ I can do this in/within one hour.
          \item  Elle pouvait r{\'e}soudre ce probl{\`e}me en 10 minutes \\ She was able to solve this problem in 10 minutes.
        \end{itemize}
    \end{itemize}
\end{itemize}

\subsubsection{Puzzling Prepositions}

\begin{itemize}
  \item  Chez can be combined with a pronoun or noun to refer to someone's home or workplace.
    \begin{itemize}
      \item  Je vais chez le dentiste. \\ I am going to the dentist's.
      \item  Elle est chez Kristy. \\ She's at Kristy's house.
    \end{itemize}
  \item  Entre means ``between'', both literally and figuratively.
    \begin{itemize}
      \item  Il est entre deux foug{\`e}res. \\ He is between two ferns.
      \item  Je te le dis, mais c'est entre nous. \\ I can tell you, but it's between us.
    \end{itemize}
  \item  Parmi means ``among'' and indicates that something is part of a larger group of assorted people, animals, or things.
    \begin{itemize}
      \item  Des lions sont parmi les animaux du zoo. \\ Lions are among the zoo animals.
      \item  Le chat dort parmi les chiens. \\ The cat sleeps among the dogs.
    \end{itemize}
  \item  However, if the larger group is uniform in some specific way, entre can also mean ``among''.
    \begin{itemize}
      \item  Ici, nous sommes entre femmes. \\ Here, we are among women.
      \item  Nous pouvons parler librement entre coll{\`e}gues. \\ We can speak freely among colleagues.
    \end{itemize}
  \item  There are some situations where both entre and parmi are acceptable.
    \begin{itemize}
      \item  Il choisit entre/parmi les options. \\ He chooses between the options.
    \end{itemize}
  \item  Devant and avant both mean "before", but devant is spatial while avant is temporal.
    \begin{itemize}
      \item  Je suis devant vous. \\ I stand before you.
      \item  Il mange avant nous. \\ He eats before us.
    \end{itemize}
\end{itemize}

\subsubsection{Peu}

Using the word peu (``few''/``little'') can be surprisingly complicated. By itself, peu is usually an adverb that diminishes what it modifies and is generally translated using ``not very/much/well''.

\begin{itemize}
  \item  Elle parle peu. \\ She doesn't talk much.
  \item  Il est peu probable. \\ It is not very likely.
  \item  Je vous connais peu. \\ I don't know you well.
  \item  Ce ph{\`e}nom{\`e}ne est peu fr{\`e}quent. \\ This phenomenon is infrequent.
  \item  peu apr{\`e}s \\ not long after
\end{itemize}

Appending de creates an adverb of quantity that modifies nouns.

\begin{itemize}
  \item  Peu de femmes disent {\c c}a. \\ Few women say that.
  \item  Peu d'eau sur la Terre est potable. \\ Little of the water on Earth is drinkable.
\end{itemize}

However, peu can also be a noun, especially when preceded by an article.

\begin{itemize}
  \item  Elle parle un peu de français. \\ She speaks a bit of French.
  \item  Tu veux manger un peu de fraises ? \\ Do you want to eat a few strawberries?
  \item  Oui, j'en veux un peu. \\ Yes, I want a few. (Or ``a little''.)
\end{itemize}

\begin{center}\begin{tabular}{l|l||l|l||l|l}
\textbf{Fran{\c c}ais} & \textbf{English} & \textbf{Fran{\c c}ais} & \textbf{English}  & \textbf{Fran{\c c}ais} & \textbf{English} \\ \hline
sur & on & contre & against & selon & according \\
sans & without & chez & at [some] place & parmi & among \\ 
entre & among & depuis & since & sauf & except \\
sous & under & pendant & for (duration) & durant & during \\ 
vers & toward & derri{\`e}re & behind & malgr{\'e} & despite \\ 
\end{tabular}\end{center}


\pagebreak
\subsection{Places}

Expressing locations in French can be tricky because many English prepositions don't have one-to-one French translations. This is especially true for ``in'', which can be dans, en, or {\`a} depending on how specific the location is.

\begin{itemize}
  \item  Dans means ``in'' for specific, known locations. It is especially appropriate when the location name has an article or possessive.
	  \begin{itemize}
		  \item  Il mange dans le restaurant. \\ He's eating in the restaurant.
			\item  Un chat est dans ma chambre. \\ A cat is in my room.
		\end{itemize}
	\item  Use {\`a} and its contractions for unspecific or vague locations.
	  \begin{itemize}
		  \item  On vit {\`a} la campagne. \\ We live in the country.
			\item  C'est dangereux {\`a} la fronti{\`e}re. \\ It's dangerous at the frontier.
		\end{itemize}
	\item  When describing a location that doesn't require a determiner (usually a type of place), use en.
	  \begin{itemize}
		  \item  Nous sommes en classe. \\ We are in class.
			\item  Elle est en prison. \\ She is in prison.
		\end{itemize}
\end{itemize}

\subsubsection{Special Rules}

For all cities (and islands), use {\`a} for ``to'' or ``in'' and de for ``from''.

\begin{itemize}
  \item  Le roi vit {\`a} Versailles. \\ The king lives in Versailles.
  \item  Nous allons {\`a} Paris. \\ We are going to Paris.
  \item  Napol{\'e}on vient de Corse. \\ Napoleon comes from Corsica.
  \item  Je l'envoie d'Orl{\'e}ans. \\ I am sending it from Orleans.
\end{itemize}

Countries, province, states and continents have gender-based rules. For feminine ones, en means ``to'' or ``in'' and de means ``from''. Luckily, all continents are feminine, as are most countries ending in -e.

\begin{itemize}
  \item  Bordeaux est en France. \\ Bordeaux is in France.
  \item  Il reste en Europe. \\ He is staying in Europe.
  \item  On vient de Californie. \\ We come from California.
  \item  Elle part d'Asie. \\ She is departing from Asia.
\end{itemize}

For masculine countries, provinces and states that start with a consonant sound, use au and du.

\begin{itemize}
  \item  Je veux aller au Qu{\'e}bec. \\ I want to go to Quebec.
  \item  Elles partent du Japon. \\ They are departing from Japan.
\end{itemize}

If they start with a vowel sound, switch back to en and de for euphony.

\begin{itemize}
  \item  Il y a une guerre en Irak. \\ There is a war in Iraq.
  \item  J'arrive d'Ontario. \\ I am coming from Ontario.
\end{itemize}

For countries with pluralized names (the USA, the Netherlands, the Philippines, etc.), use aux and des.

\begin{itemize}
  \item  On travaille aux {\'E}tats-Unis. \\ We work in the United States.
\end{itemize}

\subsubsection{Using the Present for the Future}

In both French and English, the present tense can often be used to express the near future (le futur proche). In French, this usage is basically equivalent to aller + infinitive.

\begin{itemize}
  \item  Je vais {\`a} Paris demain. \\ I am going to Paris tomorrow.
  \item  Demain, c'est samedi. \\ It's Saturday tomorrow.
  \item  La f{\^e}te commence demain. \\ The party begins tomorrow.
\end{itemize}

\begin{center}\begin{tabular}{l|l||l|l||l|l}
\textbf{Fran{\c c}ais} & \textbf{English} & \textbf{Fran{\c c}ais} & \textbf{English}  & \textbf{Fran{\c c}ais} & \textbf{English} \\ \hline
\Red{la place} & place & \Blue{le salon} & living room & \Blue{le b{\^a}timent} & building \\
\Red{la pays} & country & \Blue{le territoire} & territory & \Red{la Bordeaux} & Bordeaux \\
\Red{la terre} & soil & \Blue{le trou} & hole & \Blue{le restaurant} & restaurant \\
\Red{la rue} & street & \Blue{le ch{\^a}teau} & castle & \Blue{le pont} & bridge \\
\Blue{le lieu} & location & \Blue{le {\'e}tage} & floor & \Red{la banque} & bank \\
\Red{la chambre} & bedroom & \Blue{le appartement} & apartment & \Red{la {\'E}tats-Unis} & United States \\
\Red{la ville} & town & \Red{la adresse} & address & \Blue{le d{\'e}sert} & desert \\
\Blue{le tour} & tour & \Red{la plage} & beach & \Red{la Paris} & Paris \\
\Blue{le milieux} & middle & \Red{la prison} & prison & \Red{l'Angleterre} & England \\
\Red{la route} & road & \Blue{le bar} & bar (for drinks) & \Red{l'Asie} & Asia \\
{\`a} bord & on board & \Blue{le magasin} & store & \Red{la boulangerie} & bakery \\
\Blue{le coin} & corner & \Blue{le parc} & park & \Red{l'Espagne} & Spain \\
\Blue{le jardin} & garden & \Red{la cave} & cellar & \Red{l'Afrique} & Africa \\
\Blue{le garage} & garage & \Blue{le bain} & bath & \Red{l'Italie} & Italy \\
\Blue{l'h{\^o}tel} & hotel & \Red{la boutique} & shop & \Blue{le zoo} & zoo \\
\Red{la r{\'e}gion} & region & \Red{la fronti{\`e}re} & frontier & \Red{la campange} & countryside \\
\Red{la entreprise} & company & \Blue{les toilettes} & bathroom & \Red{la Br{\'e}sil} & Brazil \\
\Red{la c{\'o}te} & coast & \Red{la salle} & room & \Red{la Chine} & China \\
\Red{la ile} & island & \Blue{le tribunal} & court & \Blue{le immeuble} & block \\
\Blue{le port} & port & \Red{la Am{\'e}rique} & America & \Red{la discoth{\`e}que} & night club \\
\Blue{le village} & village & \Blue{le banlieue} & suburb & \Blue{le supermarch{\'e}} & supermarket \\
\Blue{le quartier} & district & \Red{la Europe} & Europe & \Blue{le couloir} & corridor \\
\Red{la propri{\'e}t{\'e}} & property & \Blue{le France} & France & visiter & to visit \\
\Red{la zone} & area & \Blue{le Allemange} & Germany \\
\end{tabular}\end{center}


\pagebreak
\subsection{Irregular Plurals}

Most French nouns and adjectives can be pluralized by adding an ending -s, like in English. Those that can't be pluralized like this normally will have plural forms that end in -x. For instance, most nouns ending in -al or -ail change to -aux.

\begin{itemize}
  \item  un animal $\rightarrow$ des animaux (``animals'')
  \item  le travail $\rightarrow$ les travaux (``work'')
\end{itemize}

Similarly, masculine singular adjectives ending in -al take on -aux endings in the plural. However, feminine singular adjectives ending in -ale simply add an ending -s.

\begin{itemize}
  \item  g{\'e}n{\'e}ral $\rightarrow$ g{\'e}n{\'e}raux (``general'')
  \item  g{\'e}n{\'e}rale $\rightarrow$ g{\'e}n{\'e}rales (``general'')
	\item  id{\'e}al $\rightarrow$ id{\'e}aux (``ideal'')
  \item  id{\'e}ale $\rightarrow$ id{\'e}ales (``ideal'')
\end{itemize}

Add -x to the end of most nouns that end in -au, -eau, and -eu to pluralize them.

\begin{itemize}
  \item  un tuyau $\rightarrow$ des tuyaux (``pipes'')
	\item  mon chapeau $\rightarrow$ mes chapeaux (``my hats'')
  \item  le feu $\rightarrow$ les feux (``fires'')
\end{itemize}

The plural forms of -au, -eau, and -eu words are homophones of their singular forms. In general, the best way to tell if a noun is plural is to listen carefully to its article or determiner. If you hear les or des, or the possessives mes, tes, ses, nos, vos, leurs, or the demonstrative ces, it's plural. Otherwise, it's probably singular.

\begin{itemize}
  \item  Ce sont vos animaux. \\ Those are your animals.
  \item  Ce sont des chevaux. \\ Those are horses.
	\item  Ses yeux sont bleus. \\ Her eyes are blue.
	\item  Les hommes aiment les jeux. \\ The men like the games.
	\item  Nous aimons lire les journaux. \\ We like reading the newspapers.
	\item  Ce b{\'e}b{\'e} a les cheveux noirs. \\ This baby has black hair.
	\item  Les animaux doivent boire. \\ Animals have to drink.
	\item  Les oiseaux mangent du riz. \\ The mice eat rice.
	\item  Leurs g{\^a}teaux sont bons. \\ Their cakes are nice.
	\item  Je suis {\`a} genoux. \\ I am on my knees.
	\item  Elle a quinze bijoux. \\ She has 15 jewels.
	\item  Je mange des morceaux de g{\^a}teau. \\ I eat pieces of cake.
	\item  Ils sont tr{\`e}s vieux. \\ They are very old.
\end{itemize}


\pagebreak
\subsection{People}

French nouns for persons of a certain nationality are capitalized, but in French, national adjectives and language names are not capitalized.

\begin{itemize}
  \item  C'est une Anglaise. \\ She's an Englishwoman.
	\item  C'est une voiture anglaise. \\ It's an English car.
	\item  Ce sont des Fran{\c c}aises. \\ They are Frenchwomen.
	\item  Elles parlent fran{\c c}ais. \\ They speak French.
\end{itemize}

Remember that nouns for nationalities (and also professions and religions) can appear after {\^e}tre without a determiner. In this usage, they are adjectives and are not capitalized.

\begin{itemize}
  \item  Je suis chinois. \\ I am Chinese. 
	\item  Mon oncle est italien. \\ My uncle is Italian.
\end{itemize}

\begin{itemize}
  \item  J'ai un \Blue{groupe} d'amis. \\ I have a group of friends.
	\item  Tu es mon \Blue{peuple}. \\ You are my people.
	\item  Oui, nous avons des amis. \\ Yes, we have friends.
	\item  Ce sont des \Blue{gens}. \\ They are people.
	\item  C'est ma \Red{culture}. \\ It is my culture.
	\item  J'ai quelques amis. \\ I have a few friends.
	\item  Je suis dans la \Red{foule}. \\ I am in the crowd.
	\item  Le lion est parmi la foule. \\ The lion is among the crowd.
	\item  C'est une \Red{personne} sale. \\ It is a dirty person.
	\item  La \Red{dame} est parfaite. \\ The lady is perfect.
	\item  Mes voisons sont espagnols. \\ My neighbors are Spanish.
	\item  Ce sont des h{\'e}ros. \\ They are heroes.
	\item  Elle n'a pas d'ennemis. \\ She does not have enemies.
	\item  Les habitants sont riches. \\ The inhabitants are rich.
	\item  Je suis un homme et un citoyen. \\ I am a man and a citizen.
	\item  Nous avons une \Red{association}. \\ We have an association.
	\item  C'est mon coll{\`e}gue. \\ It is my colleague.
	\item  Son copain est chauffeur. \\ His buddy is a driver.
	\item  Ma \Red{g{\'e}n{\'e}ration} ne mange pas de poisson. \\ My generation does not eat fish.
	\item  Mon p{\`e}re est un adulte. \\ My father is an adult.
	\item  Il est populaire parmi ses coll{\`e}gues. \\ He is popular among his colleagues.
	\item  Un adolescent mange beaucoup. \\ A teenager eats a lot.
	\item  Ce sont vos prisonniers. \\ They are your prisoners.
	\item  J'aime ma copine. \\ I love my girlfriend.
	\item  Tu es mon invit{\'e}. \\ You are my guest.
	\item  J'aime tout le monde ici. \\ I like everybody here.
	\item  Les vieux sont int{\'e}ressants. \\ The elderly are interesting.
	\item  J'attends un client aujourd'hui. \\ I am waiting for a client today.
	\item  Elle a un petit ami. \\ She has a boyfriend.
	\item  Ce magasin a beaucoup de clients. \\ That store has a lot of customers.
	\item  Il parle le russe. \\ He speaks Russian.
	\item  Le mari est aussi un Russe. \\ The husband is also Russian.
	\item  C'est un Allemand. \\ He is a German man.
\end{itemize}






















