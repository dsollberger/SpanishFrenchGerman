\subsection{To Be and To Have}

{\^E}tre and avoir are the most common verbs in French. Like many common verbs, they have irregular conjugations.

\begin{center}\begin{tabular}{|r|c|l|}
\hline
\textbf{Subject}    & \textbf{{\^E}tre ("to be")} & \textbf{Avoir ("to have")} \\ \hline
\textbf{je}         & (je)suis                & (j')ai                     \\ \hline
\textbf{tu}         & es                      & as                         \\ \hline
\textbf{il/elle/on} & est                     & a                          \\ \hline
\textbf{nous}       & sommes                  & avons                      \\ \hline
\textbf{vous}       & {\^e}tes                & avez                       \\ \hline
\textbf{ils/elles}  & sont                    & ont                        \\ \hline
\end{tabular}\end{center}

There should be a liaison between ils or elles and ont ("il-zon" or "elle-zon"). The "z" sound is essential here to differentiate between "they are" and "they have", so be sure to emphasize it.  These two verbs are very important because they can act as auxiliary verbs in French, but they differ from their English equivalents. In "Basics 2", you learned that "I write" and "I am writing" both translate to j'{\'e}cris, not je suis {\'e}cris. This is because {\^e}tre cannot be used as an auxiliary in a simple tense. It can only be used in compound tenses, which you will learn in the "Pass{\'e} Compos{\'e}" unit.  Another important distinction is that avoir means "to have" in the sense of "to possess", but not "to consume" or "to experience". Other verbs must be used for these meanings.

\subsubsection{C'est or Il Est?}

When describing people and things with {\^e}tre in French, you usually can't use a personal subject pronoun like elle. Instead, you must use the impersonal pronoun ce, which can also mean "this" or "that". Note that ce is invariable, so it can never be ces sont.

\begin{center}\begin{tabular}{|c|c|c|}
\hline
                  & \textbf{Impersonal Subject Pronoun} & \textbf{Personal Subject Pronoun} \\ \hline
\textbf{Singular} & c'est                               & il/elle est                       \\ \hline
\textbf{Plural}   & ce sont                             & ils/elles sont                    \\ \hline
\end{tabular}\end{center}

These pronouns aren't interchangeable. The basic rule is that you must use ce when {\^e}tre is followed by any determiner---for instance, an article or a possessive adjective. Note that c'est should be used for singulars and ce sont should be used for plurals.

\begin{itemize}
  \item  C'est un homme.---He's a man. / This is a man. / That is a man.
  \item  Ce sont des chats.---They're cats. / These are cats. / Those are cats.
  \item  C'est mon chien.---It's my dog. / This is my dog. / That's my dog.
\end{itemize}

If an adjective, adverb, or both appear after {\^e}tre, then use the personal pronoun.

\begin{itemize}
  \item  Elle est belle.---She is beautiful. (Or "It is beautiful.")
  \item  Il est tr{\`e}s fort.---He is very strong. (Or "It is very strong.")
\end{itemize}

As you know, nouns generally need determiners, but one important exception is that professions, nationalities, and religions can act as adjectives after {\^e}tre. \\ 
This is optional; you can also choose to treat them as nouns.

\begin{itemize}
  \item  He is a doctor.---Il est m{\'e}decin. / C'est un m{\'e}decin.
\end{itemize}

However, c'est should be used when using an adjective to make a general comment about (but not describe) a thing or situation. In this case, use the masculine singular form of the adjective.

\begin{itemize}
  \item  C'est normal ?---Is this normal?
  \item  Non, c'est {\'e}trange.---No, this is strange.
\end{itemize}

\subsubsection{Idioms with Avoir}

One of the most common idioms in French is the use of the verb avoir in certain places where English would use the verb "to be". This is especially common for states or conditions that a person may experience.

\begin{itemize}
  \item  Elle a chaud.---She is hot. (Or "She feels hot.")
  \item  Il a froid.---He is cold.
  \item  Elle a deux ans.---She is two years old.
  \item  J'ai peur !---I am afraid!
\end{itemize}

French tends to use the verb faire ("to do") idiomatically for general conditions like weather.\footnote{Note that il fait is an impersonal expression with no real subject, just like il y a from "Common Phrases".}

\begin{itemize}
  \item  Il fait chaud.---It is hot (outside).
  \item  Il fait froid.---It is cold.
  \item  Il fait nuit.---It is nighttime.
\end{itemize}

\begin{itemize}
  \item  C'est de la soupe. \\ This is soup
  \item  C'est la soupe. \\ This is the soup
  \item  C'est du vin. \\ That is wine.
\end{itemize}


\pagebreak
\subsection{Clothing}

\subsubsection{Idiomatic Plurals}

English has a number of idiomatic plural-only nouns that have to be translated carefully. These are not just nouns that are invariable with number (like "deer"), but rather nouns that cannot refer to a singular thing at all.  For instance, "the pants" can only be plural in English, but the corresponding le pantalon is singular in French. A single pair of pants is not les pantalons, which refers to multiple pairs of pants. Similarly, when translating le pantalon back to English, you can say "the pants" or "a pair of pants", but "a pant" is not correct. This also applies to un jean ("a pair of jeans").  Un v{\^e}tement refers to a single article of clothing, and it's incorrect to translate it as "clothes", which is plural and refers to a collection of clothing. This would have to be des v{\^e}tements.

\subsubsection{Diacritics}

The \textbf{acute accent} ({\'e}) only appears on E and produces a pure [e] that isn't found in English. To make this sound, say the word ``clich{\'e}'', but hold your tongue perfectly still on the last vowel to avoid making a diphthong sound.

The \textbf{grave accent} ({\`e}) can appear on A/E/U, though it only changes the sound for E (to [ɛ], which is the E in "lemon"). Otherwise, it distinguishes homophones like a (a conjugated form of avoir) and {\`a} (a preposition).

The \textbf{cedilla} ({\c c}) softens a normally hard C sound to the soft C in ``cent''. Otherwise, a C followed by an A, O, or U has a hard sound like the C in ``car''.

The \textbf{circumflex} ({\^e}) usually means that an S used to follow the vowel in Old French or Latin. (The same is true of the acute accent.) For instance, {\^i}le was once ``isle''.

The \textbf{trema} ({\"e}) indicates that two adjacent vowels must be pronounced separately, like in No{\"e}l ("Christmas") and ma{\"i}s ("corn").

\subsubsection{Nasal Vowels}

There are four nasal vowels in French. Try to learn these sounds by listening to native speakers.

\begin{center}\begin{tabular}{|c|c|c|}
\hline
\textbf{IPA} & \textbf{Letter Sequence} & \textbf{Examples}       \\ \hline
/œ̃/         & un/um                    & un/parfum               \\ \hline
/ɛ̃/         & in/im/yn/ym              & vin/pain/syndicat/sympa \\ \hline
/ɑ̃/         & an/am/en/em              & dans/chambre/en/emploi  \\ \hline
/ɔ̃/         & on/om                    & mon/ombre               \\ \hline
\end{tabular}\end{center}

These aren't always nasalized. If there's a double M or N, or if they are followed by any vowel, then the vowel should have an oral sound instead. For instance, un is nasal, but une is not. Also, vin is nasal, but vinaigre is not.

\subsubsection{Vocabulary}

\begin{center}\begin{tabular}{l|l||l|l}
\textbf{French} & \textbf{English} & \textbf{French} & \textbf{English} \\ \hline
\Blue{le sac} & bag & \Blue{le manteau} & coat  \\ 
\Red{la poche} & pocket & \Red{la vest} & jacket  \\ 
\Red{la ceinture} & belt & \Blue{le chapeau} & hat  \\ 
\Blue{le v{\^e}tement, les v{\^e}tements} & clothes & \Red{la chaussette} & sock  \\ 
\Blue{le pantalon, les pantalons} & pair of pants, pants & \Blue{le jean, les jeans} & pair of jeans, jeans \\
\Red{la chaussure} & shoe & \Blue{le parapluie} & umbrella  \\ 
\Red{la chemise} & shirt & \Blue{le gant} & glove  \\ 
\Red{la jupe} & skirt & \Red{la casquette} & cap  \\ 
\Blue{le botte} & boot  & \Red{l'{\'e}charpe} & scarf \\ 
\Blue{le costume} & suit & \Red{la cravate} & necktie \\ 
\Blue{le portefeuille} & wallet \\
\end{tabular}\end{center}


\pagebreak
\subsection{Colors}

Colors can be both nouns and adjectives. As nouns, colors are usually masculine (e.g. \guillemotleft~ Le rose \guillemotright~ = ``The pink'').

As adjectives, they agree with the nouns they modify except in two cases. First, colors derived from nouns (e.g. fruits, flowers, or gems) tend to be invariable with gender and number. Orange ("orange") and marron ("brown") are the most common examples.

\begin{itemize}
  \item  La jupe orange \\ The orange skirt
  \item  Les jupes orange \\ The orange skirts
  \item  Les chiens marron \\ The brown dogs
\end{itemize}

Second, in compound adjectives (les adjectifs composés) made up of two adjectives, both adjectives remain in their masculine singular forms.

\begin{itemize}
  \item  Sa couleur est vert pomme. \\ Its color is apple-green.
  \item  J'aime les robes rose clair. \\ I like light-pink dresses.
\end{itemize}

Most colors that end in -e in their masculine forms are invariable with gender.

\begin{itemize}
  \item  Un chien rouge \\ A red dog
  \item  Une jupe rouge \\ A red skirt
\end{itemize}

\huge

%http://www.rapidtables.com/web/color/RGB_Color.htm

\begin{center}\begin{tabular}{l|l||l|l}
\textbf{French} & \textbf{English} & \textbf{French} & \textbf{English} \\ \hline
\Blue{le couleur} & color & \textcolor[RGB]{255,0,0}{rouge} & \textcolor[RGB]{255,0,0}{red} \\
blanc & white &  \textcolor[RGB]{255,165,0}{orange} & \textcolor[RGB]{255,165,0}{orange} \\ 
noir & black & \textcolor[RGB]{255,255,0}{jaune} & \textcolor[RGB]{255,255,0}{yellow} \\ 
\textcolor[RGB]{128,128,128}{gris} & \textcolor[RGB]{128,128,128}{gray} & \textcolor[RGB]{0,128,0}{verte} & \textcolor[RGB]{0,128,0}{green} \\ 
\textcolor[RGB]{255,192,203}{rose} & \textcolor[RGB]{255,192,203}{pink} & \textcolor[RGB]{0,0,255}{bleu} & \textcolor[RGB]{0,0,255}{blue} \\ 
\textcolor[RGB]{165,42,42}{marron} & \textcolor[RGB]{165,42,42}{brown} & \textcolor[RGB]{128,0,128}{violet} & \textcolor[RGB]{128,0,128}{purple} \\ 
\end{tabular}\end{center}

\normalsize

\begin{itemize}
  \item  La fille a une jolie robe rose. \\ The girl has a pretty pink dress.
\end{itemize}


\pagebreak
\subsection{Possessives}

In English, possessive adjectives (e.g. ``his'') match the owner. However, in French, they match the thing being owned.  Consider the example of ``her lion''. The French translation is \guillemotleft~ son lion \guillemotright , because lion is masculine and both the lion and the woman are singular. Note that if we hear just \guillemotleft~ son lion \guillemotright , we can't tell if the lion is owned by a man or woman. It's ambiguous without more context. If two people own a lion, then it is  \guillemotleft~ leur lion \guillemotright .

Possessives have different forms that agree with four things: the number of owners, the number of things owned, the gender of the thing owned, and the grammatical person of the owner (e.g. "his" versus "my").  For one owner, the possessive adjectives are:

\begin{center}\begin{tabular}{|c|c|c|c|c|}
\hline
\textbf{Person} & \textbf{English} & \textbf{Masculine Singular} & \textbf{Feminine Singular} & \textbf{Plural} \\ \hline
\textbf{1st}    & my               & mon                         & ma                         & mes             \\ \hline
\textbf{2nd}    & your (singular)  & ton                         & ta                         & tes             \\ \hline
\textbf{3rd}    & his/her/its      & son                         & sa                         & ses             \\ \hline
\end{tabular}\end{center}

For multiple owners, genders don't matter:

\begin{center}\begin{tabular}{|c|c|c|c|}
\hline
\textbf{Person} & \textbf{English} & \textbf{Singular Owned} & \textbf{Plural Owned} \\ \hline
\textbf{1st}    & our              & notre                   & nos                   \\ \hline
\textbf{2nd}    & your (plural)    & votre                   & vos                   \\ \hline
\textbf{3rd}    & their            & leur                    & leurs                 \\ \hline
\end{tabular}\end{center}

The plural second-person possessive adjectives, votre and vos, should be used when addressing someone formally with vous.

\begin{center}\begin{tabular}{|c|l|l|}
\hline
\textbf{Owner}   & \multicolumn{1}{c|}{\textbf{Singular Owned}} & \multicolumn{1}{c|}{\textbf{Plural}} \\ \hline
\textbf{My}      & Mon ami---My friend                          & Mes tigres---My tigers               \\ \hline
\textbf{Your}    & Ton abeille---Your bee                       & Tes lions---Your lions               \\ \hline
\textbf{His/Her} & Son oiseau---His/her bird                    & Ses chiens---His/her dogs            \\ \hline
\textbf{Our}     & Notre bi{\`e}re---Our beer                   & Nos pommes---Our apples              \\ \hline
\textbf{Your}    & Votre sel---Your salt                        & Vos citrons---Your lemons            \\ \hline
\textbf{Their}   & Leur fromage---Their cheese                  & Leurs fromages---Their cheeses       \\ \hline
\end{tabular}\end{center}

\subsubsection{Euphony in Possessives}

For the sake of euphony, all singular feminine possessives switch to their masculine forms when followed by a vowel sound.

\begin{center}\begin{tabular}{|c|c|c|c|}
\hline
\textbf{Person} & \textbf{Masculine} & \textbf{Feminine} & \textbf{Feminine + Vowel Sound} \\ \hline
1st             & mon chat           & ma robe           & mon eau                         \\ \hline
2nd             & ton chat           & ta robe           & ton eau                         \\ \hline
3rd             & son chat           & sa robe           & son eau                         \\ \hline
\end{tabular}\end{center}

\subsubsection{Femme and Fille}

Femme can mean "woman" or "wife" and fille can mean "girl" or "daughter" depending on the context. For example, when femme and fille are preceded by a possessive adjective, then they translate to "wife" and "daughter", respectively.

\begin{itemize}
  \item  Une fille et une femme sont dans le restaurant \\ A girl and a woman are in the restaurant. (Not: "A daughter and a wife are in the restaurant.")
  \item  Ma fille \\ My daughter. (Not: "My girl".)
  \item  Ta femme \\ Your wife. (Not: "Your woman".)
\end{itemize}


\pagebreak
\subsection{Present Tense 1}

As you learned in ``Basics 1'', verbs like parler conjugate to agree with their subjects. Parler itself is an infinitive, which is a verb's base form. It consists of a root (parl-) and an ending (-er). The ending can dictate how the verb should be conjugated. In this case, almost all verbs ending in -er are regular verbs in the 1st Group that share the same conjugation pattern. To conjugate another 1st Group verb, affix the ending to that verb's root.

\begin{itemize}
  \item  Aimer ("to love"): j'aime, tu aimes, nous aimons, etc.
  \item  Marcher ("to walk"): je marche, tu marches, nous marchons, etc. 
\end{itemize}

Every verb belongs to one of three groups:

\begin{itemize}
  \item  The \textbf{1st Group} includes regular -er verbs and includes 80\% of all verbs.\footnote{Aller (``to go'') is the only fully irregular verb in Group 1, but a handful of others are slightly irregular.}
  \item  The \textbf{2nd Group} includes regular -ir verbs like finir ('to finish").
  \item  The \textbf{3rd Group} includes all irregular verbs. This includes many common verbs like {\^e}tre and avoir as well as a handful of less common conjugation patterns.
\end{itemize}

\begin{center}\begin{tabular}{|r|l|l|l|}
\hline
\textbf{Subject} & \textbf{G1: parler} & \textbf{G2: finir} & \textbf{G3: dormir} \\ \hline
je               & parl\Blue{e}               & fini\Red{s}              & dor\textbf{s}                \\ \hline
tu               & parl\Blue{es}              & fini\Red{s}              & dor\textbf{s}                \\ \hline
il/elle/on       & parl\Blue{e}               & fini\Red{t}              & dor\textbf{t}                \\ \hline
nous             & parl\Blue{ons}             & fini\Red{ssons}          & dor\textbf{mons}             \\ \hline
vous             & parl\Blue{ez}              & fini\Red{ssez}           & dor\textbf{mez}              \\ \hline
ils/elles        & parl\Blue{ent}             & fini\Red{ssent}          & dor\textbf{ment}             \\ \hline
\end{tabular}\end{center}

\textbf{Spelling-changing} verbs end in -ger (e.g. manger) or -cer (e.g. lancer, ``to throw'') and change slightly in the nous form, as well as any other form whose ending begins with an A or O. These verbs take a form like nous mangeons or nous lan{\c c}ons.

\textbf{Stem-changing} verbs have different roots in their nous and vous forms. For instance, most forms of appeler (``to call'') have two L's (e.g. j'appelle), but the N/V forms are nous appelons and vous appelez. 

\subsubsection{Semi-Auxiliary Verbs}

The only true auxiliary verbs in French are {\^e}tre and avoir, but there are a number of semi-auxiliary verbs in French that can be used with other verbs to express ability, necessity, desire, and so on. They are used in \textbf{double-verb} constructions where the first verb (the semi-auxiliary) is conjugated and the second is not.

\begin{itemize}
  \item  Je veux lire. \\ I want to read.
  \item  Il aime manger. \\ He likes to eat.
\end{itemize}

Modal verbs are the English equivalents of semi-auxiliaries.  For instance, ``can'', translates to either savoir or pouvoir. When ``can'' indicates knowledge, use savoir.  When ``can'' indicates permission or ability (apart from knowledge), use pouvoir.

\begin{itemize}
  \item  Je sais lire et {\'e}crire. \\ I know how to read and write.
  \item  Il sait parler allemand. \\ He knows how to speak German.
  \item  Il peut manger. \\ He can (or "may") eat.
  \item  Il peut parler allemand. \\ He is allowed to speak German.
\end{itemize}

One of the most important semi-auxiliary verbs is aller, which is used to express the near future (futur proche), just like the English verb ``going to''. 

\begin{itemize}
  \item  Je vais manger. \\ I am going to eat.
  \item  Vous allez lire le livre. \\ You are going to read the book.
\end{itemize}

Note that in verb constructions beginning with non-auxiliary verbs, the verbs must be separated by a preposition.

\begin{itemize}
  \item  Nous vivons pour manger. \\ We live to eat.
\end{itemize}

\subsubsection{Impersonal Expressions}

A few defective impersonal verbs can only be used in impersonal statements and must be conjugated as third-person singular with il. Remember that il is a dummy subject and does not refer to a person.  Falloir means ``to be necessary'', and it often takes the form il faut + infinitive.  Il faut can also be used transitively with a noun to indicate that it is needed.

\begin{itemize}
  \item  Il faut manger. \\ It is necessary to eat. / One must eat.
  \item  Il faut choisir. \\ It is necessary to choose. / One must choose.
  \item  Il faut du pain. \\ (Some) bread is needed.
\end{itemize}

\subsubsection{Confusing Verbs}

Used transitively, savoir and connaître both mean ``to know'', but in different ways. Savoir implies understanding of subjects, things, or skills, while connaître indicates familiarity with people, animals, places, things, or situations.

\begin{itemize}
  \item  Je sais les mots. \\ I know the words.
  \item  Je connais le garçon. \\ I know the boy.
\end{itemize}

Attendre means ``to await'', which is why it does not need a preposition.

\begin{itemize}
  \item  Il attend son ami. \\ He is awaiting (or "waiting for") his frien
\end{itemize}

\subsubsection{One Each}

The indefinite article doesn't always refer to just one thing. Sometimes, it can mean one thing each. Consider these examples:

\begin{itemize}
  \item  Ils ont un manteau \\ They have one coat / They each have one coat
  \item  Ils ont des manteaux \\ They have some coats / They each have some coats
\end{itemize}

\subsubsection{Vocabulary}

\begin{center}\begin{tabular}{l|l||l|l}
\textbf{French} & \textbf{English} & \textbf{French} & \textbf{English} \\ \hline
lire & to read & faire & to make \\ 
{\'e}crivent & to write & vais & going to \\ 
parler & to talk & acheter & to buy \\ 
aimer & to like & aider & to help \\ 
vouloir & to want & appeller & to call \\ 
boivent & to drink & apporter & to bring \\
comprender & to understand & apprender & to learn \\ 
pouvoir & to be able to & attender & to wait for \\ 
doivent & must & chercher & to look for \\ 
adorer & to love &  savent & to know \\
faut & need to & conna\^{\i}re & to know (people) \\
commander & to be in charge \\
\end{tabular}\end{center}


\pagebreak
\subsubsection{Demonstratives 1}

Demonstrative adjectives (this, that, these, and those) modify nouns so they refer to something or someone specific. They can be used in place of articles. Like other adjectives, they must agree with the nouns they modify.

\begin{center}\begin{tabular}{|c|c|c|}
\hline
\textbf{Gender}    & \textbf{Singular} & \textbf{Plural} \\ \hline
\textbf{Masculine} & ce/cet            & ces             \\ \hline
\textbf{Feminine}  & cette             & ces             \\ \hline
\end{tabular}\end{center}

The singular masculine ce becomes cet in front of a vowel sound for euphony.

\begin{itemize}
  \item  Ce livre est rouge. \\ That book is red.
  \item  Cet arbre est grand. \\ That tree is big.
  \item  Cette pomme est rouge. \\ That apple is red.
  \item  Ces livres et ces pommes sont rouges. \\ Those books and those apples are red.
\end{itemize}

Ce can mean either ``this'' or ``that''. It's ambiguous between the two. To specify, use the suffix -ci (``here'') or -l{\`a} (``there'') on the modified noun.

\begin{itemize}
  \item  Ce livre-ci est rouge. \\ This book is red.
  \item  Ces chats-l{\`a} sont noirs. \\ Those cats are black. 
\end{itemize}

French learners often confuse the demonstrative adjective ce with the pronoun ce (from ``{\^E}tre-Avoir''). Discerning between them is easy, however: an adjective must modify a noun, while a pronoun can stand alone as a subject or object. Compare:

\begin{itemize}
  \item  Adjective: Ces hommes sont mes amis. \\ These men are my friends.
  \item  Pronoun: Ce sont mes amis. \\ They are my friends. 
\end{itemize}

In the first example, ces is an adjective that modifies hommes, but in the second, ce is a subject pronoun.

\subsubsection{{\c C}a}

The indefinite demonstrative pronoun {\c c}a refers to an unnamed concept or thing. When it's used as an object, it usually translates to ``this'' or ``that''.

\begin{itemize}
  \item  Tu manges {\c c}a. \\ You are eating this.
  \item  Je veux {\c c}a. \\ I want that.
\end{itemize}

{\c C}a can also be used as a subject, in which case it can also mean ``it''.

\begin{itemize}
  \item  {\c C}a sent bon. \\ It smells good.
  \item  {\c C}a semble simple. \\ This seems simple.
\end{itemize}

\subsubsection{{\c C}a or Ce?}

A simple rule of thumb to follow is that ce should be used with {\^e}tre, including in the double-verb constructions pouvoir {\^e}tre and devoir {\^e}tre.

\begin{itemize}
  \item  C’est un très bon vin ! \\ This is a really good wine!
  \item  Ce sont des garçons. \\ They are boys.
  \item  Ce peut être triste en hiver. \\ It can be sad in winter.
  \item  Ce doit être ton fils. \\ It must be your son.
\end{itemize}

{\c C}a should be used with all other verbs.

\begin{itemize}
  \item  {\c C}a va bien. \\ It's going well.
  \item  {\c C}a dure un jour. \\ That lasts a day.
  \item  {\c C}a m'intéresse beaucoup. \\ That interests me a lot.
\end{itemize}

However, when an object pronoun comes before {\^e}tre, then you must use {\c c}a, not ce. This is relatively rare.

\begin{itemize}
  \item  {\c C}a m'est {\'e}gal. \\ It's all the same to me.
\end{itemize}

Also, note that {\c c}a is informal and is usually replaced by cela (``that'') or ceci (``this'') in writing.


\pagebreak
\subsection{Conjunctions 1}

Conjunctions function by hooking up words, phrases, and clauses. This unit focuses on coordinating conjunctions, which link two or more similar elements in a sentence. For instance, \guillemotleft et \guillemotright may be used to link two nouns together.

\begin{itemize}
  \item  Je mange une pomme et une orange. \\ I am eating an apple and an orange.
  \item  Elle a un chien et un chat. \\ She has a dog and a cat.
\end{itemize}

It may also link two adjectives or even two clauses.

\begin{itemize}
  \item  La robe est grande et jolie. \\ The dress is big and pretty.
  \item  Le chat est noir et le chien est blanc. \\ The cat is black and the dog is white.
\end{itemize}

For the most part, French coordinating conjunctions behave very similarly to their English counterparts.

\begin{center}\begin{tabular}{|c|c|c|c|}
\hline
\textbf{Conjunction} & \textbf{English} & \textbf{Exemple} & \textbf{Example}                             \\ \hline
et                   & and              & Elle a un chien et un chat. & She has a dog and a cat.          \\ \hline
mais                 & but              & Mais pas maintenant. & But not now.                             \\ \hline
ou                   & or               & Oui ou non ? & Yes or no?                                       \\ \hline
comme                & as/like          & Je suis comme ça. & I am like that.                             \\ \hline
donc                 & so/thus          & Il est jeune, donc il est petit. & He is young, so he is small. \\ \hline
car                  & because          & Jelis, car j'aime ce livre. & I read because I like this book.  \\ \hline
\end{tabular}\end{center}

The conjunction \guillemotleft car \guillemotright means ``because'', and it's usually reserved for writing. The subordinating conjunction \guillemotleft parce que  \guillemotright is preferred in speech; you'll learn this in \textit{Conjunctions 2}.

\begin{itemize}
  \item  Comme vous savez. \\ As you know.
  \item  Elle mange du pain \Blue{quand} elle veut. \\ She eats bread \Blue{when} she wants.
  \item  Je lis, \Blue{car} j'aime le livre. \\ I read \Blue{because} I like the book.
\end{itemize}


\pagebreak
\subsection{Questions}
\subsubsection{Inversions}

The most formal way of asking a question is to use an \textbf{inversion}, where the verb appears before its pronoun and the two are connected by a hyphen.

\begin{itemize}
  \item  Boit-il ? \\ Does he drink? / Is he drinking? / He drinks?
  \item  Boivent-ils du lait ? \\ Do they drink milk? / Are they drinking milk? / They drink milk?
\end{itemize}

However, if the subject of the sentence is a noun, then the noun should appear before the verb, although a pronoun still needs to appear afterwards.

\begin{itemize}
  \item  Le lait est-il froid ? \\ Is the milk cold?
  \item  Les chats sont-ils noirs ? \\ Are the cats black?
\end{itemize}

If the verb ends in a vowel, the letter T must be inserted after the verb for euphony. This T is chained onto the pronoun and is meaningless.

\begin{itemize}
  \item  A-t-il un chien ? \\ Does he have a dog?
  \item  Parle-t-elle anglais ? \\ Does she speak English?
\end{itemize}

Inverted forms still obey other grammar rules, like those for \guillemotleft~ il est \guillemotright versus \guillemotleft~ c'est \guillemotright . However, the pronoun in an inversion cannot elide.

\begin{itemize}
  \item  Est-ce un probl{\`e}me ? \\ Is it a problem?
  \item  Est-elle m{\'e}decin ? \\ Is she a doctor?
  \item  Puis-je aider les enfants ? \\ Can I help the children?
\end{itemize}

\subsubsection{Est-ce Que}

\textbf{Est-ce que} (pronounced like ``essk'') can be added in front of a statement to turn it into a question. Remember that \guillemotleft~ que \guillemotright elides in front of vowel sounds.

\begin{itemize}
  \item  \textbf{Est-ce qu'il} boit ? \\ Does he drink? / Is he drinking?
  \item  \textbf{Est-ce que} c'est un probl{\`e}me ? \\ Is it a problem?
  \item  \textbf{Est-ce qu'il} a un chien ? \\ Does he have a dog?
\end{itemize}

\subsubsection{Intonation}

In informal speech, one of the most common ways to ask a question is simply to raise your intonation at the end of a statement, like you'd do in English.

\begin{itemize}
  \item  Il boit ? \\ Is he drinking?
  \item  Il pleut ? \\ Is it raining?
\end{itemize}


\pagebreak
\subsection{Interrogatives}

An \textbf{interrogative} word introduces a question. French has interrogative adjectives, pronouns, and adverbs.

\subsubsection{Interrogative Adjectives}

French has one interrogative adjective with four forms. It translates to ``which'' or ``what'' depending on the context.

\begin{center}\begin{tabular}{|l|l|l|}
\hline
\textbf{Gender}    & \textbf{Singular} & \textbf{Plural} \\ \hline
\textbf{Masculine} & quel              & quels           \\ \hline
\textbf{Feminine}  & quelle            & quelles         \\ \hline
\end{tabular}\end{center}

An interrogative adjective cannot stand alone. It must modify (and agree with) a noun, and that noun must either be adjacent to it or separated by a form of {\^e}tre.

\begin{itemize}
  \item  Quelle fille ? \\ Which girl?
  \item  Quel est le probl{\`e}me ? \\ What is the problem?
\end{itemize}

Quel is also an exclamatory adjective in statements.

\begin{itemize}
  \item  Quelle chance ! \\ What luck!
  \item  Quel grand gar{\c c}on il est ! \\ What a tall boy he is!
\end{itemize}

\subsubsection{Interrogative Pronouns}

Unlike an adjective, an interrogative pronoun can stand alone. For instance, the interrogative pronoun lequel can replace quel + noun. Note that it agrees with the noun it replaces.

\begin{center}\begin{tabular}{|c|c|}
\hline
\textbf{Quel Form}                            & \textbf{Lequel Form}                     \\ \hline
Quel cheval ?---Which horse?                  & Lequel?---Which one?                     \\ \hline
Quels hommes mangent ?---Which men eat?       & Lesquels mangent ?---Which ones eat?     \\ \hline
Quelle robe est rose ?---Which dress is pink? & Laquelle est rose ?---Which one is pink? \\ \hline
Quelles lettres ?---Which letters?            & Lesquelles ?---Which ones?               \\ \hline
\end{tabular}\end{center}

The most common interrogative pronouns are \guillemotleft qui \guillemotright (for people) and \guillemotleft que \guillemotright (for everything else). However, the construction changes based on a number of factors. Qui is the only pronoun that can start a question by itself, but both qui and que can be used with inversion.

\begin{itemize}
  \item  Qui parle ? \\ Who is speaking?
  \item  Qui es-tu ? \\ Who are you?
  \item  Que fait-il ? \\ What is he making?
\end{itemize}

Both can also use est-ce, but est-ce que (which you learned above) can only be used in a question with {\^e}tre or when the pronoun is the object (``what'' or ``whom''). When it is the subject, est-ce qui must be used.

\begin{itemize}
  \item  Qui est-ce qui parle ? \\ Who's speaking? (subj.)
  \item  Qu'est-ce qui se passe ? \\ What is going on? (subj.)
  \item  Qui est-ce que tu appelles ? \\ Whom are you calling? (obj.)
  \item  Qu'est-ce que c'est ? \\ What is it? (question with {\^e}tre)
\end{itemize}

After prepositions and at the end of questions, \guillemotleft que \guillemotright becomes \guillemotleft quoi \guillemotright .

\begin{itemize}
  \item  Le probl{\`e}me est quoi ? \\ What's the problem?
  \item  {\`A} quoi pensez-vous ? \\ What are you thinking about?
\end{itemize}

Qui and que can be very confusing because they can also be relative pronouns. Que can also be a subordinating conjunction. You will learn these uses later.

\subsubsection{Interrogative Adverbs}

A number of interrogative adverbs can be used to request information

\begin{itemize}
  \item  Pourquoi (``why''): Pourquoi manges-tu du pain ?---Why are you eating bread?
  \item  Comment (``how''): Comment allez-vous ?---How are you?
  \item  Quand (``when''): Quand est-ce que tu vas manger ?---When are you going to eat?
  \item  Combien (``how many/much''): Combien d'eau ?---How much water?
  \item  O{\`u} (``where''): O{\`u} suis-je ?---Where am I?
\end{itemize}

Note that when these adverbs are used with intonation-based questions, they can appear at the beginning or the end of the sentence (except pourquoi).

\begin{itemize}
  \item  Tu vas comment ? \\ How are you?
  \item  Vous {\^e}tes d'o{\`u} ? \\ Where are you from?
\end{itemize}


\pagebreak
\subsection{Present Tense 2}

As you learned in ``Verbs Present 1'', Group 3 verbs are considered irregular, but some sparse patterns do exist among the -ir and -er verbs in this group.

\begin{center}\begin{tabular}{|l|l|l|l|l|l|}
\hline
\textbf{Subject} & \textbf{G1 parler} & \textbf{G2 finir} & \textbf{G3 dormir} & \textbf{G3 ouvrir} & \textbf{G3 vendre} \\ \hline
je               & parl\textbf{e}              & fini\textbf{s}             & dor\textbf{s}               & ouvr\textbf{e}              & vend\textbf{s}              \\ \hline
tu               & parl\textbf{es}             & fini\textbf{s}             & dor\textbf{s}               & ouvr\textbf{es}             & vend\textbf{s}              \\ \hline
il/elle/on       & parl\textbf{e}              & fini\textbf{t}             & dor\textbf{t}               & ouvr\textbf{e}              & vend               \\ \hline
nous             & parl\textbf{ons}            & fini\textbf{ssons}         & dor\textbf{mons}            & ouvr\textbf{ons}            & vend\textbf{ons}            \\ \hline
vous             & parl\textbf{ez}             & fini\textbf{ssez}          & dor\textbf{mez}             & ouvr\textbf{ez}             & vend\textbf{ez}             \\ \hline
ils/elles        & parl\textbf{ent}            & fini\textbf{ssent}         & dor\textbf{ment}            & ouvr\textbf{ent}            & vend\textbf{ent         }   \\ \hline
\end{tabular}\end{center}

Among the G3 -ir verbs, some conjugate like dormir, while verbs like ouvrir conjugate as though they're -er verbs. Note that singular conjugations of dormir drop the last letter of the root. Also, while some -re verbs (such as attendre, entendre, and perdre) conjugate like vendre, dozens of other conjugation patterns exist, so it's best to memorize each verb's conjugation individually.

\subsubsection{Transitive and Intransitive Verbs}

Sentences can have grammatical objects, which are nouns that are affected by a verb. There are two types of objects: \textbf{direct objects}, which are nouns acted upon, and \textbf{indirect objects}, which are nouns that are indirectly affected by the action.  In the example ``Ben threw the ball at him'', ``Ben'' is the subject, ``the ball'' is the direct object, and ``him'' is the indirect object. You can usually recognize indirect objects in English by looking for a preposition after a verb. Identifying objects is important, especially in French.  Verbs can be \textbf{transitive}, \textbf{intransitive}, or both. Transitive verbs can have direct objects, while intransitive verbs cannot. However, both types of verbs can have indirect objects.

\begin{itemize}
  \item  Transitive: Je lance une chaussure. \\ I throw a shoe.
  \item  Intransitive: Je parle {\`a} Jacques. \\ I am speaking to Jacques.
\end{itemize}

\textit{Parler} is an interesting example because it's intransitive for everything but language names.

\begin{itemize}
  \item  Transitive: Je parle anglais. \\ I speak English.
\end{itemize}

French verbs can be tricky for Anglophones because some transitive verbs in French have intransitive English translations and vice versa. Pay attention to this.

\begin{itemize}
  \item  Transitive: Le chat regarde le chien. \\ Intransitive: The cat is looking at the dog.
  \item  Intransitive: Il t{\'e}l{\'e}phone {\`a} son ami. \\ Transitive: He is calling his friend.
\end{itemize}

\subsubsection{Stative Verbs in English}

Unlike dynamic verbs, which describe actions and processes, stative verbs describe states of being, physical and mental states, possession, sensations, and so on. The most common stative verb is ``to be''. Here are some other common examples:

\begin{itemize}
  \item  Possessing: belong, get, have, own, possess
  \item  Feeling: hate, like, love, need, want
  \item  Sensing: feel, hear, see, smell, taste
  \item  Thinking: believe, know, recognize, think, understand
\end{itemize}

The most important detail about stative verbs is that they can't be used in continuous tenses in English.

\begin{itemize}
  \item  C'est mon fils. \\ He is my son. (Not ``is being''.)
  \item  Je veux une pomme. \\ I want an apple. (Not ``am wanting''.)
  \item  Elle aime son chien. \\  She loves her dog. (Not ``is loving''.) 
  \item  On a deux amis. \\ We have two friends. (Only cannibals ``are having'' their friends.)
\end{itemize}

You may have noticed that some verbs can be both stative and dynamic based on context. For instance:

\begin{itemize}
  \item  ``To have'' can be dynamic when it means ``to consume''.
  \item  ``To feel'' is stative, but ``to feel sick" or "to feel better'' are dynamic.
  \item  ``To be'' can be dynamic when it means ``to act''.
\end{itemize}

Pay attention to this nuance when translating into English. This problem rarely occurs when translating to French because it lacks continuous tenses. 

\subsubsection{Impersonal Expressions}

A number of other impersonal verbs have to do with weather.

\begin{itemize}
  \item  Pleuvoir (``to rain''): Il pleut. \\ It is raining.
  \item  Neiger (``to snow''): Il neige. \\ It is snowing.
  \item  Faire chaud (``to be warm''): Il fait chaud aujourd'hui. \\ It is warm today.
\end{itemize}

Chaud can be replaced with a number of other adjectives, like froid (``cold'') or humide (``humid'').

\subsubsection{Confusing Verbs}

Like their English counterparts, voir (``to see'') and regarder (``to watch'') differ based on the subject's intention. If the subject is actively watching or looking for something, use regarder. Otherwise, use voir.

\begin{itemize}
  \item  Le chat regarde le poisson. \\ The cat is watching the fish.
  \item  Elle peut voir la ville. \\ She can see the city.
\end{itemize}

\subsubsection{Vocabulary}

\begin{center}\begin{tabular}{l|l||l|l}
\textbf{French} & \textbf{English} & \textbf{French} & \textbf{English} \\ \hline
respecter & to respect & porter & to carry \\ 
contenir & to contain & poser & to set down \\ 
couper & to cut & poss{\`e}der & to own \\ 
courir & to run & prendre & to take \\ 
cuire & to cook & proposer & to suggest \\ 
dire & to say & saisir & to grab \\ 
donner & to give & souhaiter & to wish \\ 
entrer & to enter & suivre & to follow \\ 
finir & to finish & terminer & to finish \\ 
gagner & to win & tenir & to hold \\ 
int{\'e}resser & to interest & trouver & to find \\ 
lancer & to throw & vendre & to sell \\ 
laver & to wash & voir & to see \\ 
marcher & to walk & vivre & to live \\ 
m{\'e}riter & to deserve & ajouter & to add \\ 
mettre & to put & am{\'e}liorer & to improve \\ 
montrer & to show & concerner & to concern \\ 
cacher & to hide & casser & to break \\ 
motiver & to motivate & devenir & to become \\ 
neiger & to snow & dormir & to sleep \\ 
pleuter & to rain & entendre & to hear \\ 
ourvir & to open & regarder & to watch \\ 
perdre & to lose \\
\end{tabular}\end{center}

\begin{itemize}
%  \item  J'agis. \\ I am taking action.
%  \item  J'ai mal agi. \\ I have acted poorly.
  \item  O{\`u} vivez-vous? \\ Where do you live?
  \item  Le livre concerne une femme. \\ The book has to do with a woman.
\end{itemize}


\pagebreak
\subsection{Adjectives 2}

When multiple adjectives modify a noun, they should come before or after the noun based on the same rules as if they were the only adjective. This means that adjectives may straddle the noun if one is a BANGS adjective.

\begin{itemize}
  \item  La grande robe rouge \\ The big red dress
  \item  Une jeune fille fran{\c c}aise \\ A young French girl
\end{itemize}

When arranging multiple adjectives on the same side, concrete adjectives should usually be placed closer to the noun than abstract ones.

\begin{itemize}
  \item  J'ai un joli petit chat gris. \\ I have a lovely little grey cat.
  \item  J'ai un chat blanc courageux. \\ I have a brave white cat.
\end{itemize}

You can add conjunctions and adverbs to break up multiple adjectives.

\begin{itemize}
  \item  J'ai un chapeau blanc et bleu. \\ I have a white and blue hat.
  \item  L'homme fort et s{\'e}rieux \\ The strong and serious man.
  \item  Voici mon premier petit chat blanc et courageux. \\ Here is my first small white and brave cat.
  \item  J'adore mon propre tout petit chat blanc tr{\`e}s doux. \\ I love my own tiny white and very soft cat.
\end{itemize}

When there are multiple nouns being described by one adjective, that adjective takes the masculine plural by default.

\begin{itemize}
  \item  Un garçon et une fille italiens \\ An Italian boy and girl
  \item  J'ai une chemise et un manteau simples. \\ I have a simple shirt and coat.
\end{itemize}

However, if the nouns are all feminine, then they can take the feminine plural.

\begin{itemize}
  \item  La robe et la jupe vertes \\ The green dress and skirt
\end{itemize}

\pagebreak
\subsubsection{Grand or Gros?}

Grand and gros can both mean "big", but they're only partly interchangeable.  Grand tends to be used for:

\begin{itemize}
  \item  General size: La grande robe---The big dress
  \item  Height: L'enfant est grand.---The child is tall/big.
  \item  Area: La ville est grande.---The city is big.
  \item  Figurative size: La grande richesse---The great wealth
  \item  Importance: Un grand homme---A great man
\end{itemize}

Gros tends to be used for:

\begin{itemize}
  \item  Thickness or volume: Une grosse boîte de petits-pois---A big can of peas
  \item  Fatness: Un gros chat---A fat cat
  \item  Things that are round: Une grosse pomme---A big apple
  \item  Seriousness: Un gros probl{\`e}me---A big (serious) problem
\end{itemize}

\subsubsection{Faux Amis}

Many English and French words look alike and share meanings. This is because English is heavily influenced by French and Latin. However, there are faux amis (``false friends'') that look similar but do not have the same meaning. For instance, gros looks like ``gross'', but their meanings are not the same. Be careful before assuming a French word's meaning based on its English lookalike.

\pagebreak
\subsubsection{Vocabulary}

\begin{center}\begin{tabular}{l|l||l|l}
\textbf{French} & \textbf{English} & \textbf{French} & \textbf{English} \\ \hline
bon & good (BANGS) & gros & fat \\ 
m{\^e}me & same & b{\^e}te & stupid \\ 
premi{\`e}re & first & {\'e}norme & enormous \\ 
long & long & dr{\^o}le & funny \\ 
bien & good & m{\'e}chant & vicious \\ 
dernier & shirt & laid & ugly \\ 
beau & beautiful & japonais & Japanese \\ 
g{\'e}nial & brilliant & sombre & dark (colored) \\ 
belle & beautiful (BANGS) & anglaise & English \\ 
simple & simple & europ{\'e}ens & European \\ 
fran{\c c}aise & french & italien & Italian \\ 
second & second & chinois & Chinese \\ 
mauvais & bad (BANGS) & espagnol & Spanish \\ 
difficile & difficult & africaine & African \\ 
nombreux & numerous & sympa & friendly \\ 
impossible & impossible & am{\'e}ricain & American \\ 
deuxi{\`e}me & second (BANGS) & s{\'e}rieux & serious \\ 
troisi{\`e}me & third (BANGS) & pr{\^e}t & ready \\ 
diff{\'e}rent & different & sec & dry \\ 
important & important & allemand & German \\ 
clair & light (colored) & gentil & gentle \\ 
fort & strong & seul & only (BANGS) \\ 
large & big & propre & clean \\ 
pauvre & poor & sale & dirty \\ 
dur & hard & l{\'e}g{\`e}r & light[weight] \\ 
grosse & large & joli & pretty \\ 
\end{tabular}\end{center}


\pagebreak
\subsection{Pronouns}

\subsubsection{On}

\guillemotleft On \guillemotright is a versatile and ubiquitous French indefinite subject pronoun. Francophones usually say on to refer to ``us'', ``them'', or one or more unidentified persons. \guillemotleft On \guillemotright is always masculine and third-person singular, which is why conjugation charts often list il/elle/on together.

\begin{itemize}
  \item  On mange. \\ We are eating.
  \item  On est seul. \\ We are alone.
\end{itemize}

\guillemotleft On \guillemotright can also be used more formally in the passive voice or for general statements, much like the English ``one''.

\begin{itemize}
  \item  On doit dormir assez. \\ One must sleep adequately.
\end{itemize}

\subsubsection{Direct Objects}

\begin{center}\begin{tabular}{|c|c|}
\hline
\textbf{English}           & \textbf{Direct Object} \\ \hline
me                         & me                     \\ \hline
you (sing.)                & te                     \\ \hline
him                        & le                     \\ \hline
her                        & la                     \\ \hline
us                         & nous                   \\ \hline
you(plur. or formal sing.) & vous                   \\ \hline
them                       & les                    \\ \hline
\end{tabular}\end{center}

Direct object pronouns usually come before their verbs.

\begin{itemize}
  \item  L'enfant me voit. \\ The child sees me.
  \item  Le lion le mange. \\ The lion eats it (or "him"!).
  \item  Vous nous aimez. \\ You love us.
  \item  Je t'aime. \\ I love you.
\end{itemize}

Me/te/le/la elide, so make sure you notice them when they hide in the first syllable of a verb.

\begin{itemize}
  \item  Elle m'attend. \\ She is waiting for me.
  \item  L'enfant l'appelle. \\ The child calls to him/her.
\end{itemize}

Le and les only contract when they're articles, not when they're object pronouns.

\begin{itemize}
  \item  Je suis en train de le faire. (Not du faire) \\ I am in the process of doing it.
\end{itemize}

\subsubsection{En Replaces De + Noun}

The adverbial pronoun \guillemotleft en \guillemotright can be used to replace objects introduced by de. For instance, it can replace a partitive article + noun.

\begin{itemize}
  \item  Avez-vous de l'argent ? \\ Do you have some money?
  \item  Oui, j'en ai. \\ Yes, I have some.
\end{itemize}

\guillemotleft En \guillemotright may replace nouns or pronouns in verb constructions that use de, like parler de (``to talk about'').

\begin{itemize}
  \item  Marc parle de son fr{\`e}re ? \\ Is Marc talking about his brother?
  \item  Oui, il en parle. \\ Yep, he's talking about him.
\end{itemize}

Nouns in adverbs of quantity can also be replaced with \guillemotleft en \guillemotright .

\begin{itemize}
  \item  Achetez-vous beaucoup de livres ? \\ Are you buying a lot of books?
  \item  Oui, j'en ach{\`e}te beaucoup. \\ Yes, I am buying a lot [of them].
\end{itemize}

\subsubsection{Y Can Refer to a Place}

The adverbial pronoun \guillemotleft y \guillemotright can refer to a previously mentioned or implied place, in which case it's usually translated as ``there''.  In English, ``there'' may be omitted, but the same is not true of \guillemotleft y \guillemotright in French. \guillemotleft Je vais \guillemotright is not a complete sentence without \guillemotleft y \guillemotright .

\begin{itemize}
  \item  Allez-vous au restaurant ? \\ Are you going to the restaurant?
  \item  Oui, j'y vais. \\ Yes, I'm going there.
\end{itemize}

\subsubsection{The Relative Pronouns Que and Qui}

\textbf{Relative pronouns} introduce relative clauses, which are subordinate clauses that elaborate upon a previously mentioned noun (the antecedent). Use \guillemotleft que \guillemotright when the relative pronoun is the direct object (``whom'' in English) and use \guillemotleft qui \guillemotright when it is the subject (``who'' in English).

\begin{itemize}
  \item  C'est l'homme que je connais. \\ He's the man whom (or ``that'') I know.
  \item  La fille qui lit un menu. \\ The girl who (or ``that'') reads a menu.
\end{itemize}

If you have trouble figuring out whether to use \guillemotleft qui \guillemotright or \guillemotleft que \guillemotright , try rephrasing the sentence without the relative pronoun. Use \guillemotleft qui \guillemotright if the antecedent is the subject; otherwise, use \guillemotleft que \guillemotright .

\begin{itemize}
  \item  Subject: La fille qui lit un menu. $\Rightarrow$ La fille lit un menu.
  \item  Object: C'est l'homme que je connais. $\Rightarrow$ Je connais l'homme.
\end{itemize}

\subsubsection{The Reflexive Pronoun Se}

A \textbf{reflexive pronoun} like \guillemotleft se \guillemotright can be used to indicate that a verb acts upon the subject. \guillemotleft Se \guillemotright is used with all third-person subjects, regardless of gender and number.

\begin{itemize}
  \item  Il s'aime. \\ He loves himself.
  \item  Il s'appelle comment ? \\ What's his name? (Lit, "He calls himself what?")
  \item  Elle se demande pourquoi. \\ She wonders why. (Lit, "She asks herself why.")
\end{itemize}

When \guillemotleft se \guillemotright refers to a plural subject, it can also be reciprocal or mutual (``each other'').

\begin{itemize}
  \item  Ils s'aiment. \\ They love each other.
  \item  Les filles se parlent. \\ The girls speak to each other.
  \item  On se parle quand ? \\ When do we speak to each other?
  \item  On se voit bient{\^o}t. \\ We will see each other soon.
\end{itemize}

Certain pronouns can be added to the end of the sentence to differentiate between reflexive and reciprocal uses if necessary.

\begin{itemize}
  \item  Ils s'aiment eux-m{\^e}mes. \\ They love themselves.
  \item  Elles s'aiment elles-m{\^e}mes. \\ They love themselves.
  \item  Ils s'aiment l'un l'autre. \\ They love each other.
  \item  Elles s'aiment les unes les autres. \\ They love one another.
\end{itemize}

\begin{itemize}
  \item  {\c C}a se peut. \\ It can be done.
  \item  Qu'est-ce que tu as ? \\ What have you got?
  \item  Je sais o{\`u} c'est. \\ I know where it is.
  \item  Je veux boire quelque chose. \\ I want to drink something.
\end{itemize}