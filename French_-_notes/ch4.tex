\subsection{Present 3}

A \textbf{pronominal verb} requires a reflexive pronoun, which is a special kind of pronoun that agrees with and refers back to the subject. They're identical to direct object pronouns except for the third-person se.

\begin{center}\begin{tabular}{|c|c|c|}
\hline
\textbf{Person} & \textbf{Singular} & \textbf{Plural} \\ \hline
1st             & me                & nous            \\ \hline
2nd             & te                & vous            \\ \hline
3rd             & se                & se              \\ \hline
\end{tabular}\end{center}

One type of pronominal verb, the reflexive verb, describes an action being done by the subject to the subject.

\begin{itemize}
  \item  Je me dis que ce n'est pas possible. \\ I tell myself that it isn't possible.
	\item  Vous vous levez. \\ You are getting up. (Lit. You raise yourself.)
	\item  La femme se prom{\`e}ne. \\ The woman goes for a walk. (Lit. walks herself.)
\end{itemize}

Reflexive verbs include se in their infinitive forms (e.g. se promener). It isn't necessary to include the reflexive pronoun in the English translation. Also, the reflexive verb should come after ne in negations.

\begin{itemize}
  \item  Ils se rasent. \\ They are shaving.
	\item  Elle ne se rase pas. \\ She doesn't shave. 
\end{itemize}

The other kinds of pronominal verbs are reciprocal, passive, and subjective. You will learn these later.

\subsubsection{Pronoun Order}

When two object pronouns modify the same verb, they always appear in a predefined order: \\
me/te/nous/vous/se $\rightarrow$ le/la/les $\rightarrow$ lui/leur $\rightarrow$ y $\rightarrow$ en.

\begin{itemize}
  \item  Je vous la laisse. \\ I am leaving it for you.
	\item  Nous nous la r{\'e}servons. \\ We reserve it for ourselves.
	\item  Ils nous le donnent. \\ They are giving it to us.
	\item  Ils le lui donnent. \\ They are giving it to him.
\end{itemize}

\subsubsection{Verbs with {\`A} and De}

As you learned previously, {\`a} or de can appear after a verb to introduce an infinitive or object. You should consider such a preposition to be an integral part of the verb that completes or changes its meaning.

\begin{itemize}
  \item  Je commence {\`a} manger. \\ I am starting to eat.
	\item  Ma ni{\`e}ce essaie de dormir. \\ My niece is trying to sleep.
	\item  Je pense {\`a} des {\'e}l{\'e}phants roses. \\ I am thinking about pink elephants.
	\item  Que pensez-vous de ce film ? \\ What do you think of that film?
\end{itemize}

However, recall from ``Verbs: Present 1'' that semi-auxiliary verbs can introduce other verbs without needing a preposition.

\begin{itemize}
  \item  Je veux lire. \\ I want to read.
	\item  Il aime manger. \\ He likes to eat.
\end{itemize}

For verbs appended with {\`a} (like penser {\`a}), the adverbial pronoun y can replace {\`a} + a thing.

\begin{itemize}
  \item  Tu penses {\`a} l'examen ? Oui, j'y pense encore. \\ Are you thinking about the test? Yes, I'm thinking about it again.
	\item  Il croit aux fant{\^o}mes ? Oui, il y croit. \\ Does he believe in ghosts? Yes, he believes in them.
\end{itemize}

To replace {\`a} + a person or animal, use an indirect object pronoun instead.

\begin{itemize}
  \item  Je lui parle. \\ I am talking to him/her.
	\item  Elle me t{\'e}l{\'e}phone maintenant. \\ She is calling me right now.
\end{itemize}

\subsubsection{Confusing Verbs}

Demander {\`a} means ``to ask to'' when followed by an infinitive.

\begin{itemize}
  \item  Elle demande {\`a} payer avec des dollars. \\ She asks to pay with dollars.
\end{itemize}

However, when used with nouns, demander is particularly confusing because its direct and indirect object are the opposite of its English counterpart, ``to ask''.

\begin{itemize}
  \item  Je demande une baguette. \\ I ask for a baguette. (Not ``I ask a baguette.'')
	\item  Je demande une baguette {\`a} la boulang{\`e}re. \\ I ask the baker for a baguette.
	\item  Je lui demande de me donner une baguette. \\ I ask him/her to give me a baguette.
\end{itemize}

{\'E}couter means ``to listen'' in the literal sense of intentionally listening or paying attention to something.

\begin{itemize}
  \item  J'{\'e}coute de la musique. \\ I am listening to music.
	\item  Elle {\'e}coute la voix de la sagesse. \\ She listens to the voice of reason.
\end{itemize}

Entendre can mean hear, listen, or (rarely) understand.

\begin{itemize}
  \item  J'entends du bruit. \\ I hear noise.
	\item  Elle ne veut rien entendre. \\ She won't listen.
\end{itemize}

Manquer means ``to miss'', but the pronouns are flipped from its English counterpart. If it helps, you can think of manquer as ``to be missed by''.

\begin{itemize}
  \item  Vous me manquez. \\ I miss you.
	\item  Je vous manque. \\ You miss me.
\end{itemize}

Plaire {\`a} is commonly translated as ``to like'', but for grammatical purposes, think of it as ``to please'' or ``to be pleasing to''.

\begin{itemize}
  \item  La jupe pla\^{\i}t aux filles. \\ The girls like the skirt. / The skirt is pleasing to the girls.
	\item  {\c C}a me pla\^{\i}t. \\ I like it. / That is pleasing to me.
\end{itemize}

Se lever (``to get up'') means to physically get up from a non-standing position, not to wake up, which is se r{\'e}veiller.


\begin{center}\begin{tabular}{l|l||l|l||l|l}
\textbf{Fran{\c c}ais} & \textbf{English} & \textbf{Fran{\c c}ais} & \textbf{English}  & \textbf{Fran{\c c}ais} & \textbf{English} \\ \hline
affirmer & to claim & jouer {\`a} & to play (with) & remarquer & to notice  \\ 
appartener & to belong & laisser & to leave & retour & to return (there)  \\ 
arriver & to arrive & commencer & to start & r{\'e}pondre & to respond  \\ 
se bronzer & to [get] tan & se l{\'e}ver & to rise (get up) & reposer & to rest  \\ 
continuer & to continue & manquer & to miss (out) & repr{\'e}senter & to represent  \\ 
venir & to come & mentir & to lie & r{\'e}server & to reserve  \\ 
demander & to ask for & partir & to leave & rester & to stay (vacation)  \\ 
douter de & to doubt (something) & passer & to spend (time) & revenir & to come back  \\ 
croie & to believe & penser & to think & sembler & to seem  \\ 
d{\'e}pendre de & to depend (on) & ressembler & to look like & utilser & to use  \\ 
{\'e}change & to exchange & plait & is liked & se sentir & to feel  \\ 
{\'e}couter & to listen & planter & to plant & \Red{signifie} & meaning  \\ 
travailler & to work & se pr{\'e}f{\`e}rer & to prefer & sorter & to go out  \\ 
servir & to serve & pr{\'e}senter & to present & supposer & to suppose  \\ 
essaier & to try & se prom{\`e}ner & to go for a walk & tomber & to fall  \\ 
exister & to exist & re{\c c}evoir & to receive & (se) enfuir & to flee  \\ 
expliquer & to explain & offrer & to offer/buy & se raser & to shave  \\ 
habiter & to live (in) & refuser & to refuse  \\  
\end{tabular}\end{center}

\begin{itemize}
  \item  Ce costume m'appartient. \\ This suit belongs to me. 
	\item  Je viens d'Am{\'e}rique. \\ I come from America.
	\item  {\c C}a d{\'e}pend. \\ It depends.
	\item  J'habite en banlieue. \\ I live in the suburbs.
	\item  Je pr{\'e}f{\`e}re le riz au pain. \\ I prefer rice over bread.
	\item  {\c C}a repr{\'e}sente quoi ? \\ What does it represent?
	\item  Ils semblent mauvais. \\ They seem bad.
	\item  Les gar{\c c}ons resent des gar{\c c}ons. \\ Boys will be boys.
	\item  Ce th{\'e} sent bon. \\ The tea smells good.
	\item  Qu'est-ce que {\c c}a signifie ? \\ What is the meaning of this?
\end{itemize}


\pagebreak
\subsection{Past Imperfect}

French has a few past tenses, one of which is the imperfect (imparfait). You can construct it by taking the present indicative nous form of any verb and replacing the -ons with the imperfect ending. Notice that all the conjugated forms except the nous and vous forms have the same sound.

\begin{center}\begin{tabular}{|c|c|c|c|c|c|}
\hline
\textbf{Subject} & \textbf{Ending} & \textbf{{\^E}tre} & \textbf{Parler} & \textbf{Manger} & \textbf{Aller} \\ \hline
je (j')          & -ais            & {\'e}tais         & parlais         & mangeais        & allais         \\ \hline
tu               & -ais            & {\'e}tais         & parlais         & mangeais        & allais         \\ \hline
il/elle/on       & -ait            & {\'e}tait         & parlait         & mangeait        & allait         \\ \hline
nous             & -ions           & {\'e}tions        & parlions        & mangions        & allions        \\ \hline
vous             & -iez            & {\'e}tiez         & parliez         & mangiez         & alliez         \\ \hline
ils/elles        & -aient          & {\'e}taient       & parlaient       & mangeaient      & allaient       \\ \hline
\end{tabular}\end{center}

The only irregular imperfect verb is {\^e}tre, which takes on an {\'e}t- root. However, for spelling-changing verbs that end in -ger or -cer (e.g. manger), add an ``e'' to the root so the consonant remains soft.

\begin{itemize}
  \item  Kilroy {\'e}tait ici. \\ Kilroy was here.
	\item  Elle mangeait avec ses amis. \\ She was eating with her friends.
\end{itemize}

\subsubsection{Translating the Imperfect}

Translating the past tense between English and French can be difficult because there is no simple mapping between the English past tenses and the two main French past tenses, the imparfait and the pass{\'e} compos{\'e} (taught in the next unit). When choosing a tense, pay close attention to what you're trying to express.

The imperfect describes situations, states of mind, and habits in the past. In a story, it sets the scene or background; thus, it often translates to and from the English past continuous tense.

\begin{itemize}
  \item  Il rentrait chez lui. \\ He was going home.
	\item  Dis donc ! Je mangeais {\c c}a ! \\ Hey! I was eating that!
\end{itemize}

For repeated actions or habits, you can also use constructions with ``used to'' or ``would''.

\begin{itemize}
  \item  Nous visitions un monument chaque semaine. \\ We used to visit one monument every week.
	\item  {\`A} l'{\'e}poque, elle chantait souvent. \\ Back then, she would often sing.
\end{itemize}

A lot of confusion stems from the versatile English preterit (simple past), which overlaps both French tenses. For instance, the preterit can also be used for habits.

\begin{itemize}
  \item  Nous visitions un monument chaque semaine. \\ We visited one monument every week.
	\item  {\`A} l'{\'e}poque, elle chantait souvent. \\ Back then, she often sang.
\end{itemize}

As you learned in \textit{Verbs: Present 2}, stative verbs (e.g. ``to be'', ``to think'') usually can't be used in English continuous tenses. When used in past tenses, they should translate to the preterit.

\begin{itemize}
  \item  Il croyait son p{\`e}re. \\ He believed his father. (Not ``was believing''.)
	\item  Nous avions trois cousins. \\ We had three cousins. (Using ``were having'' would make you a confessed cannibal.)
\end{itemize}

\subsubsection{Using the Imperfect}

\begin{itemize}
  \item  \textbf{States or situations}  Use the preterit here to describe mental or physical conditions, scenes, dates or times, weather, etc. Remember that you should never use English continuous tenses for stative verbs. In the examples below, looked, smelled, and understood are stative verbs.
	
	\begin{itemize}
  \item  Il {\'e}tait malade. \\ He was sick.
	\item  Elle avait froid. \\ She was cold.
	\item  Nous avions vingt ans. \\ We were twenty.
	\item  Tu semblais heureux. \\ You looked happy. (Not "were looking".)
	\item  Il {\'e}tait trois heures. \\ It was 3:00.
	\item  Vos fleurs sentaient si bon ! \\ Your flowers smelled so nice! (Not "were smelling".)
	\item  Elle comprenait mes sentiments. \\ She understood my feelings. (Not "was understanding".)
	\item  Il y avait des bateaux. \\ There were boats.
\end{itemize}
	
	Also, when using il y a in other tenses, conjugate avoir to match. For the Imperfect, it becomes avait.
	
	\item  \textbf{Actions or processes}  The continuous past can be used here to set up a scene by describing an action or process.
	
	\begin{itemize}
  \item  Je marchais lentement. \\ I was walking slowly. 
	\item  Vous regardiez la mer. \\ You were watching the sea.
	\item  Elles pensaient à leurs enfants. \\ They were thinking of their children. (``Thinking'' is a process here.)
	\item  Nous sentions la soupe \\ We were smelling the soup. (Process of perceiving an odor.)
	\item  Il pleuvait fort. \\ It was raining hard.
\end{itemize}

  Note that ``was'' and ``were'' are the preterit forms of ``to be'', but they are also auxiliary verbs for the continuous past when used before another verb in gerund.
	
	\item  \textbf{A habit or repeated action}
	
	\begin{itemize}
  \item  Nous nous entraînions chaque semaine. \\ We used to train every week.
	\item  Il r{\'e}citait des po{\`e}mes. \\ He would (or) used to recite poems.
	\item  Je ressentais souvent de la douleur. \\ I frequently felt pain.
\end{itemize}

  Note that you shouldn't use the past continuous here, but as mentioned before, you may use the preterit, ``used to'', or ``would''.
\end{itemize}


\pagebreak
\subsection{Compound Past}

Compound verbs contain at least two words: a conjugated auxiliary and a participle. In this unit, we will cover the pass{\'e} compos{\'e} (PC), which can translate to the English present perfect.

\begin{itemize}
  \item  Elle a vu ce chien. \\ She has seen that dog.
	\item  Ils ont dit la verit{\'e}. \\ They have told the truth.
\end{itemize}

In both languages, the compound verb begins with a conjugated auxiliary verb (avoir and ``to have'' here) that agrees with the subject. A past participle (e.g. vu or ``seen'') follows the auxiliary.

\subsubsection{Auxiliaries}

In English, the active present perfect has only one auxiliary verb ("to have"), but the PC has two: avoir and être. Most verbs use avoir.

\begin{itemize}
  \item  J'ai {\'e}t{\'e} malade. \\ I have been sick.
	\item  Il a appel{\'e} un docteur. \\ He has called a doctor.
\end{itemize}

A handful of verbs use {\^e}tre. The mnemonic ``ADVENT'' may help you remember these.

\begin{center}\begin{tabular}{|l|l|l|}
\hline
\textbf{Initial Verb} & \textbf{Opposite Verb} & \textbf{Related Verbs}             \\ \hline
Arriver (arrive)      & partir (leave)         &                                    \\ \hline
Descendre (descend)   & monter (ascend)        &                                    \\ \hline
Venir (come)          & aller (go)             & devenir (become), revenir (return) \\ \hline
Entrer (enter)        & sortir (leave)         & rentrer (re-enter)                 \\ \hline
Na{\^\i}tre (be born)      & mourir (die)           &                                    \\ \hline
Tomber (fall)         &                        &                                    \\ \hline
\end{tabular}\end{center}

The remaining verbs are passer (pass), rester (stay), retourner (return), and accourir (run up). Notice that {\^e}tre verbs involve movement or transformation.

\begin{itemize}
  \item  Il est venu. \\ He has come.
	\item  Septembre est pass{\'e}. \\ September has passed.
	\item  Je suis devenu roi. \\ I have become king.
\end{itemize}

Also, all pronominal verbs use {\^e}tre.


\begin{itemize}
  \item  Il s'est souvenu de ses amis. \\ He has remembered his friends.
	\item  Il s'est ras{\'e}. \\ He has shaved.
\end{itemize}

Object pronouns, negations, and inversions appear around the auxiliary.

\begin{itemize}
  \item  Je l'ai entendu. \\ I have heard him.
	\item  Il ne m'a pas trouv{\'e}. \\ He has not found me.
	\item  Avez-vous vu les robes ? \\ Have you seen the dresses?
	\item  Pourquoi l'avez-vous fait ? \\ Why have you done it?
\end{itemize}

\subsubsection{Past Participles}

A participle is a special non-conjugated form of a verb. Most participles are formed by adding an ending to a verb's root.

\begin{center}\begin{tabular}{|c|c|c|}
\hline
\textbf{Group} & \textbf{Ending} & \textbf{Example} \\ \hline
-er verbs      & -{\'e}              & manger $\rightarrow$ mang{\'e}   \\ \hline
-ir verbs      & -i              & choisir $\rightarrow$ choisi \\ \hline
-re verbs      & -u              & vendre $\rightarrow$ vendu          \\ \hline
\end{tabular}\end{center}

Unfortunately, most irregular verbs have irregular participles. For instance, the past participle of venir is venu.

\begin{itemize}
  \item  Il est venu. \\ He has come.
	\item  Les filles sont venues. \\ The girls have come.
\end{itemize}

Note that participles vary with gender and number just like adjectives.

\begin{center}\begin{tabular}{|c|c|c|}
\hline
\textbf{Gender} & \textbf{Singular} & \textbf{Plural} \\ \hline
Masculine       & venu              & venus           \\ \hline
Feminine        & venue             & venues          \\ \hline
\end{tabular}\end{center}

Adverbs appear right before the participle.

\begin{itemize}
  \item  Je l'ai souvent entendu. \\ I often heard him/her/it.
	\item  Je vous en ai d{\'e}j{\`e} parl{\'e}. \\ I already talked to you about it.
\end{itemize}

A participle that follows avoir is usually invariable.

\begin{itemize}
  \item  L'homme a mang{\'e}. \\ The man has eaten.
	\item  Les femmes ont mang{\'e}. \\ The women have eaten.
\end{itemize}

However, if a Direct Object appears before avoir, its participle agrees with the DO. Below, vues agrees with the plural feminine robes because les precedes the verb.

\begin{itemize}
  \item  Tu as vu les robes ? \\ Have you seen the dresses?
	\item  Oui, je les ai vues. \\ Yes, I have seen them.
\end{itemize}

A participle that follows {\^e}tre agrees with the subject.

\begin{itemize}
  \item  L'homme est venu. \\ The man has come.
	\item  Les hommes sont venus. \\ The men have come.
	\item  La femme est venue. \\ The woman has come.
	\item  Les femmes sont venues. \\ The women have come.
\end{itemize}

However, if a pronominal verb has no DO, then the participle is invariable. For instance, compare s'appeler (transitive) to se telephoner (no DO).

\begin{itemize}
  \item  Nous nous sommes appel{\'e}s. \\ We called each other. (For a masculine nous.)
	\item  Nous nous sommes t{\'e}l{\'e}phon{\'e}. \\ We called each other. (For both genders of nous.)
\end{itemize}

Translating the past tense can be difficult because the English simple past (preterit) overlaps the French pass{\'e} compos{\'e} and imparfait (taught in the previous unit). The PC can translate to the preterit when it narrates events or states that began and ended in the past. In this usage, the PC often appears with expressions of time or frequency like il y a, which means ``ago'' when followed by a duration.

\begin{itemize}
  \item  La fille a mang{\'e} il y a cinq minutes. \\ The girl ate five minutes ago. (A single specific event.)
	\item  Les enfants ont eu froid hier. \\ The children were cold yesterday. (A state on a specific date.)
	\item  Je suis tomb{\'e}(e) plusieurs fois. \\ I fell several times. (Multiple specific actions.)
	\item  Je suis d{\'e}j{\`a} tomb{\'e}(e). \\ I already fell. (An event in an undetermined time frame.)
\end{itemize}

The PC can also translate to the present perfect for actions and states that started in the past and are still true.

\begin{itemize}
  \item  Il n'a jamais mang{\'e} de p{\^a}tes. \\ He has never eaten pasta.
	\item  Tu as perdu tes cl{\'e}s. \\ You have lost your keys.
\end{itemize}

\begin{itemize}
  \item  Cette page ? Oui, je l'ai lue. \\ That page?  Yes, I have read it.
	\item  Elle a {\'e}t{\'e} terrible. \\ She has been terrible.
	\item  Tu as d{\'e}j{\`a} mang{\'e}. \\ You have already eaten.
	\item  Elles ont {\'e}t{\'e} tr{\'e}s m{\'e}chantes. \\ They have been very mean girls.
	\item  Elles ont bu du lait avec leurs amis. \\ They drank milk with their friends.
	\item  O{\`u} as-tu mang{\'e} hier ? \\ Where did you eat yesterday?
	\item  Il est all{\'e} au magasin. \\ He went to the store.
	\item  Elle n'est pas n{\'e}e en Angleterre. \\ She was not born in England.
	\item  Elle est venue seule. \\ She came alone.
	\item  Elle est all{\'e}e pas {\`a} pas.  She went step by step.
	\item  Il est venu avec nous. \\ He has come with us.
	\item  Deux personnes sont mortes. \\ Two people have died.
	\item  L'oiseau est tomb{\'e}. \\ The bird has fallen.
	\item  Elle a essay{\'e}. \\ She has tried.
	\item  Il m'a donn{\'e} une montre. \\ He has given me a watch.
	\item  Elle a conduit sa voiture. \\ She has driven her car.
	\item  Son p{\`e}re a disparu. \\ His father has disappeared.
	\item  Ils n'ont jamais cuisin{\'e}. \\ They have never cooked.
	\item  Ils ont port{\'e} leurs chapeaux. \\ They have worn their hats.
	\item  C'est la robe qu'elle a port{\'e}e hier. \\ That is the dress that she had on yesterday.
	\item  Les sandwiches, nous les avons rendus. \\ We have returned the sandwiches.
	\item  Nous avons r{\'e}ussi ! \\ We have succeeded !
	\item  Elle a pris une pomme verte. \\ She has taken a green apple.
	\item  J'en ai entendu assez. \\ I have heard enough.
	\item  Je l'ai vu r{\'e}cemment. \\ I have seen it recently.
	\item  Personne ne m'a cru au d{\'e}but. \\ Nobody believed me at first.
\end{itemize}


\pagebreak
\subsection{Directions}

\begin{itemize}
  \item  Je troune {\`a} droite. \\ I turn right.
	\item  Je vois l'entr{\'e}e. \\ I see the entrance.
	\item  Elle est au centre du village. \\ It is in the center of the village.
	\item  Quelle est ta position ? \\ What is your position?
	\item  Sur ta gauche ! \\ On your left!
	\item  Elle fait un pas dans le jardin. \\ She steps into the garden.
	\item  Il y a un panneau rouge avec une croix blanche. \\ There is a red sign with a white cross.
	\item  La sortie est ici. \\ The exit is here.
	\item  Nous allons vers l'est. \\ We are going eastward.
	\item  Mon magazine est au fond de mon sac. \\ My magazine is on the bottom of my bag.
	\item  Je dois prendre la direction inverse. \\ I have to take the opposite direction.
	\item  Plan du ch{\^a}teau \\ castle map
	\item  Le d{\'e}part est {\`a} Marseille. \\ The departure is in Marseille.
\end{itemize}


\pagebreak
\section{Compound Past 2}

The imparfait and pass{\'e} compos{\'e} can work together in the same sentence. A verb in the imparfait may be used as a background for an action given by a verb in the pass{\'e} compos{\'e}.

\begin{itemize}
  \item  Elle chantait quand elle est arriv{\'e}e. \\ She was singing when she arrived.
	\item  Vous m'avez t{\'e}l{\'e}phon{\'e} pendant que je d{\^{\i}}nais. \\ You called me while I was having dinner.
	\item  Il dormait quand il a entendu un bruit. \\ He was sleeping when he heard a noise.
	\item  Je marchais quand je suis tomb{\'e}. \\ I was walking when I fell.
\end{itemize}

Remember that while you shouldn't use English continuous tenses for stative verbs (such as ``to be''), any French verb can take the imparfait. Thus, you may often need to translate the imparfait into the English preterit when dealing with verbs that describe background feelings or states.

\begin{itemize}
  \item  Je le savais mais je l'ai oubli{\'e}. \\ I knew it but I forgot it. (Not ``was knowing''.)
	\item  Je connaissais l'histoire qu'elle a racont{\'e}e hier. \\ I knew the story she told yesterday.
	\item  Je le comprenais, alors je l'ai accept{\'e}. \\ I understood it, so I accepted it.
\end{itemize}

\subsubsection{{\^E}tre and Direct Objects}

Six {\^e}tre verbs can be used transitively with a direct object: monter, descendre, sortir, rentrer, retourner, and passer. When used transitively, they switch from {\^e}tre to take avoir as an auxiliary.

\begin{itemize}
  \item  Je suis mont{\'e}(e). \\ I went up.
	\item  J'ai monté les valises. \\ I brought up the suitcases.
	\item  Il est sorti. \\ He left.
	\item  Il a sorti son portefeuille. \\ He took out his wallet.
	\item  Sur ta gauche ! \\ On your left!
	\item  Septembre est pass{\'e}. \\ eptember has passed.
	\item  J'ai passé trois heures ici. \\ I spent three hours here.
\end{itemize}

Notice that the transitive versions of these verbs have a different meaning than the intransitive versions.

\subsubsection{Past Participles as Adjectives}

Just like in English, past participles can be used as adjectives in French.

\begin{itemize}
  \item  La baguette grill{\'e}e. \\ The toasted baguette
	\item  Des biens vendus. \\ Sold goods
	\item  Elle est mari{\'e}e. \\ She is married.
	\item  C'est du temps perdu. \\ It is lost time.
\end{itemize}

\subsubsection{Advanced Participle Agreement}

You learned in the first compound verb lesson that participles that follow an avoir auxiliary are invariable unless a direct object (often a pronoun) precedes the verb.

\begin{itemize}
  \item  Voici nos livres. Je les ai achet{\'e}s hier. \\ Here are our books. I bought them yesterday.
	\item  O{\`u} est leur voiture ? Ils l'ont vendue ? \\ Where is their car? Did they sell it?
	\item  C'est la fille que j'ai vue. \\ She is the girl that I saw.
\end{itemize}

An avoir participle also agrees with any form of quel + a noun as long as the noun is the object of the compound verb.

\begin{itemize}
  \item  Quelle femme avez-vous vue ? \\ Which woman did you see?
	\item  Quels bonbons a-t-il achet{\'e}s ? \\ Which candies did he buy?
\end{itemize}

This is also true for lequel (plus its other forms) and combien.

\begin{itemize}
  \item  Laquelle des filles as-tu vue ? \\ Which of the girls did you see?
	\item  Lesquelles de ces chemises a-t-il aim{\'e}es ? \\ Which of those shirts did he like?
	\item  Combien de robes ta fille a-t-elle achet{\'e}es? \\ How many dresses did your daughter buy?
\end{itemize}

Participles do not agree with indirect objects, y, nor en.

\begin{itemize}
  \item  Je leur ai parl{\'e}. \\ I talked to them.
	\item  J'y ai pens{\'e}. \\ I thought about it.
	\item  Nous en avons vendu. \\ We have sold some.
\end{itemize}

\subsubsection{C'est in the Compound Past}

In the present indicative tense, c'est can be used to identify or describe nouns. In the pass{\'e} compos{\'e}, {\^e}tre takes avoir as an auxiliary. One consequence of this is that ce actually becomes ç' because it must elide before the vowel beginnings of all forms of avoir while still retaining its original soft consonant sound.

\begin{itemize}
  \item  {\c C}'a {\'e}t{\'e} un succ{\`e}s ! \\ This has been a success!
	\item  {\c C}'a {\'e}t{\'e} un d{\'e}sastre ! \\ This has been a disaster!
\end{itemize}

Since this form is somewhat awkward, many Francophones prefer to use the imparfait instead.

\begin{itemize}
  \item  C'{\'e}tait tr{\`e}s agr{\'e}able. \\ That was very pleasant.
	\item  C'{\'e}tait tr{\`e}s bon pour l'{\'e}conomie. \\ This was very good for the economy.
\end{itemize}

In informal writing, you may also see the ungrammatical form Ça a {\'e}t{\'e}. When spoken, both ``A'' sounds fuse into one long vowel. Erudite Francophones may also use ce fut as a subsitute. This alternative uses the pass{\'e} simple tense, one of French's literary tenses.

\begin{itemize}
  \item  Ce fut bref mais intense ! \\ That was short but intense!
	\item  Ce fut une ann{\'e}e tr{\`e}s int{\'e}ressante. \\ This has been a very interesting year.
\end{itemize}

\begin{itemize}
  \item  Tu as beaucoup chang{\'e}. \\ You have changed a lot.
	\item  Il a eu un chien. \\ He has had a dog. 
	\item  Nous avons v{\'e}cu au Canada. \\ We have lived in Canada.
	\item  J'ai compris. \\ I did understand.
	\item  Je suis retr{\'e}e tr{\`e}s tard. \\ I have returned very late.
	\item  Elle est rent{\'e}e. \\ She has returned.
	\item  Si, j'ai voulu cette voiture. \\ Yes, I have wanted that car. (``Si'' can be `yes' in a response to a question.)
	\item  Quelle page as-tu apprise ? \\ Which page have you learned?
	\item  D{\'e}sol{\'e}, ce n'est pas la boisson que j'ai command{\'e}e. \\ Sorry, this is not the drink that I ordered.
\end{itemize}

\pagebreak
\subsection{Numbers 2}

UGH.  There are so many notes!  74 = [sixty fourteen]?  WTF?

\begin{itemize}
  \item  \Blue{le chiffre} \\ the figure
  \item  Quelle est \Red{la somme} ? \\ What is the sum?
	\item  Trois et cinq sont des nombres. \\ 3 and 5 are numbers.
	\item  Une \Red{dizaine} de canards \\ a dozen ducks
	\item  Il est cinq heures trente. \\ It is 5:30.
	\item  C'est un nombre {\`a} deux chiffres. \\ It is a two-digit number.
	\item  Quarante-sept ans \\ 47 years
	\item  Cinquante personnes travaillent ici. \\ 50 people work here.
	\item  J'ai soixante-dix amis. \\ I have 70 friends.
	\item  Notre oncle a soixante-et-onze ans. \\ Our uncle is 71 years old.
	\item  J'ai quatre-vingts chemises. \\ I have 80 shirts.
	\item  Votre grand-p{\`e}re a quarte-vingt-un ans. \\ Your grandfather is 81 years old.
	\item  Notre grand-m{\`e}re a quatre-vingt-neuf ans. \\ Our grandmother is 89 years old.
	\item  Elles parlent pendant quarte-vingt-dix minutes. \\ They speak for 90 minutes.
	\item  Nous avons cent ours ! \\ We have 100 bears!
	\item  Je connais des centaines de personnes. \\ I know hundreds of people.
	\item  Nous avons un mille pi{\`e}ces. \\ We have 1000 pieces.
	\item  un milliard \\ one billion
\end{itemize}


\pagebreak
\subsection{Feelings}

\begin{itemize}
  \item  Tu as des \Red{pens{\'e}es} bizarres. \\ You have weird thoughts
	\item  avec \Blue{plaisir} \\ with pleasure
	\item  Le gar{\c c}on a \Red{peur} des chiens. \\ The boy is afraid of dogs.
	\item  un grand moment de \Blue{bonheur} \\ a big moment of happiness
	\item  Oui, avec joie. \\ Yes, with joy.
	\item  \Blue{Le d{\'e}sir} et \Red{la peur} \\ the desire and the fear
	\item  Je connais ce \Blue{sentiment}. \\ I know that feeling.
	\item  J'ai r{\^e}v{\'e} d'elle. \\ I dreamed of her.
	\item  Il est temps pour \Red{la paix}. \\ It is time for peace.
	\item  Tu es notre seul \Blue{espoir}. \\ You are our only hope.
	\item  J'aime \Blue{le go{\^u}t} du chocolat. \\ I like the taste of chocolate.
	\item  Elle joue avec mon sentiments. \\ She plays with my emotions.
	\item  Je suis une personne tr{\`e}s triste. \\ I am a very sad person.
	\item  Nous avons honte. \\ We are ashamed.
	\item  C'est seulement son \Red{imagination}. \\ It is only his imagination.
	\item  \Red{La douleur} est l{\'e}g{\`e}re. \\ The pain is light.
	\item  Je ne suis pas en col{\`e}re. \\ I am not angry.
	\item  Je suis tr{\`e}s heureux de vous voir. \\ I am very happy to see you.
	\item  Je crois en l'amiti{\'e}. \\ I believe in friendship.
	\item  Cette nouvelle coupe de cheveux est une \Red{horreur} ! \\ That new haircut is horrible!
	\item  Cet enfant a un joli \Blue{sourire}. \\ That child has a pretty smile.
	\item  Elle me rend heureuse. \\ She makes me happy.
	\item  Je suis si heureux. \\ I am so happy.
	\item  Je hais les l{\'e}gumes. \\ I hate vegetables.
	\item  Il a envie de pleurer. \\ He feels like crying.
	\item  Elle a beaucoup de souvenirs. \\ She has a lot of memories.
	\item  Et pourquoi souffrir ? \\ And why suffer?
	\item  Je suis de bonne humeur. \\ I am in a good mood.
	\item  Le vin est doux. \\ The wine is sweet.
	\item  Si ma m{\'e}moire est bonne. \\ If my memory is good.
	\item  Elle a une bonne \Red{odeur}. \\ She has a good scent.
	\item  Ton chapeau est-il doux ? \\ Is your hat soft?
	\item  Dans tes r{\^e}ves ! \\ In your dreams!
	\item  Les parents sont g{\^e}n{\'e}s. \\ The parents are embarrassed.
	\item  Je ne suis pas inquiet. \\ I am not worried.
	\item  Il a sommeil. \\ He is sleepy.
	\item  \Red{La confiance} est importante. \\ Confidence is important.
	\item  Elle n'est pas fatigu{\'e}e. \\ She is not tired.
	\item  C'est une personne de confiance. \\ She is trustworthy.
	\item  Je suis s{\^u}r. \\ I am sure
	\item  Elle doit te ha{\"i}r. \\ She must hate you.
	\item  Le lion a soif. \\ The lion is thirsty.
	\item  Avez-vous faim ? \\ Are you hungry?
	\item  Elle est amoureuse de sa voiture. \\ She is in love with her car.
	\item  Ce cheval est nul ! \\ That horse is useless!
	\item  {\c C}a devient ennuyeux. \\ That is becoming boring.
	\item  C'est tout {\`a} fait juste. \\ It is absolutely fair.
	\item  Il est amoureux d'une femme qui ne l'aime pas. \\ He is in love with a woman who does not love him.
\end{itemize}


\pagebreak
\subsection{Possessives 3}

Possessive pronouns replace a possessive adjective + a noun. Like most other pronouns, they agree in gender and number with the noun they replace. 

\begin{itemize}
  \item  Est-ce ton chapeau ? \\ Is that your hat?
	\item  Oui, c'est le mien. \\ Yes, it's mine.
\end{itemize}

Possessive pronouns take different forms depending on how many things are possessed. First, let's take another look at the forms used when a single thing is possessed.

\begin{center}\begin{tabular}{|c|c|c|c|c|}
\hline
\textbf{Owners} & \textbf{Person} & \textbf{English} & \textbf{Masculine Singular} & \textbf{Feminine Singular} \\ \hline
singular        & 1st             & mine             & le mien                     & la mienne                  \\ \hline
singular        & 2nd             & yours            & le tien                     & la tienne                  \\ \hline
singular        & 3rd             & his/hers         & la sien                     & la sienne                  \\ \hline
plural          & 1st             & ours             & le n{\^o}tre                    & la n{\^o}tre                   \\ \hline
plural          & 2nd             & yours            & le v{\^o}tre                    & la v{\^o}tre                   \\ \hline
plural          & 3rd             & theirs           & le leur                     & la leur                    \\ \hline
\end{tabular}\end{center}

To change these to the forms used when multiple things are possessed, simply add an -s to the end of the pronoun and change the definite article to les.

\begin{center}\begin{tabular}{|c|c|c|c|c|}
\hline
\textbf{Owners} & \textbf{Person} & \textbf{English} & \textbf{Masculine Singular} & \textbf{Feminine Singular} \\ \hline
singular        & 1st             & mine             & les miens                     & les miennes                  \\ \hline
singular        & 2nd             & yours            & les tiens                     & les tiennes                  \\ \hline
singular        & 3rd             & his/hers         & les siens                     & les siennes                  \\ \hline
plural          & 1st             & ours             & les n{\^o}tres                    & les n{\^o}tres                   \\ \hline
plural          & 2nd             & yours            & les v{\^o}tres                    & les v{\^o}tres                   \\ \hline
plural          & 3rd             & theirs           & les leurs                     & les leurs                    \\ \hline
\end{tabular}\end{center}

Note that the plural forms here are invariable with gender.

\begin{itemize}
  \item  Ces enfants sont les miens. \\ These (or ``those'') children are mine.
	\item  Ce sont les tiens. \\ They are yours.
	\item  Ces photos sont les siennes. \\ These photos are his (or ``hers'').
	\item  Ces jupes sont les leurs. \\ Those skirts are theirs.
\end{itemize}

Possessive pronouns act like modified nouns, so you must use ce/c' when referring to them with {\^e}tre.

\begin{itemize}
  \item  Est-ce ton fils ? \\ Is he your son?
	\item  Oui, c'est le mien. (Not il est) \\ Yes, he is mine.
	\item  Est-ce que ce sont tes filles ? \\ Are they your daughters?
	\item  Oui, ce sont les miennes. (Not elles sont) \\ Yes, they are mine.
\end{itemize}

\begin{itemize}
  \item  J'ai ferm{\'e} ta chambre mais pas les leurs. \\ I have closed your room but not theirs.
	\item  Ceci est un cadeau pour vous et les v{\^o}tres. \\ Here is a gift for you and yours.
	\item  Peut-il apporter ses photos et les leurs ? \\ Can he bring his photos and theirs?
	\item  Il vient avec ses enfants et nous avec les n{\^o}tres. \\ He comes with his children and we with ours.
	\item  Voici tes cl{\'e}s et voil{\`a} les n{\^o}tres. \\ Here are your keys and there are ours.
\end{itemize}


\pagebreak
\subsection{Demonstratives 3}

Demonstrative pronouns (e.g. ``this one'' or ``those'') replace and agree with a demonstrative adjective + noun.  You learned four such pronouns in ``Demonstratives 2''.

\begin{center}\begin{tabular}{|c|c|c|}
\hline
\textbf{Type}      & \textbf{Adjective + Noun $\Rightarrow$ Pronoun} & \textbf{English}         \\ \hline
masculine singular & ce + noun $\Rightarrow$ celui                   & the one / this / that    \\ \hline
masculine plural   & ces + noun $\Rightarrow$ ceux                   & the ones / these / those \\ \hline
feminine singular  & cette + noun $\Rightarrow$ celle                & the one / this / that    \\ \hline
feminine plural    & ces + noun $\Rightarrow$ celles                 & the ones / these / those \\ \hline
\end{tabular}\end{center}

Demonstratives like ce and celui are ambiguous and can mean either ``this'' or ``that''. To remove this ambiguity, you can add a suffix to the end of each pronoun. Add -ci for ``this/these'' and -l{\`a} for ``that/those''.

\begin{itemize}
  \item  Tu veux celui-ci. \\ You want this one.
	\item  Je pr{\'e}f{\`e}re celle-l{\`a}. \\ I prefer that one.
	\item  Celles-ci sont noires. \\ These are black.
	\item  Elle n'aime pas celles-l{\`a}. \\ She doesn't like those.
\end{itemize}

These suffixes can also be used with demonstrative adjectives in many situations.

\begin{itemize}
  \item  Je suis tr{\`e}s occup{\'e} ces jours-ci. \\ I am very busy these days.
	\item  Ils vous ont vus ce jour-l{\`a}. \\ They saw you that day.
	\item  Le magasin est-il sur ce c{\^o}t{\'e}-ci de la rue ? \\ Is the store on this side of the street?
	\item  Elle conna{\^{\i}}t ce type-l{\`a}. \\ She knows that guy.
\end{itemize}

In conversations, be aware that using demonstrative pronouns like celui-l{\`a} to refer to people who aren't present can be considered condescending.


\pagebreak
\subsection{Adjectives 4}

The French past participle, which you learned in ``Verbs: Compound Past'', can often be used as an adjective. Conveniently, this also occurs in English, though we may sometimes use the present participle instead of the past.

\begin{itemize}
  \item  L'homme fatigu{\'e} veut dormir. \\ The tired man wants to sleep.
	\item  L'examen est termin{\'e}. \\ The test is finished.
	\item  Je ne suis pas occup{\'e}. \\ I am not busy.
	\item  On va parler avec les parties int{\'e}ress{\'e}es. \\ We will speak with the interested parties.
\end{itemize}


\subsubsection{Neuf}

The adjective neuf (``new'') describes something that has just been created or manufactured. Don't confuse it with nouveau, which describes something that has just been acquired by a new owner but may already be quite old. Remember that nouveau becomes nouvel in front of vowel sounds.

\begin{itemize}
  \item  J'ach{\`e}te seulement des sous-v{\^e}tements neufs. \\ I only buy new underwear.
	\item  Cette voiture est flambant neuve. \\ This car is brand-new.
	\item  Voici ma nouvelle montre ancienne. \\ Here's my new antique watch.
	\item  J'aime ton nouvel appartement. \\ I like your new apartment.
\end{itemize}

While neuf (new) and neuf (9) are homonyms, you can often distinguish them based on context. For instance, neuf (9) comes before its noun, isn't accompanied by any articles, and is invariable.

\begin{itemize}
  \item  J'ai neuf livres. \\ I have nine books.
	\item  J'ai des livres neufs. \\ I have new books.
\end{itemize}

\begin{center}\begin{tabular}{l|l||l|l||l|l}
\textbf{Fran{\c c}ais} & \textbf{English} & \textbf{Fran{\c c}ais} & \textbf{English}  & \textbf{Fran{\c c}ais} & \textbf{English} \\ \hline
futur & future & super & great & certain & certain \\
r{\'e}cent & recent/new & actuels & actual & d{\'e}licieuse & delicious \\ 
prochain & next & al{\'e}atoire & random & fragile & fragile \\
agr{\'e}able & pleasant & g{\'e}n{\'e}ral & general & original & original \\
inconnu & unknown & nouvelle & new & {\'e}troit & tight/narrow \\
magique & magical & pr{\'e}sent & present/here & exact & number \\
magnifique & magnificent/gorgeous & technique & technical & curieux & curious \\
universel & universal & national & national & rare & rare \\
autre & another/other & double & double & suivante & following \\
brillant & shiny/brilliant & occup{\'e} & busy & utile & useful \\
culturel & cultural & physique & physical & pur & pure \\
demi- & half (of a) & n{\'e}essaire & necessary & faible & weak \\
excellente & excellent & frais & fresh & extraordinaire & extraordinary \\
mince & thin & moderne & modern & {\'e}trang{\`e} & strange \\
gratuit & free & immense & immense & confortable & comfortable \\
international & international & sup{\'e}rieur & superior & lente & slow \\
mondial & global & publique & public & int{\'e}ress{\'e} & interested \\
neuf & new & extr{\^e}me & extreme & sympathique & nice/sympathetic \\
positif & positive & ouvert & open & court & short \\
n{\'e}gatif & negative & capable & capable & grave & serious \\
professionnel & professional & classique & classic & vrai & true/genuine \\
%naturel & natural \\
\end{tabular}\end{center}

\begin{itemize}
  \item  Je veux une autre chose. \\ I want something else.
	\item  En g{\'e}n{\'e}ral c'est vert. \\ Generally it is green.
	\item  Je ne suis pas occup{\'e} maintentant. \\ I am not currently busy.
	\item  Je suis certain de cela. \\ I am certain of that.
	\item  Il termine sa carri{\`e}re l'ann{\'e}e suivante. \\ He finishes his career next year.
\end{itemize}


\pagebreak
\subsection{Pronouns 2}

French has three sets of personal object pronouns: direct object pronouns (from ``Pronouns 1''), indirect object pronouns, and disjunctive pronouns.

\begin{center}\begin{tabular}{|c|c|c|c|}
\hline
\textbf{English} & \textbf{Direct Object} & \textbf{Indirect Object} & \textbf{Disjunctive} \\ \hline
me               & me                     & me                       & moi                  \\ \hline
you (singular)   & te                     & te                       & toi                  \\ \hline
him              & le                     & \textbf{lui}             & \textbf{lui}         \\ \hline
her              & la                     & \textbf{lui}             & \textbf{elle}        \\ \hline
us               & nous                   & nous                     & nous                 \\ \hline
you (plural)     & vous                   & vous                     & vous                 \\ \hline
them (masculine) & les                    & \textbf{leur}            & \textbf{eux}         \\ \hline
them (feminine)  & les                    & \textbf{leur}            & \textbf{elles}       \\ \hline
\end{tabular}\end{center}

Notice that only the third-person pronouns differ between direct and indirect objects.

\subsubsection{Indirect Objects}

As you learned in ``Verbs: Present 2'', indirect objects are nouns that are indirectly affected by a verb; they are usually introduced by a preposition.

\begin{itemize}
  \item  Il {\'e}crit une lettre {\`a} Mireille. \\ He is writing a letter to Mireille.
	\item  Vous pouvez parler aux juges. \\ You can talk to the judges.
	\item  Elle parle de son amie. \\ She is talking about her friend.
\end{itemize}

A personal indirect object pronoun can replace {\`a} + indirect object. For instance, the first two examples above could be changed to the following:

\begin{itemize}
  \item  Il lui {\'e}crit une lettre. \\ He is writing a letter to her.
	\item  Vous pouvez leur parler. \\ You can talk to them.
\end{itemize}

Also, il faut can take an indirect object pronoun to specify where the burden falls.

\begin{itemize}
  \item  Il lui faut manger. \\ He has to eat. / She has to eat.
	\item  Il nous faut le croire. \\ We have to believe it/him. / It is necessary for us to believe it/him.
\end{itemize}

\subsubsection{Disjunctive Pronouns}

Disjunctive pronouns (a.k.a. stressed or tonic pronouns) must be used in certain situations. For instance, only disjunctive pronouns can follow prepositions.

\begin{itemize}
  \item  Il parle avec toi. \\ He speaks with you.
	\item  Elle p{\`e}se moins que moi. \\ She weighs less than me.
	\item  Ils sont rentr{\'e}s chez eux. \\ They returned home.
	\item  C'est pour lui. \\ This is for him.
\end{itemize}

Note that lui can be masculine or feminine when it's an indirect object, but it can only be masculine when it's disjunctive.

\begin{itemize}
  \item  Je lui parle. (indirect object) \\ I am talking to him/her.
	\item  Je parle de lui. (disjunctive) \\ I am talking about him.
	\item  Je parle d'elle. (disjunctive) \\ I am talking about her.
\end{itemize}

The construction {\^e}tre + {\`a} + disjunctive pronoun indicates possession.

\begin{itemize}
  \item  Le livre est {\`a} moi. \\ The book is mine.
	\item  Celui-l{\`a} est {\`a} toi. \\ That one is yours.
	\item  Ceux-l{\`a} sont {\`a} eux. \\ Those are theirs.
\end{itemize}

However, using {\`a} + pronoun is incorrect when a verb can accept a preceding pronoun.

\begin{itemize}
  \item  Incorrect: Je parle à lui.
	\item  Correct: Je lui parle.
\end{itemize}

Disjunctive pronouns are also used for emphasis, for multiple subjects, or in sentence fragments without a verb.

\begin{itemize}
  \item  Moi ? Je l'aime. \\ Me? I love him.
	\item  Lui et elle mangent. \\ He and she are eating.
	\item  Vous aussi. \\ You, too.
\end{itemize}

\subsubsection{Indirect Objects and Y}

For most verbs, personal indirect object pronouns like lui can only refer to people or animals, but you can use the adverbial pronoun y for inanimate things.

\begin{itemize}
  \item  Elle ressemble {\`a} sa m{\`e}re. $\Rightarrow$ Elle lui ressemble. \\ She resembles her.
	\item  {\c C}a ressemble {\`a} un robot. $\Rightarrow$ {\c C}a y ressemble. \\ It resembles it.
\end{itemize}

Some verbs allow personal pronouns like lui to be used with anything you can personify. These verbs are dire {\`a}, demander {\`a}, donner {\`a}, parler {\`a}, t{\'e}l{\'e}phoner {\`a}, and ressembler {\`a}.

\begin{itemize}
  \item  L'enfant parle {\`a} son jouet. $\Rightarrow$ L'enfant lui parle.
	\item  Je demande un renseignement {\`a} la banque. $\Rightarrow$ Je lui demande un renseignement.
\end{itemize}

Some French expressions don't allow any preceding indirect objects, notably {\^e}tre {\`a}, faire attention {\`a}, s'habituer {\`a}, penser {\`a}, revenir {\`a}, and tenir {\`a}.

\begin{itemize}
  \item  Tu fais attention {\`a} elle. (Not Tu lui fais...) \\ You are paying attention to her.
	\item  Il pense {\`a} elle. (Not Il lui pense...) \\ He thinks of her.
\end{itemize}

Remember that y can also refer to locations.

\begin{itemize}
  \item  J'y vais. \\ I'm going there.
	\item  Il y {\'e}tait. \\ He was there.
\end{itemize}

\subsubsection{Quelque}

The indefinite adjective quelque (``some'') can be combined with pronouns or nouns to create indefinite pronouns. For instance, chose means ``thing'', so quelque chose means ``something''.

\begin{itemize}
  \item  Nous {\'e}crivons quelque chose. \\ We are writing something.
	\item  Je veux manger quelque chose. \\ I want to eat something.
\end{itemize}

Quelque can combine and elide with un (``one'') to give quelqu'un (``someone''), which is singular.

\begin{itemize}
  \item  Quelqu'un est ici. \\ Someone is here.
	\item  Je connais quelqu'un au restaurant. \\ I know someone at the restaurant.
\end{itemize}

For multiple people or things, use the plural forms quelques-uns (masc) and quelques-unes (fem), which are normally translated as``a few'', or perhaps ``some''.  While quelqu'un only refers to people, quelques-un(e)s can refer to anything.

\begin{itemize}
  \item  Ce sont quelques-uns de nos meilleurs amis. \\ These are a few of our best friends.
	\item  Quelques-unes de ces questions sont difficiles. \\ Some of these questions are difficult.
\end{itemize}

\begin{itemize}
  \item  Je suis d'accord avec eux. \\ I agree with them.
	\item  Pourquoi eux ? \\ Why them?
	\item  Merci pour tout. \\ Thanks for everything.
	\item  Il aide souvent les autres. \\ He often helps others.
	\item  Nous sommes sept, dont moi. \\ We are seven, including me.
	\item  Aucune ! \\ None!
	\item  Je peux manger n'importe quoi. \\ I can eat no matter quoi.
	\item  J'ai le m{\^e}me. \\ I have the same.
	\item  Quelqu'un me disait {\c c}a. \\ Someone said that to me.
	\item  Elle n'a pas la m{\^e}me que nous. \\ She does not have the same as us.
\end{itemize}


\pagebreak
\subsection{Infinitives 2}

As you learned in Verbs: Infinitive 1, verbs in the infinitive mood are not conjugated and are not paired with a subject pronoun. The infinitive is more versatile in French than in English. For instance, an infinitive can act as a noun (where gerunds might be used in English).

\begin{itemize}
  \item  Faire du caf{\'e} est facile. \\ Making coffee is easy.
	\item  Cuisiner et nettoyer sont ses responsabilit{\'e}s. \\ Cooking and cleaning are his responsibilities.
\end{itemize}

In French, the infinitive is also used for generalized instructions like those in product manuals, public notices, recipes, and proverbs.

\begin{itemize}
  \item  Lire le mode d'emploi avant utilisation. \\ Read the instructions before using.
	\item  Garder hors de la port{\'e}e des enfants. \\ Keep out of reach of children.
	\item  Battre les {\oe}ufs. \\ Beat the eggs.
	\item  Vaut mieux pr{\'e}venir que gu{\'e}rir. \\ It is worth more to prevent than to cure.
\end{itemize}

Conjugated verbs are the only verbs that can appear inside a negation, so when a negation is used with an infinitive, both parts of the negation come before the infinitive.

\begin{itemize}
  \item  Ne pas entrer. \\ Do not enter.
	\item  Ne rien administrer par la bouche. \\ Do not administer by mouth.
\end{itemize}

An infinitive can also be used to pose a question. These sentences may not translate literally to English.

\begin{itemize}
  \item  Comment obtenir ça ? \\ How does one obtain that?
	\item  Qui croire ? \\ Whom should I believe?
	\item  Quoi faire ? \\ What can we do?
	\item  Comment ne pas tomber amoureux d'elle ? \\ How can I not fall in love with her?
\end{itemize}

\subsubsection{Impersonal Expressions}

Recall that the subject in the impersonal construction il est + adjective + de must be a dummy subject. If it's a real subject, you must use {\`a} instead of de.

\begin{itemize}
  \item  Il est impossible de vivre sur cette {\^\i}le. \\ It is impossible to live on that island.
	\item  Il est facile de comprendre le livre. \\ It is easy to understand the book.
	\item  Il est amusant de cuisiner. \\ It is fun to cook. / Cooking is fun.
	\item  Je n'aime pas ce livre. Il est difficile {\`a} comprendre. \\ I don't like this book. It's difficult to understand.
	\item  Ce plat est bon parce qu'il est facile {\`a} cuisiner. \\ This dish is good because it is easy to cook.
	\item  Il est difficile {\`a} faire. \\ It is difficult to do.
\end{itemize}

\subsubsection{Register}

Communication in French can occur at several different levels of formality, which are called registers. Different registers may vary in word choice, sentence structure, and even pronunciation. For instance, the use of liaisons is relatively formal. By comparison, English verbal formality is arguably less intricate.  The most obvious indication of register is pronoun choice. As you know by now, addressing someone with the pronoun vous is considered more formal. This is described by the French verb vouvoyer.

\begin{itemize}
  \item  Il doit vouvoyer son professeur. \\ He must speak formally with his professor.
	\item  Je ne veux pas vouvoyer mes amis. \\ I don't want to address my friends formally.
\end{itemize}

The more familiar tu form should be used with friends, peers, relatives, or children. If you're not sure who's a vous and who's a tu, consider matching the register of your interlocutor. Alternatively, you can directly ask if you can speak informally by using the verb tutoyer.

\begin{itemize}
  \item  On peut se tutoyer ? \\ Can we be on familiar terms?
	\item  Je peux tutoyer mes amis. \\ I can be on familiar terms with my friends.
\end{itemize}

Question structure is another key ingredient of register. Inversions are considered formal.

\begin{itemize}
  \item  Pouvons-nous nous tutoyer ? \\ Can we be on familiar terms? (said very formally.)
	\item  Comment allez-vous ? \\ How are you?
\end{itemize}

Use the conditional forms of aimer and vouloir for polite requests. More on this in the ``Verbs: Conditional'' unit.

\begin{itemize}
  \item  J'aimerais une tasse de caf{\'e}, s'il vous pla{\^\i}t. \\ I would like a cup of coffee, please.
	\item  Je voudrais vous remercier. \\ I would like to thank you.
\end{itemize}

\subsubsection{Faire vs Rendre}

In Verbs: Present 1, you learned about the causative faire, which can indicate that the subject has directed someone to perform an action. Notice that faire is followed by an infinitive here.

\begin{itemize}
  \item  Je le fais r{\'e}parer. \\ I am having it fixed.
	\item  Elle lui a fait perdre 5 kilos. \\ She made him/her/it lose 5 kilos.
	\item  Je leur ai fait faire de l'exercice. \\ I made them (get some) exercise.
\end{itemize}

The verb rendre (``to make'') can also indicate that the subject has caused something to happen, but it's used with adjectives instead of verbs.

\begin{itemize}
  \item  Elle le rend heureux. \\ She makes him happy.
	\item  {\c C}a me rend fou ! \\ That drives me crazy!
	\item  L'erreur a rendu le texte incomprehensible. \\ The error rendered the text incomprehensible.
\end{itemize}

$$~$$

\begin{itemize}
  \item  Je ne peux pas vous entendre. \\ I cannot understand you.
	\item  J'aime passer du temps avec elle. \\ I like passing time with her.
	\item  Merci de me laisser tranquille ! \\ Thanks for leaving me in peace!
	\item  Il semble nous conna{\^\i}tre. \\ He seems to know us.
	\item  {\`A} suivre \\ to be continued
	\item  Il a peur de tomber. \\ He is afraid of falling.
	\item  Je vais rester trois nuits. \\ I am going to stay three nights.
	\item  Il n'a pas envie de mourir. \\ He does not feel like dying.
	\item  Il est ici pour rester. \\ It is here to stay.
	\item  Je n'ai pas de temps {\`a} perdre. \\ I do not have time to lose.
	\item  {\^E}tes-vous s{\^u}r de vouloir un cheval ? \\ Are you sure you want a horse?
	\item  Oublier le temps. \\ Forget about the time.
	\item  Elle doit revenir dans son pays. \\ She must return to her land.
	\item  Je ne peux pas imaginer {\c c}a. \\ I cannot imagine that.
	\item  Mais qui va garder les enfants ? \\ But who is going to keep the children?
	\item  Il peut compter sur elle. \\ He can count on her.
	\item  Il ne peut pas tuer le poulet. \\ He cannot kill the chicken.
	\item  Ne pas toucher! \\ Do not touch!
	\item  Ils peuvent cr{\'e}er leur entreprise {\`a} tout moment. \\ They can setup their business anytime.
	\item  C'est ma chambre {\'a} coucher. \\ It is my bedroom.
	\item  Il a d{\'e}cid{\'e} d'essayer. \\ He has decided to try.
	\item  Nous mangeons pour vivre, et vivons pour manger. \\ We eat to live, and we live to eat.
\end{itemize}







