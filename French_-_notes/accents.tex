\subsection{Accent Marks}

\subsubsection{Acute Accent}

\begin{center}\begin{tabular}{|c|c|c||c|c|c|} \hline
  \textbf{letter} & \textbf{TeX} & \textbf{Windows} & \textbf{letter} & \textbf{TeX} & \textbf{Windows} \\ \hline
  {\'E} & \{\textbackslash 'E\} & ALT + 0201 & {\'e} & \{\textbackslash 'e\} & ALT + 130 \\ \hline
\end{tabular}\end{center}

\subsubsection{Grave Accent}

\begin{center}\begin{tabular}{|c|c|c||c|c|c|} \hline
  \textbf{letter} & \textbf{TeX} & \textbf{Windows} & \textbf{letter} & \textbf{TeX} & \textbf{Windows} \\ \hline
  {\`A} & \{\textbackslash `A\} & ALT + 0192 & {\`a} & \{\textbackslash `a\} & ALT + 133 \\ \hline
  {\`E} & \{\textbackslash `E\} & ALT + 0200 & {\`e} & \{\textbackslash `e\} & ALT + 138 \\ \hline
  {\`U} & \{\textbackslash `U\} & ALT + 0217 & {\`u} & \{\textbackslash `u\} & ALT + 151 \\ \hline
\end{tabular}\end{center}

\subsubsection{Cedilla}

\begin{center}\begin{tabular}{|c|c|c||c|c|c|} \hline
  \textbf{letter} & \textbf{TeX} & \textbf{Windows} & \textbf{letter} & \textbf{TeX} & \textbf{Windows} \\ \hline
  {\c C} & \{\textbackslash c C\} & ALT + 0199 & {\c c} & \{\textbackslash c c\} & ALT + 135 \\ \hline
\end{tabular}\end{center}

\subsubsection{Curcumflex}

\begin{center}\begin{tabular}{|c|c|c||c|c|c|} \hline
  \textbf{letter} & \textbf{TeX} & \textbf{Windows} & \textbf{letter} & \textbf{TeX} & \textbf{Windows} \\ \hline
  {\^A} & \{\textbackslash\textasciicircum A\} & ALT + 0194 & {\^a} & \{\textbackslash\textasciicircum a\} & ALT + 131 \\ \hline
  {\^E} & \{\textbackslash\textasciicircum E\} & ALT + 0202 & {\^e} & \{\textbackslash\textasciicircum e\} & ALT + 136 \\ \hline
  {\^I} & \{\textbackslash\textasciicircum I\} & ALT + 0206 & \^{\i} & \textbackslash\textasciicircum \{\textbackslash i\} & ALT + 140 \\ \hline
  {\^O} & \{\textbackslash\textasciicircum O\} & ALT + 0212 & {\^o} & \{\textbackslash\textasciicircum o\} & ALT + 147 \\ \hline
  {\^U} & \{\textbackslash\textasciicircum U\} & ALT + 0219 & {\^u} & \{\textbackslash\textasciicircum u\} & ALT + 150 \\ \hline
\end{tabular}\end{center}

\subsubsection{Trema}

\begin{center}\begin{tabular}{|c|c|c||c|c|c|} \hline
  \textbf{letter} & \textbf{TeX} & \textbf{Windows} & \textbf{letter} & \textbf{TeX} & \textbf{Windows} \\ \hline
  {\"A} & \{\textbackslash "A\} & ALT + 0196 & {\"a} & \{\textbackslash "a\} & ALT + 132 \\ \hline
  {\"E} & \{\textbackslash "E\} & ALT + 0203 & {\"e} & \{\textbackslash "e\} & ALT + 137 \\ \hline
  {\"I} & \{\textbackslash "I\} & ALT + 0207 & \"{\i} & \textbackslash "\{\textbackslash i\} & ALT + 139 \\ \hline
  {\"O} & \{\textbackslash "O\} & ALT + 0214 & {\"o} & \{\textbackslash "o\} & ALT + 148 \\ \hline
  {\"U} & \{\textbackslash "U\} & ALT + 0220 & {\"u} & \{\textbackslash "u\} & ALT + 129 \\ \hline
\end{tabular}\end{center}

\subsubsection{Miscellaneous}

\begin{center}\begin{tabular}{|c|c|c||c|c|c|} \hline
  \textbf{letter} & \textbf{TeX} & \textbf{Windows} & \textbf{letter} & \textbf{TeX} & \textbf{Windows} \\ \hline
  {\OE} & \{\textbackslash OE\} & ALT + 0140 & {\oe} & \{\textbackslash oe\} & ALT + 0156 \\ \hline
  \guillemotleft & \textbackslash guillemotleft & ALT + 0171 & \guillemotright & \textbackslash guillemotright & ALT + 0187 \\ \hline
\end{tabular}\end{center}


\pagebreak
\subsubsection{Diacritics}

The \textbf{acute accent} ({\'e}) only appears on E and produces a pure [e] that isn't found in English. To make this sound, say the word ``clich{\'e}'', but hold your tongue perfectly still on the last vowel to avoid making a diphthong sound.

The \textbf{grave accent} ({\`e}) can appear on A/E/U, though it only changes the sound for E (to [ɛ], which is the E in "lemon"). Otherwise, it distinguishes homophones like a (a conjugated form of avoir) and {\`a} (a preposition).

The \textbf{cedilla} ({\c c}) softens a normally hard C sound to the soft C in ``cent''. Otherwise, a C followed by an A, O, or U has a hard sound like the C in ``car''.

The \textbf{circumflex} ({\^e}) usually means that an S used to follow the vowel in Old French or Latin. (The same is true of the acute accent.) For instance, {\^i}le was once ``isle''.

The \textbf{trema} ({\"e}) indicates that two adjacent vowels must be pronounced separately, like in No{\"e}l ("Christmas") and ma{\"i}s ("corn").